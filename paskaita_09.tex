\chapter{Žaidimas}

\begin{itemize}
  \item Produkto idėja, kaina. Kuriant, reali.
  \item Produkto tipinis vartotojas.
  \item Galimi produkto pakaitalai, konkurentai.
  \item Politiniai / ekonominiai produkto apribojimai ir kliūtys.
  \item Kaip galima produktą išreklamuoti.
  \item Ilgalaikės produkto vystymosi perspektyvos.
  \item Kokių padalinių reikėtų organizacijai sukurti, pagaminti ir 
    realizuoti produktą nuo idėjos iki rinkos.
\end{itemize}

\chapter{Galia ir valdžia valdymo sistemoje}

Valdžia nuo galios skiriasi tuo, kad valdžia yra formali galios suteikimo
išraiška. Du požiūriai: iš viršaus į apačia ir iš apačios į viršų.

Patariamoji, ta, kuri yra pavaldi tik konkrečiam savo vadovui. 
(Patarėjai.) Jie niekada neturi pavaldinių.

Funkcinė panaši į matricinę struktūrą.



Pusiausvyros: kalėjimo prižiūrėtojo pavyzdys.


Įgaliojimai: teisė padaryti kažką. Linijinis – vadovo funkcinių 
pasidalijimas su pavaldiniais. Aparatinis – kai vadovas suteikia
galimybę jam padėti.



