\chapter{Atlyginimų valdymas}

\section{Kas tai?}

\begin{defn}[Darbo užmokestis]
  Atlyginimas už darbą, darbuotojo atliekamą pagal darbo sutartį,
  kuris apima pagrindinį darbo užmokestį ir visus papildomus
  uždarbius, bet kokiu būdų tiesiogiai darbdavio išmokamus
  darbuotojui už jo atliktą darbą.
\end{defn}

\begin{defn}[Atlyginimo už darbą valdymas]
  FIXME: Čia nėra apibrėžimas.

  Visas atlyginimo už darbą sistemos formavimo, koregavimo arba
  keitimo procesas turi būti valdomas ir nesuvedamas tik į darbo
  užmokesčio apskaičiavimą.
\end{defn}

\section{Atlyginimo už darbą sistemos tikslai}

\begin{itemize}
  \item Padėti organizacijai konkuruoti darbo / prekių rinkose:
    nustačiusi per žemą darbo užmokesčio lygį, organizacija nesugebės
    pritraukti ir išlaikyti reikiamus darbuotojus, o per aukštas
    darbo užmokestis padidins darbo kaštus (ir prekės kainą) lyginant
    su konkurentais.
  \item Garantuoti darbo sąnaudų efektyvumą: pilnai išnaudoti turimą
    darbuotojų potencialą, neturėti perteklinio personalo.
  \item Motyvuoti darbuotojus tinkamai atlikti darbą.
\end{itemize}

\section{Darbo užmokesčio sistemos}

\subsection{Vienetinė darbo užmokesčio sistema}

\begin{equation*}
  \text{darbo užmokestis} = \text{įkainis} \cdot \text{vienetų skaičius}
\end{equation*}

Vienetinės darbo užmokesčio formos atmainos:
\begin{itemize}
  \item tiesioginė;
  \item premijinė;
  \item progresyvinė;
  \item regresyvinė;
  \item diferencijuota (baudų);
  \item fiksuotų priedų ir asmeninių priedų;
  \item akordinė;
  \item netiesioginė.
\end{itemize}

\subsection{Laikinė darbo užmokesčio sistema}

\begin{equation*}
  \text{darbo užmokestis} =
    \text{valandinis tarifinis atlygis} \cdot \text{dirbtų valandų skaičius}
\end{equation*}

\begin{defn}[Tarifinis atlygis (valandos, dienos, mėnesio)]
  Fiksuotas pinigų kiekis už valandos, dienos, mėnesio trukmės darbą
  konkrečioje darbo vietoje, esant normalioms darbo sąlygoms.
\end{defn}

Laikinės darbo užmokesčio sistemos atmainos:
\begin{itemize}
  \item paprasta laikinė;
  \item laikinė premijinė.
\end{itemize}

\begin{defn}[Pareiginė alga]
  Darbuotojui skiriamas fiksuotas pinigų kiekis už darbą einant konkrečias
  pareigas (darbo vietoje) per mėnesį \emph{esant normalioms darbo
  sąlygoms} pagal kurį ir mokamas darbo užmokestis.
\end{defn}

\section{Atlyginimo už darbą struktūra}

Atlyginimas už darbą susideda iš:
\begin{itemize}
  \item pagrindinio užmokesčio – pareiginė alga, tarifinis atlygis;
  \item kintamo užmokesčio – premijos, priemokos, priedai;
  \item netiesioginis užmokestis.
\end{itemize}

\begin{description}
  \item[Priemokos] – kompensacinės išmokos tiesiogiai susijusios su darbu
    (už nukrypimus nuo normalių darbo sąlygų, už darbą švenčių ir
    poilsio dienomis, naktį, už viršvalandžius ir kitais LR įstatymais
    numatytais atvejais).
  \item[Priedai] – skatinamojo pobūdžio išmokos už darbuotojo
    profesionalumą, tiesiogiai susijusį su darbu.
  \item[Premijos] – materialinė paskata už labai gerą darbą, skubių ir
    svarbių užduočių įvykdymą ar ypač reikšmingą organizacijai veiklą.
\end{description}

\section{Papildomas darbo apmokėjimas}

TODO

\section{Atlyginimų politikos}

TODO
