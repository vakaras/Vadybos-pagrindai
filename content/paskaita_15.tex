\chapter{Konfliktai}

\section{Kas yra konfliktas?}

Lotyniškai \emph{conflictus} – susidūrimas.

\begin{defn}[Konfliktas]
  Priešingų, nesuderinamų interesų, pažiūrų, elgesio tendencijų
  susidūrimas, kai kito žmogaus ar grupės pozicija kuriuo nors
  klausimu yra atmetama ir laikoma kliūtimi tolesnei veiklai.
\end{defn}


\begin{defn}[Konfliktas]
  Kova už vertybes ir statusą, valdžią, išteklius. Šios kovos tikslas
  – priešininko neutralizavimas arba sunaikinimas.
\end{defn}

\begin{defn}[Konfliktas]
  Dviejų pusių priešiškumo būsena. (\emph{Wikipedia})
\end{defn}

\begin{defn}[Organizacinis konfliktas]
  Konfliktas, kurį sukelia organizacijos specifinės savybės,
  struktūros ypatybės bei jos sąveika su kitomis organizacijomis.
\end{defn}

\section{Konfliktų rūšys}

\begin{description}
  \item[Užduoties konfliktas] susijęs su darbo turiniu ir tikslais.
  \item[Santykių konfliktas] turi ryšį su žmonių santykiais.
  \item[Proceso konfliktas] susijęs su tuo, kaip atliekamas darbas.
\end{description}

\section{Pagrindinės konfliktų priežastys}

\begin{itemize}
  \item Ištekliai – materialiniai, žmogiškieji, laiko.
  \item Poreikiai – fiziologinių, saugumo, pripažinimo (pvz. plagijavimas)
    ar kitų poreikių netenkinimas (žr.: \nameref{sec:maslow}).
  \item Vertybės – religinių, politinių ir kitų požiūrių nesutapimas.
  \item Informacija – kova dėl noro žinoti, duomenų, informacijos
    slėpimo ar iškraipymo.
\end{itemize}

\section{Konfliktų tipai}

\begin{description}
  \item[Funkciniai (konstruktyvūs)] – sąlygojantys organizacijos efektyvumo
    augimą.
  \item[Disfunkciniai (destruktyvūs)] – sąlygojantys asmeninio
    pasitenkinimo, grupinio bendradarbiavimo ir organizacijos efektyvumo
    mažėjimą.
\end{description}

\section{Konfliktų lygiai}

\begin{itemize}
  \item Asmeninis (vidinis) – dviejų poreikių konfliktas viename žmoguje.
  \item Tarpasmeninis – konfliktas tarp dviejų žmonių.
  \item Asmens ir grupės – kažkas vienas konfliktuoja su daug žmonių
    (pavyzdžiui, pirkėjas su organizacija).
  \item Tarpgrupinis – pavyzdžiui, tarp organizacijų, tarp tautų.
\end{itemize}

\section{Konfliktų savybės}

\xtable
{
  w [ 5 | 5 ]
  a [ p | p ]
  h [ Teigiamos | Neigiamos ]
  %
  e [ Mažėja įtampa tarp konfliktuojančių šalių.
    | Prarandama daug emocinių ir materialinių išteklių. ]
  e [ Gaunama nauja informacija apie oponentą.
    | Blogėja psichologinis klimatas, mažėja drausmė, galimi
      organizacijos narių atleidimai. ]
  e [ Kolektyvas susitelkia prieš bendrą „priešą“.
    | Nugalėtojų grupė suvokiama, kaip priešai. ]
  e [ Skatinami pokyčiai.
    | Dėl pernelyg aktyvaus įsitraukimo į konfliktą nukenčia darbas. ]
  e [ Išnyksta pavaldinių nuolankumo sindromas.
    | Sudėtingas dalykinių santykių atkūrimas po konflikto. ]
}

\section{Vadovavimo konfliktams stiliai}

\begin{itemize}
  \item Prievartinis stilius – viena pusė bando pasiekti savo
    tikslus nekreipdama dėmesio, kokį tai poveikį turės kitai
    pusei, pasitelkdama bendro vadovo valdžią, kaip dominuojančią
    jėgą.
  \item Bendradarbiavimo stilius – bandyti pagelbėti abiem pusėms
    nelendant į konfliktą. Yra bandoma išspręsti problemą ir
    išsiaiškinti, kas konkrečiai skiriasi, o ne suvienodinti skirtingas
    pažiūras.
  \item Prisitaikymo stilius – pasirenka kokią nors sau patogesnę
    pusę. FIXME: Viena iš konfliktuojančiųjų pusių iškelia kitos
    pusės interesus aukščiau savųjų.
  \item Vengimas – laukimas, kol konfliktas išsispręs savaime. Kadangi
    konfliktas gali neišsispręsti ilgai, organizacijos veikla dėl to
    gali apskritai pradėti strigti.
  \item Kompromisinis stilius – vadovas užima tarpininko vaidmenį, kad
    abi pusės susitartų, siūlo kompromisus. FIXME: Kompromisas nuo
    bendradarbiavimo skiriasi tuo, kad kompromiso atveju kiekviena
    šalis ko nors atsisako.
  \item Stimuliavimo – vadovas kursto konstruktyvų konfliktą siekdamas
    geresnių rezultatų.
\end{itemize}

\section{Konflikto sprendimo strategijos}

\subsection{Win-Loose}

Viena pusė laimi, kita – pralaimi.

Konfliktai sprendžiami vadovaujantis principu „arba – arba“, nors tai
dar labiau supriešina konfliktuojančias puses. Pavyzdžiui: į stažuotę
gali važiuoti tik vienas. Čia priemonės pergalei pasiekti nėra
svarbios.

\subsection{Loose-Loose}

Pralaimi abi pusės.

Abi pusės sutinka dalinai atsisakyti trokštamų tikslų. Pavyzdžiui, vadovai
susipyksta matant pavaldiniams, tad praranda dalį pagarbos, autoriteto
jų akyse.

\subsection{Win-Win}

Abi pusės laimi.

Pagrindinis konfliktuojančių pusių uždavinys – rasti sprendimą, kuris
tenkintų abi puses (konsensusas). Pavyzdžiui, turimas apelsinas
pasidalinamas per pusę.
