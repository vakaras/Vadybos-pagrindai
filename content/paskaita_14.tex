\chapter{Komunikacija}

\section{Kas yra komunikacija?}

\begin{defn}[Komunikacija]
  Keitimasis informacija, idėjomis, nuomonėmis ir jausmais.
\end{defn}

\begin{defn}[Komunikacija]
  Pasikeitimas prasme tarp individų, naudojant bendrą simbolių sistemą.
\end{defn}

\begin{defn}[Komunikacija]
  Keitimasis sukurta/sutvarkyta informacija tarp dviejų ar daugiau
  žmonių, siekiant bendro supratimo.
\end{defn}

\section{Komunikacijos modelis}

TODO

Iš esmės tas pats, kuo užsiima kanalinis lygis…

Supaprastintas komunikacijos modelis susideda iš šaltinio, perdavėjo,
signalo, gavėjo ir paskirties. Pilnas komunikacijos modelis
supaprastintąjį papildo pranešimo užkodavimu ir dekodavimu.

\section{Komunikacijos formos}

\begin{description}
  \item[Kalbinė] – tiesioginis bendravimas, grupiniai sprendimai,
    telefoniniai pokalbiai.
  \item[Rašytinė] – korespondencija, ataskaitos, bylos.
  \item[Nežodinė] – mimika, gestai, įvaizdis.
\end{description}

\section{Komunikacijos lygiai}

\begin{description}
  \item[Formali komunikacija] – susijusi su organizacijos struktūra ir
    yra kontroliuojama vadovų. Juda formaliais informacijos kanalais
    – augant organizacijai formalūs kanalai plečiasi.
  \item[Neformali komunikacija] – paprastai nesankcionuota komunikacija
    organizacijos viduje, pavyzdžiui, gandai.
\end{description}

\section{Komunikacija organizacijoje}

\begin{description}
  \item[Vertikali komunikacija] – informacijos perdavimas ir priėmimas
    skirtinguose hierarchijos lygiuose:
    \begin{description}
      \item[iš viršaus į apačią] – paliepimų perdavimas organizacijos
        nariams (darbuotojams);
      \item[iš apačios į viršų] – perduodamos žinios, pateikiamos idėjos
        ar problemos, kurias reikėtų spręsti.
    \end{description}
  \item[Horizontali komunikacija] – komunikacija tarp įvairių organizacijos
    padalinių, komunikacija viename hierarchiniame lygyje. Horizontalios
    komunikacijos tikslai:
    \begin{itemize}
      \item užduočių koordinavimas;
      \item problemų sprendimas;
      \item konfliktų šalinimas.
    \end{itemize}
\end{description}

\section{Komunikacijos kliūtys}

Vertikali komunikacija iš viršaus į apačią:
\begin{itemize}
  \item nepakankama arba neaiški informacija;
  \item informacijos perteklius;
  \item netinkamas laikas;
  \item iškraipymas ir filtravimas;
  \item netinkamas priemonių ir kanalų pasirinkimas.
\end{itemize}

Vertikali komunikacija iš apačios į viršų:
\begin{itemize}
  \item rizika;
  \item iškraipymas;
  \item padėties skirtumai.
\end{itemize}

Horizontali komunikacija:
\begin{itemize}
  \item konkurencija;
  \item specializacija;
  \item motyvacijos trūkumas;
  \item fiziniai barjerai.
\end{itemize}

Tarpasmeninė komunikacija:
\begin{itemize}
  \item suvokimo skirtumas;
  \item semantiniai barjerai;
  \item neverbaliniai trikdžiai;
  \item blogas grįžtamasis ryšys.
\end{itemize}

\section{Pagrindiniai pasikeitimo informacija tobulinimo būdai}

\begin{itemize}
  \item Informacijos srauto reguliavimas.
  \item Valdymo veiksniai.
  \item Atgalinio ryšio sistema.
  \item Pasiūlymų surinkimo sistema.
  \item Informaciniai biuleteniai.
  \item Informacinės technologijos tobulinimas.
\end{itemize}
