\chapter{Vadovavimas}

\section{Kas yra lyderiavimas?}

Lyderis yra žmogus, paskui kurį visada seka bent keletas žmonių.

Lyderis be su pareigom gautų galių gauna dar ir:
\begin{itemize}
  \item ekspertinę (ne visada) ir
  \item patrauklumo.
\end{itemize}

Moralus lyderiavimas: vadovas-lyderis, kuris yra moraliai atsakingas.
Pavyzdžiui, „Body-Shop“ įkūrėja skatina užsiimti socialine atsakomybe.
Moralus lyderis palieka alternatyvas savo pavaldiniams.

Lyderiavimo funkcijos:
\begin{itemize}
  \item užduoties
  \item socialines funkcijas
\end{itemize}
Padalinto lyderiavimo fenomenas: formalus lyderis siekia užvesti siekti
užduoties įvykdymo, o neformalus siekia užtikrinti bendravimą.

\chapter{Vadovai ir lyderiai}

Vadovai (managers): labiau kontroliuojantis, valdantis.

Lyderiai (leaders): skatina inovativumą, ne tik pripažįsta, bet ir vykdo.

\chapter{Lyderių tipai}

Pozityvus lyderis: įdėjinis žmogus, kuris pats tiki savo idėja ir įtikina
kitus. Jis tiki, kad jo pavaldiniai yra patys geriausi.

Makiavelio kūrinyje „Vadovas“ buvo aprašytas žmogus, kuris turėjo sekėjų,
bet buvo neigiama asmenybė.

Makiaveliškas lyderis – žmogus sugebantis manipuliuoti žmonėmis.
Neigiamas lyderis. Pavyzdžiai: Rišeljė, Steve Jobs, 

\chapter{Ūkinė situacija ir lyderiavimas}


