\chapter{Vadovavimas}

\section{Kas yra vadovavimas?}

\begin{defn}[Vadovavimas]
  Grupės ar visos organizacijos narių veiklos nukreipimas, įtakos
  darymas siekiant, kad būtų pasiekti grupės ir organizacijos tikslai.
\end{defn}

Vadovai mėgina įtikinti kitus telktis ir kartu siekti rezultatų,
aiškėjančių iš planavimo bei organizavimo etapų.

\subsection{Vadovavimo stiliai ir lyderiavimas}

\begin{defn}[Vadovavimo stilius]
  Vadovo elgesys komandos narių (pavaldinių) atžvilgiu, siekiant juos
  paveikti ir paskatinti siekti projekto (organizacijos) tikslų.
  Vadovavimo stilių formuoja vadovo vertybių sistema, komandos lūkesčiai ir
  esama situacija.
\end{defn}

Platono valdžios skirstymas (šiek tiek iškreiptai):
\begin{description}
  \item[Timokratija] – vadovas labai ambicingas, tikslas yra valdžios
    siekimas (valdžia dėl valdžios);
  \item[Oligarchija] – valdančiųjų atrinkimas, paremtas turto cenzu
    (turtingųjų valdžia);
  \item[Demokratija] – tautos valdžia, vadovauja tautos išrinktas vadovas;
  \item[Tironija] – žiauraus valdymo forma, kai vadovas savo tikslų
    siekia žiauriais metodais.
\end{description}

Vadovavimo stilių koncepcija išplaukia iš žmogaus požiūrių į elgseną.
D. McGregor X ir Y teorija:
\begin{description}
  \item[X] (tokį požiūrį turintis žmogus yra autokratas):
    \begin{enumerate}
      \item žmogus iš prigimties nemėgsta darbo ir norės jo išvengti;
      \item kadangi žmogus nemėgsta darbo, apeliuoti į jo sąžinę yra
        beviltiška, todėl reikia jį priversti, kontroliuoti, nukreipti,
        gąsdinti baudomis tam, kad būtų priverstas dirbti siekdamas
        organizacijos tikslų;
      \item vidutinis žmogus tikisi, kad jam vadovaus, nes taip pats
        išvengs atsakomybės ir įgys saugumą.
    \end{enumerate}
  \item[Y] (tokį požiūrį turintis žmogus yra demokratas):
    \begin{enumerate}
      \item darbas toks pats natūralus kaip ir žaidimas, žmgus yra
        sutvertas darbui;
      \item žmogus gali įgyvendinti savivaldą, savirealizaciją ir
        savikontrolę, tarnaudamas tikslams, kuriems jis palankus,
        atsidavęs (atsidavimas pasireiškia, kaip paskatų rezultatas,
        susijęs su tikslų pasiekimu, žmogus gali pasinaudoti savo
        žiniomis, jis trokšta darbo);
      \item vidutinis žmogus siekia atsakomybės (jei jis atsakomybės
        vengia, tai yra kaip taisyklė susiję su nusivylimu praeityje
        bei blogu vadovavimu iš viršaus), yra apdovanotas aukštu
        fantazijos, vaizduotės lygiu ir išradingumu, kurį noriai
        panaudoja darbe.
    \end{enumerate}
\end{description}

\subsection{Klasikinis vadovavimo stilių klasifikavimas}

Pagal K. Levin.

\begin{description}
  \item[Autokratinis] – valdingas, paremtas vienvaldyste, diktatoriška
    valdžia. Tai stilius, kuriuo vadovas siekia įtvirtinti savo valdžią,
    įtaką, autoritetą. Autokratas nėra tironas, jis yra griežtas,
    valdingas, bet veikia interesų srityje. Tiesiog, jo griežtumas yra
    jo veikimo metodas. Kaip motyvavimo priemonę pripažįsta tik pinigus.
    Toks valdymas pasiteisina mažose organizacijose, kur nuolat vyksta
    darbuotojų kaita (su darbuotojais nėra laiko susidraugauti).
    Šio valdymo stiliaus trūkumas yra tai, kad nėra gerų santykių
    tarp vadovo ir pavaldinių.
  \item[Demokratinis] – kolegialus stilius, numatantis pavaldinių
    veikimo laisvę sutinkamai su jų turima kompetencija ir darbo
    pobūdžiu vadovo kontrolės ribose. Vadovas pripažįsta daugumos
    sprendimą, tačiau pasilieka teisę pats priimti ir kontroliuoti
    sprendimus. Įsiklauso į pavaldinių nuomonę, pasitaria su jais,
    bet sprendimo teisę pasilieka sau. Vadovo ir pavaldinių
    santykiai yra labai geri. Demokratas išmano, kaip galima
    motyvuoti pavaldinius, jis supranta, kad pinigų motyvas kada
    nors nebeveiks. Šio valdymo stiliaus trūkumas yra tai, kad
    sprendimo priėmimo procesas labai ilgai užtrunka.
  \item[Liberalusis] – „minkštas“ stilius, nusakantis minimalų vadovo
    dalyvavimą valdyme, suteikiant pavaldiniams veikimo ir sprendimo
    laisvę. Liberalus vadovas tikisi, kad pavaldiniai patys pasakys,
    kas jiems yra gerai ar negerai. Iš esmės, tai yra vadovavimas be
    vadovavimo. Privalumas yra tai, kad pavaldinys gali realizuoti
    save, nes turi didelę laisvę. (TODO: Išsiaiškinti: Atsakomybės
    neprisiima vadovas, ar ji nėra perduodama pavaldiniams?) Trūkumas,
    kad vadovas nesuteikia pavaldiniams atsakomybės už savo veiksmus
    suteikdamas visišką laisvę.
\end{description}

\xtable
{
  w [ 1 | 1 | 1 | 1 ]
  a [ p | p | p | p ]
  h [ | Autokratinis | Demokratinis | Liberalusis ]
  e [ Pagal vertinimo būdą
    | Vadovas save traktuoja pranašesniu už pavaldinį ir tai akivaizdžiai 
      demonstruoja.
    | Vadovas savo pavaldinius traktuoja ir elgiasi su jais, kaip su
      lygiais.
    | Vadovą domina tik profesinis pavaldinių sumanumas. ]
  e [
    | Dėl savo klaidų nesiteisina.
    | Domisi bendradarbių ir pavaldinių problemomis.
    | Beveik nesidomi darbuotojų asmeninėmis problemomis. ]
  e [
    | Nepakankamai dėmesio skiria pavaldiniams, naudoja
      įžeidžiančio elgesio  formas: ironiją, sugėdinimą,
      pažeminimą.
    | Pripažįsta savo klaidas, naudoja neužgaulias elgesio formas.
    | Kalbos maniera bejausmė, išlaikanti distanciją. ]
  e [ Pagal tikėjimo būdą
    | Mažas pasitikėjimas pavaldinių galimybėmis ir kompetencija.
    | Žino kiekvieno asmeninius sugebėjimus ir skatina juos tobulinti.
    | Nesidomi nei pavaldiniais, nei uždaviniais, kuriuos pastarieji sau
      formuluoja.
    ]
  e [
    | Į viską žvelgia pesimistiškai.
    | Būdingas optimizmas sprendžiant kiekvieną problemą.
    | Neutralus nusistatymas, nepuoselėja nei teigiamų, nei neigiamų
      vilčių. ]
  e [
    | Nepakankamas pavaldinių valios ugdymas; pasisakymai keliantys baimę.
    | Paskatinimas ir pozityvių rezultatų patvirtinimas.
    | Remiasi nuostata, kad pavaldiniai patys pasirūpins savo atlyginimu.
    ]
  e [ Pagal bendravimo būdą
    | Griežtas elgesio būdas, tikslūs nurodymai ir įsakymai bei detali
      kontrolė
    | Beveik neįsakinėja, kontroliuoja tik tai kas būtina.
    | Beveik jokio vadovavimo.
    ]
  e [
    | Daug kalba, klausia ir nurodinėja, bet mažai klauso.
    | Linkęs diskutuoti kiekvienu klausimu.
    | Džiaugiasi, kad pavaldiniai neklausinėja ir nesikreipia.
    ]
  e [
    | Palieka mažai laisvės pavaldinių aktyvumui ir asmeninei iniciatyvai.
    | Tiesiog reikalauja  iš pavaldinių aktyvumo, asmeninės iniciatyvos ir
      kūrybingumo.
    | Darbuotojų kūrybingumas ir iniciatyva beveik nėra valdomi.
    ]
}

\subsection{R. Blake ir D. Mouton „valdymo tinklelis“}

\begin{description}
  \item[abscisė (X ašis)] – gamybos poreikių įvertinimas;
  \item[ordinatė (Y ašis)] – žmogaus poreikių įvertinimas.
\end{description}

\begin{description}
  \item[Menkas valdymas (1; 1)] – minimalus rūpinimasis gamyba ir žmonėmis.
    Vadovas daro tik tiek, kad jo neišmestų iš darbo. Pavyzdžiui,
    dėstytojas, kuris nėra orientuotas į rezultatus ir santykius.
    Balansavimas ant išmetimo iš darbo ribos.
  \item[Kaimo klubo valdymas (1; 9)] – maksimalus rūpinimasis žmonėmis
    ir minimalus dėmesys skiriamas gamybiniams rodikliams. Pavyzdžiui,
    dėstytojas bendrauja su studentais, bet neverčia jų siekti
    rezultatų. Puikiai jaučiasi kompanijoj, bet nieko nedaro.
  \item[Autoritarinis valdymas (9; 1)] – prioritetas skiriamas gamybinėms
    užduotims, pilnai panaudojami valdžios įgaliojimai, grupės moralinis
    klimatas vadovui beveik nerūpi. Pavyzdžiui, dėstytojas kirvis.
    Jam visiškai nerūpi, kad studentai jo nepažįsta. Jam rūpi, kad jie
    įsisavintų medžiagą.
  \item[Vidurio kelio valdymas (5; 5)] – vadovas randa balansą tarp
    gamybinio efektyvumo ir grupės klimato. Stilius gana konservatyvus,
    orientuotas į taikų sambūvį. Vadovas stengiasi rasti kompromisą
    tarp vadovavimo ir tarpasmeninių santykių. Pamirštamas tikslas,
    jog galima tobulėti.
  \item[Komandinis valdymas (9; 9)] – vienodai didelis dėmesys skiriamas
    ir darbuotojams, ir gamybos efektyvumui. Darbuotojai dirba idėjiškai,
    daro daugiau ir geriau nei prašomi. Pavyzdžiui, dėstytojo, kuris
    sugeba „uždegti“ dėstomu dalyku, studentai išmoksta daugiau, nei
    iš jų yra reikalaujama.
\end{description}

\subsection{F. Fiedler situacinio vadovavimo teorija}

F. Fielder modelis akcentuoja situacija ir aprašo tris faktorius,
darančius įtaką vadovo elgesiui.
\begin{itemize}
  \item Vadovo ir pavaldinių tarpusavio santykiai. Turima omenyje
    pavaldinių lojalumą, pasitikėjimą savo vadovu ir vadovo
    asmenybės patrauklumą.
  \item Užduoties struktūra. Turima omenyje užduoties įprastumą,
    jos formulavimo ir struktūrizavimo aiškumą.
  \item Pareiginiai įgaliojimai. Tai teisėtos valdžios apimtis, kuri
    leidžia vadovui apdovanoti, o taip pat paramos, kurią vadovui
    teikia formali organizacija, lygis.
\end{itemize}

F. Fiedler, norėdamas nustatyti vadovo asmenines savybes, atliko
apklausą: prašė vadovus prisiminti visus žmones, su kuriais jiems
teko dirbti ir tada žmogų, su kuriuo jiems blogiausiai sekėsi
dirbti įvertinti skalėje nuo 1 iki 8. Pavyzdžiui:
\xtable
{
  w [ 1 | 1 | 1 ]
  a [ p | p | p ]
  e [ Nedraugiškas | 1 2 3 4 5 6 7 8 | Draugiškas ]
  e [ Nelinkęs bendradarbiauti | 1 2 3 4 5 6 7 8 | Linkęs bendradarbiauti ]
  e [ Priešiškas | 1 2 3 4 5 6 7 8 | Palaikantis ]
  e [ … | 1 2 3 4 5 6 7 8 | … ]
  e [ Uždaras | 1 2 3 4 5 6 7 8 | Atviras ]
}
Atsakymai būdavo susumuojami ir suvidurkinami: didelis taškų skaičius
reikšdavo, kad vadovas yra linkęs į žmogiškų ryšių palaikymą, o
žemas rodydavo polinkį į gerą užduoties atlikimą. Plačiau (ypač, kodėl
testas pasako apie testą atliekantįjį žmogų, o ne apie prasčiausią
kolegą) galima paskaityti čia:
\url{http://en.wikipedia.org/wiki/Fiedler_contingency_model#Least_preferred_co-worker_.28LPC.29}

\subsection{P. Hersey ir K. Blanchard situacinio vadovavimo modelis
(vadovavimo ciklų teorija)}

\begin{enumerate}
  \item Kadangi darbuotojai yra nauji, kurie nemoka ir nenori dirbti,
    tai jiems reikia įsakinėti, grasinti.
  \item Kai darbuotojai jau moka dirbti, bet dar nenori, tai vadovas,
    kad jie dirbtų, turi juos įtikinti, bandyti „įkurti ugnį“ komandoje.
    Kitaip tariant turi juos motyvuoti.
  \item Kai darbuotojai jau nori dirbti, tai tuomet vadovo pareiga yra tik
    palaikyti santykius, kad nedingtų motyvacija. (Pasižymi vadovo, kaip
    lyderio savybės.)
  \item Kai darbuotojai jau viską žino ir reikalauja daugiau atsakomybės,
    jiems vadovas jau gali deleguoti. Vadovo pareigos šiuo atveju yra tik
    prižiūrėti, kad darbuotojai neišeitų iš rėmų.
\end{enumerate}

\section{Kas yra lyderiavimas?}

\begin{defn}[Lyderiavimas]
  Grupės narių veiklos, reikalingos užduočiai atlikti (ar tikslui pasiekti),
  nukreipimo ir lyderio poveikio nariams procesas.
\end{defn}

\begin{defn}[Lyderis]
  Grupės narys, kuriam kiti grupės nariai pripažįsta teisę daryti
  sprendimus, susijusius su grupės veikla.
\end{defn}

Iš esmės, lyderis yra žmogus, paskui kurį visada seka bent keletas žmonių.
Lyderis siejamas su charizmatišku žmogumi, nors dabar mokslininkai
atsieja šias dvi sąvokas, nes lyderis nebūtinai turi būti charizmatiškas.

Lyderiavimas:
\begin{itemize}
  \item įtraukia kitus – darbuotojus ir pasekėjus;
    \begin{note}
      Žmonės nebūtinai mano, kad lyderis kažką daro gerai, bet jie mano,
      kad jis juos kažkur nuves.
    \end{note}
  \item reiškia nevienodą galios (jėgos) paskirstymą tarp lyderio ir
    grupės narių;
    \begin{note}
      Turi pavaldumo (FIXME: Pavaldumo ar patrauklumo?) ir gali turėti
      ekspertinę galią. Nors jis gali ir nebūti pats protingiausias,
      bet žmonės gali tikėti, kad jis toks yra.
    \end{note}
  \item gebėjimas panaudoti skirtingas galios formas, įvairiais būdais
    darant įtaką savo pasekėjų elgesiui;
    \begin{note}
      Lyderis gali turėti ir neigiamų savybių. Tarkime, kad jis palaiko
      šešėlinę ekonomiką. Tada ir jo pasekėjai palaikys ją. Todėl šiuo
      atveju atsirado sąvoka „moralus lyderiavimas“. Jis stengiasi
      savo pasekėjams įteigti, kad sektų moralės keliu.
    \end{note}
  \item susijęs su vertybėmis.
    \begin{note}
      Kokias vertybes puoselėja lyderis, tokias dažniausiai puoselėja
      ir jo pasekėjai. Moralus lyderis palieka alternatyvas savo
      pavaldiniams.
    \end{note}
\end{itemize}

Lyderiavimo funkcijos:
\begin{itemize}
  \item susijusios su užduotimi;
  \item grupės išlaikymo (socialinės).
\end{itemize}

Padalinto lyderiavimo fenomenas: formalus lyderis siekia užvesti siekti
užduoties įvykdymo, o neformalus siekia užtikrinti bendravimą.

Lyderis be su pareigom gautų galių gauna dar ir:
\begin{itemize}
  \item ekspertinę (ne visada) ir
  \item patrauklumo.
\end{itemize}

\subsection{Lyderių tipai}

Pozityvus lyderis – įdėjinis žmogus, kuris pats tiki savo idėja ir įtikina
kitus. Jis tiki, kad jo pavaldiniai yra patys geriausi.

Patrauklūs charizmatiški vadovai:
\begin{itemize}
  \item yra socialinių judėjimų priekyje;
  \item „Aš Jus išgelbėsiu“;
  \item „Aš Jūsų gyvenimui suteiksiu prasmę ir reikšmę“;
  \item būdingi bruožai:
    \begin{itemize}
      \item nepaprasta jėga ir įžvalgumas;
      \item neeilinė praktiško vadovavimo jėga;
      \item įkvėpta veikla siekiant gyvenimo tikslo;
      \item nepaprastas pasitikėjimas savimi.
    \end{itemize}
\end{itemize}

Makiavelio kūrinyje „Valdovas“ buvo aprašytas žmogus, kuris turėjo
pasekėjų, bet buvo neigiama asmenybė.

Makiaveliškas lyderis – žmogus sugebantis manipuliuoti žmonėmis.
Neigiamas lyderis. Pavyzdžiai: Rišeljė, Steve Jobs, FIXME: ar nebaigta?

Makiaveliško tipo vadovai:
\begin{itemize}
  \item kaupia ir laiko rankose valdžią;
  \item „žmonės silpni, suklaidinti, lengvatikiai ir neypatingai verti
    pasitikėjimo“;
  \item „yra daug bjaurių žmonių, būtina manipuliuoti kitais, kai reikia
    pasiekti tikslą“;
  \item būdingi bruožai:
    \begin{itemize}
      \item mažai emocionalūs tarpasmeniniuose santykiuose;
      \item abejingi moralės reikalavimams;
      \item nėra psichopatalogiški;
      \item pragmatiški ir abejingi ideologiniams įsipareigojimams.
    \end{itemize}
\end{itemize}

\subsection{Ūkinė situacija ir lyderiavimas}

Jei dabar yra, tai lyderis turi būti:
\begin{enumerate}
  \item staigaus verslo aktyvumo stadija: verslus, apsukrus, novatorius,
    kūrėjas, išsiskiriantis optimizmu, linkęs į riziką (jis turi pamatyti
    ir atskleisti organizacijos potencialą);
  \item situacijos stabilizavimo stadija: profesionalus, konformistas,
    gerą linkintis, ramus, atidus ir sistemingas darbe (jis turi 
    nusiraminti, išlaikyti idėjinę būseną);
  \item galimo smukimo stadija: atsargus, griežtas, kantrus, taupus,
    aiškiai suvokiantis prioritetus ir padėties stabilizavimo šalininkas
    (turi padėti darbuotojams stabilizuoti organizaciją, juos skatinti
    tam);
  \item pakilimo po nuosmukio stadija: avantiūrinės sandaros, su
    analitiniu protu, nebijantis sunkumų ir galįs rizikuoti
    (vėl entuziastingai kuria naujus planus, naujas strategijas).
\end{enumerate}

\section{Vadovų ir lyderių skirtumai}

\xtable
{
  w [ 1 | 1 ]
  a [ p | p ]
  h [ Vadovai (managers) | Lyderiai (leaders)]
  e [ funkcionieriai | inovatoriai ]
  e [ gina savo veiklą | tobulina savo veiklą ]
  e [ pripažįsta atsakomybę | siekia atsakomybės ]
  e [ kontroliuoja darbuotojus | pasitiki darbuotojais ]
  e [ kompetentingi | kūrybingi ]
  e [ specialistai | lankstūs ]
  e [ minimizuoja riziką | apskaičiuoja riziką ]
  e [ pripažįsta pokalbio galimybes | didina pokalbio galimybes ]
  e [ nustato realius tikslus | kelia padidintus tikslus ]
  e [ ramybė | iššūkis ]
  e [ siekia patogios darbo aplinkos | siekia jaudinančios darbo aplinkos ]
  e [ atsargiai naudoja galią | įtaigiai naudoja galią ]
  e [ deleguoja atsargiai | deleguoja entuziastingai ]
  e [ darbuotojus traktuoja, kaip samdinius
    | darbuotojus traktuoja, kaip potencialius pasekėjus ]
}

Kuo iš esmės skiriasi vadovai nuo lyderių:
\begin{itemize}
  \item vadovas yra labiau kontroliuojantis, valdantis;
  \item lyderis skatina inovatyvumą, ne tik pripažįsta, bet ir vykdo.
\end{itemize}
