\chapter{Personalo valdymas}

\begin{defn}[Personalo valdymas]
  Organizacijos darbuotojų verbavimo, mokymo ir atlyginimo už darbą
  valdymas ir administravimas.
\end{defn}

\begin{defn}[Žmonių išteklių valdymas]
  FIXME: Neturėtų būti „Žmogiškųjų išteklių valdymas“?
  
  Strateginis, nuoseklus, visapusis požiūris į organizacijos žmonių
  išteklių valdymą ir ugdymą.
\end{defn}

Personalo valdymo procesas susideda iš tokių žingsnių:
\begin{enumerate}
  \item planavimas;
  \item verbavimas;
  \item atranka;
  \item samdymas;
  \item orientavimas;
  \item vertinimas;
  \item atlyginimas.
\end{enumerate}

\section{Personalo poreikio planavimas}

\begin{defn}[Darbuotojų poreikio planavimas]
  Procesas, kuriuo numatoma, kiek ir kokios kvalifikacijos darbuotojų
  organizacija turi turėti, norėdama įgyvendinti užsibrėžtus
  tikslus.
\end{defn}

Skiriama į dvi lygiagrečias kryptis:
\begin{itemize}
  \item darbo vietų skaičiaus ir struktūros planavimas;
  \item darbuotojų planavimas.
\end{itemize}

\section{Verbavimas}

\begin{defn}[Verbavimas]
  Kandidatų į darbuotojus paieška. Šios veiklos rezultatas yra norinčiųjų
  dirbti organizacijoje kandidatų grupė, iš kurios bus atrenkamas
  samdomas darbuotojas.
\end{defn}

Verbavimas gali būti:
\begin{description}
  \item[vidinis (vidinė darbo rinka)] – kandidato paieška esamų
    organizacijos darbuotojų tarpe;
  \item[išorinis (išorinė darbo rinka)] – kandidatas ieškomas už
    organizacijos ribų. Išorinio verbavimo galimybės:
    \begin{itemize}
      \item darbo skelbimai;
      \item darbo birža;
      \item aukštųjų mokyklų studentų praktika;
      \item rekomendacijos, esamų darbuotojų giminės / draugai.
    \end{itemize}
\end{description}

\section{Atranka}

\begin{defn}[Atranka]
  Procesas, kurio metu iš turimų pretendentų į darbo vietą atrenkamas
  ir pasamdomas tinkamiausias. Atrankos procesas prasideda pretendentų
  sąrašo pateikimu priėmimui ir baigiasi darbuotojo samdos įforminimu.
\end{defn}

Kandidatų atrankos metodai:
\begin{itemize}
  \item interviu;
  \item psichometriniai testai;
  \item bandomasis darbo testas;
  \item biografinių duomenų analizė;
  \item kiti metodai.
\end{itemize}

\section{Verbavimo ir atrankos proceso stadijos}

\begin{enumerate}
  \item \emph{Verbavimas} Laisva darbo vieta. (Arba sukurta, arba
    atleistas darbuotojas.)
  \item \emph{Verbavimas} Darbų analizavimas. (Vedantis prie darbo ar
    pareigybių aprašymo sudarymo.)
  \item \emph{Verbavimas} Kandidatų į darbuotojus grupės sudarymas.
    (Svarbus tinkamo metodo parinkimas.)
  \item \emph{Atranka} Sąrašo sudarymas. (Kandidatų rūšiavimas
    palyginant prašymus su pareigybių specifikacijose nurodytais
    reikalavimais.)
  \item \emph{Atranka} Atranka. (Taikant interviu arba testus.)
  \item \emph{Atranka} Pasiūlymas ir pritarimas. (Sėkmingai
    praėjusiam atranką kandidatui siūloma darbo sutartis.)
  \item \emph{Atranka} Priėmimas.
\end{enumerate}

\section{Adaptacija}

\begin{defn}[Adaptacija]
  Abipusis prisiderinimo procesas: individo prie organizacijos ir
  atvirkščiai.
\end{defn}

Skiriami du adaptacijos lygiai:
\begin{description}
  \item[pirminis] – jaunų, neturinčių patirties darbuotojų prisitaikymo
    procesas;
  \item[antrinis] – jau turinčių darbo patirti darbuotojų prisitaikymo
    procesas.
\end{description}

Skiriami du adaptacijos tipai:
\begin{description}
  \item[techninė (arba profesinė) adaptacija] – darbuotojas supažindinamas
    su atliekamomis funkcijomis, darbo vieta ir darbo sąlygomis;
  \item[socialinė adaptacija] – darbuotojas susipažįsta ir prisitaiko
    prie naujo kolektyvo ir viršininko, pradeda suprasti ir priimti
    naujo kolektyvo vertybes, elgesio normas ir nuomones.
\end{description}

\section{Vertinimas}

\begin{defn}[Darbuotojų veiklos vertinimas]
  Formali, struktūrizuota sistema, kuri matuoja ir įvertina darbuotojų
  veiklos rezultatus bei jų elgesį, įgalindama nustatyti darbuotojo
  produktyvumo lygį.
\end{defn}

Veiklos vertinimo metodai:
\begin{itemize}
  \item rangavimas;
  \item porinis palyginimas;
  \item kritinių įvykių registravimas;
  \item su poelgiais susijusios vertinimo skalės;
  \item $360^{\circ}$ laipsnių grįžtamasis ryšys;
  \item kiti metodai.
\end{itemize}

\section{Personalo politika}

\begin{defn}[Personalo politika]
  Nustatyta ir kryptinga efektyvų žmogiškųjų išteklių valdymą užtikrinanti
  veiksmų ir sprendimų sistema.
\end{defn}

% \begin{tabularx}{\textwidth}[]{l | X | X}
%   Personalo veiklos & Atvira & Uždara \\
%   \hline
%   Personalo verbavimas
%     & iš išorinės darbo rinkos
%     & iš vidinės darbo rinkos \\
%   \hline
%   Personalo adaptacija
%     & greita, priimtinas „naujokų“ pasiūlytų naujų veiklos metodų diegimas
%     & „naujokai“ priskiriami senbuvių globai, būdingas tradicinių veiklos
%       metodų taikymas \\
%   \hline
%   Mokymas ir ugdymas
%     & dažnai vykdomi už įmonės ribų, skatina personalo novatoriškumą,
%       naujovių diegimą
%     & dažnai vykdomi įmonės viduje, formuoja vieningą nuomonę \\
%   \hline
%   Karjera
%     & vyraujant samdai iš išorės, sumažėja galimybė kilti karjeros laiptais
%     & aukštesnes pareigas dažniausiai užima senbuviai, karjera planuojama \\
%   \hline
%   Motyvacija
%     & pirmenybė teikiama išorinei (piniginei) motyvacijai
%     & pirmenybė teikiama vidinei motyvacijai (stabilumui, socialinių
%       garantijų užtikrinimui)
% \end{tabularx}

\section{Veiksniai, darantys įtaką personalo valdymui Lietuvoje}

Lietuva – pokomunistinio šalių bloko atstovė. Su tuo susiję veiksniai,
kurie personalo valdymą veikia negatyviai:
\begin{itemize}
  \item žemas pragyvenimo lygis;
  \item tarybiniais laikais susiklosčiusios tradicijos;
  \item socialinės, ekologinės ir kt. atsakomybės trūkumas.
\end{itemize}

Lietuva – nauja ES narė. Tai lemia naujas tradicijas ir tendencijas
personalo valdyme:
\begin{itemize}
  \item naujos galimybės iniciatyviems, kvalifikuotiems darbuotojams;
  \item užsienio kompanijų, atnešančių vakarietiškas tradicijas
    žmogiškųjų išteklių valdyme, padalinių steigimas Lietuvoje;
  \item darbuotojų emigracija į vakarus.
\end{itemize}
