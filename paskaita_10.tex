\chapter{Vadovavimas}

\section{Kas yra vadovavimas?}

\begin{defn}[Vadovavimas]
  Grupės ar visos organizacijos narių veiklos nukreipimas, įtakos
  darymas siekiant, kad būtų pasiekti grupės ir organizacijos tikslai.
\end{defn}

Vadovai mėgina įtikinti kitus telktis ir kartu siekti rezultatų,
aiškėjančių iš planavimo bei organizavimo etapų.

\subsection{Vadovavimo stiliai ir lyderiavimas}

\begin{defn}[Vadovavimo stilius]
  Vadovo elgesys komandos narių (pavaldinių) atžvilgiu, siekiant juos
  paveikti ir paskatinti siekti projekto (organizacijos) tikslų.
  Vadovavimo stilių formuoja vadovo vertybių sistema, komandos lūkesčiai ir
  esama situacija.
\end{defn}

Vadovavimo stilių koncepcija išplaukia iš žmogaus požiūrių į elgseną.
D. McGregor X ir Y teorija:
\begin{description}
  \item[X]:
    \begin{enumerate}
      \item žmogus iš prigimties nemėgsta darbo ir norės jo išvengti;
      \item kadangi jis nemėgsta darbo, žmogų reikia priversti,
        kontroliuoti, nukreipti, gąsdinti baudomis, kad būtų priverstas
        dirbti siekdamas organizacijos tikslų;
      \item vidutinis žmogus tikisi, kad jam vadovaus, nes taip pats
        išvengs atsakomybės ir įgys saugumą.
    \end{enumerate}
  \item[Y]:
    \begin{enumerate}
      \item darbas toks pats natūralus kaip ir žaidimas;
      \item žmogus gali įgyvendinti savivaldą ir savikontrolę,
        tarnaudamas tikslams, kuriems jis palankus, atsidavęs
        (atsidavimas pasireiškia, kaip paskatų rezultatas, susijęs
        su tikslų pasiekimu);
      \item vidutinis žmogus siekia atsakomybės (jei jis atsakomybės
        vengia, tai yra kaip taisyklė susiję su nusivylimu praeityje
        bei blogu vadovavimu iš viršaus), yra apdovanotas aukštu
        fantazijos, vaizduotės lygiu ir išradingumu, kurį noriai
        panaudoja darbe.
    \end{enumerate}
\end{description}

\subsection{Klasikinis vadovavimo stilių klasifikavimas}

Pagal K. Levin.

\begin{description}
  \item[Autokratinis] – valdingas, paremtas vienvaldyste, diktatoriška
    valdžia. Tai stilius, kuriuo vadovas siekia įtvirtinti savo valdžią,
    įtaką, autoritetą.
  \item[Demokratinis] – kolegialus stilius, numatantis pavaldinių
    veikimo laisvę sutinkamai su jų turima kompetencija ir darbo
    pobūdžiu vadovo kontrolės ribose. Vadovas pripažįsta daugumos
    sprendimą, tačiau pasilieka teisę pats priimti ir kontroliuoti
    sprendimus.
  \item[Liberalusis] – „minkštas“ stilius, nusakantis minimalų vadovo
    dalyvavimą valdyme, suteikiant pavaldiniams veikimo ir sprendimo
    laisvę.
\end{description}

\begin{tabularx}{\textwidth}[]{l | X | X | X}
  & Autokratinis & Demokratinis & Liberalusis \\
  \hline
  \multirow{3}{0.1\textwidth}{Pagal vertinimo būdą}
  & Vadovas save traktuoja pranašesniu už pavaldinį ir tai akivaizdžiai 
    demonstruoja.
  & Vadovas savo pavaldinius traktuoja ir elgiasi su jais, kaip su
    lygiais.
  & Vadovą domina tik profesinis pavaldinių sumanumas. \\
  & Dėl savo klaidų nesiteisina.
  & Domisi bendradarbių ir pavaldinių problemomis.
  & Beveik nesidomi darbuotojų asmeninėmis problemomis. \\
  & Nepakankamai dėmesio skiria pavaldiniams, naudoja įžeidžiančio elgesio 
    formas: ironiją, sugėdinimą, pažeminimą.
  & Pripažįsta savo klaidas, naudoja neužgaulias elgesio formas.
  & Kalbos maniera bejausmė, išlaikanti distanciją. \\
  \hline
  \multirow{3}{0.1\textwidth}{Pagal tikėjimo būdą}
  & Mažas pasitikėjimas pavaldinių galimybėmis ir kompetencija.
  & Žino kiekvieno asmeninius sugebėjimus ir skatina juos tobulinti.
  & Nesidomi nei pavaldiniais, nei uždaviniais, kuriuos pastarieji sau
    formuluoja. \\
  & Į viską žvelgia pesimistiškai.
  & Būdingas optimizmas sprendžiant kiekvieną problemą.
  & Neutralus nusistatymas, nepuoselėja nei teigiamų, nei neigiamų
    vilčių. \\
  & Nepakankamas pavaldinių valios ugdymas; pasisakymai keliantys baimę.
  & Paskatinimas ir pozityvių rezultatų patvirtinimas.
  & Remiasi nuostata, kad pavaldiniai patys pasirūpins savo atlyginimu. \\
  \hline
  \multirow{3}{0.1\textwidth}{Pagal bendravimo būdą}
  & Griežtas elgesio būdas, tikslūs nurodymai ir įsakymai bei detali
    kontrolė
  & Beveik neįsakinėja, kontroliuoja tik tai kas būtina.
  & Beveik jokio vadovavimo. \\
  & Daug kalba, klausia ir nurodinėja, bet mažai klauso.
  & Linkęs diskutuoti kiekvienu klausimu.
  & Džiaugiasi, kad pavaldiniai neklausinėja ir nesikreipia. \\
  & Palieka mažai laisvės pavaldinių aktyvumui ir asmeninei iniciatyvai.
  & Tiesiog reikalauja  iš pavaldinių aktyvumo, asmeninės iniciatyvos ir
    kūrybingumo.
  & Darbuotojų kūrybingumas ir iniciatyva beveik nėra valdomi. \\
\end{tabularx}

\subsection{R. Blake ir D. Mouton „valdymo tinklelis“}

\begin{description}
  \item[abscisė (X ašis)] – gamybos poreikių įvertinimas;
  \item[ordinatė (Y ašis)] – žmogaus poreikių įvertinimas.
\end{description}

\begin{description}
  \item[Menkas valdymas (1; 1)] – minimalus rūpinimasis gamyba ir žmonėmis.
  \item[Kaimo klubo valdymas (1; 9)] – maksimalus rūpinimasis žmonėmis
    ir minimalus dėmesys skiriamas gamybiniams rodikliams.
  \item[Autoritarinis valdymas (9; 1)] – prioritetas skiriamas gamybinėms
    užduotims, pilnai panaudojami valdžios įgaliojimai, grupės moralinis
    klimatas vadovui beveik nerūpi.
  \item[Vidurio kelio valdymas (5; 5)] – vadovas randa balansą tarp
    gamybinio efektyvumo ir grupės klimato. Stilius gana konservatyvus,
    orientuotas į taikų sambūvį.
  \item[Komandinis valdymas (9; 9)] – vienodai didelis dėmesys skiriamas
    ir darbuotojams, ir gamybos efektyvumui.
\end{description}

\subsection{F. Fiedler situacinio vadovavimo teorija}

Situaciniai veiksniai, darantys įtaką vadovavimo stilių efektyvumui:
\begin{itemize}
  \item vadovo ir pavaldinių tarpusavio santykiai;
  \item užduoties struktūra, jos aiškumas ir įprastumas;
  \item vadovo įgaliojimai (jo turimos galimybės kontroliuoti pavaldinių
    veiksmus, skatinti jų aktyvumą).
\end{itemize}

\subsection{P. Hersey ir K. Blanchard situacinio vadovavimo modelis
(vadovavimo ciklų teorija)}

TODO

\section{Kas yra lyderiavimas?}

\begin{defn}[Lyderiavimas]
  Grupės narių veiklos, reikalingos užduočiai atlikti (ar tikslui pasiekti),
  nukreipimo ir lyderio poveikio nariams procesas.
\end{defn}

\begin{defn}[Lyderis]
  Grupės narys, kuriam kiti grupės nariai pripažįsta teisę daryti
  sprendimus, susijusius su grupės veikla.
\end{defn}

Iš esmės, lyderis yra žmogus, paskui kurį visada seka bent keletas žmonių.

Lyderiavimas:
\begin{itemize}
  \item įtraukia kitus – darbuotojus ir pasekėjus;
  \item reiškia nevienodą galios (jėgos) paskirstymą tarp lyderio ir
    grupės narių;
  \item gebėjimas panaudoti skirtingas galios formas, įvairiais būdais
    darant įtaką savo pasekėjų elgesiui;
  \item susijęs su vertybėmis.
\end{itemize}

Lyderiavimo funkcijos:
\begin{itemize}
  \item susijusios su užduotimi;
  \item grupės išlaikymo (socialinės).
\end{itemize}

Padalinto lyderiavimo fenomenas: formalus lyderis siekia užvesti siekti
užduoties įvykdymo, o neformalus siekia užtikrinti bendravimą.

Lyderis be su pareigom gautų galių gauna dar ir:
\begin{itemize}
  \item ekspertinę (ne visada) ir
  \item patrauklumo.
\end{itemize}

\subsection{Lyderių tipai}

Pozityvus lyderis – įdėjinis žmogus, kuris pats tiki savo idėja ir įtikina
kitus. Jis tiki, kad jo pavaldiniai yra patys geriausi.

Patrauklūs charizmatiški vadovai:
\begin{itemize}
  \item yra socialinių judėjimų priekyje;
  \item „Aš Jus išgelbėsiu“;
  \item „Aš Jūsų gyvenimui suteiksiu prasmę ir reikšmę“;
  \item būdingi bruožai:
    \begin{itemize}
      \item nepaprasta jėga ir įžvalgumas;
      \item neeilinė praktiško vadovavimo jėga;
      \item įkvėpta veikla siekiant gyvenimo tikslo;
      \item nepaprastas pasitikėjimas savimi.
    \end{itemize}
\end{itemize}

Makiavelio kūrinyje „Valdovas“ buvo aprašytas žmogus, kuris turėjo
pasekėjų, bet buvo neigiama asmenybė.

Makiaveliškas lyderis – žmogus sugebantis manipuliuoti žmonėmis.
Neigiamas lyderis. Pavyzdžiai: Rišeljė, Steve Jobs, FIXME: ar nebaigta?

Makiaveliško tipo vadovai:
\begin{itemize}
  \item kaupia ir laiko rankose valdžią;
  \item „žmonės silpni, suklaidinti, lengvatikiai ir neypatingai verti
    pasitikėjimo“;
  \item „yra daug bjaurių žmonių, būtina manipuliuoti kitais, kai reikia
    pasiekti tikslą“;
  \item būdingi bruožai:
    \begin{itemize}
      \item mažai emocionalūs tarpasmeniniuose santykiuose;
      \item abejingi moralės reikalavimams;
      \item nėra psichopatalogiški;
      \item pragmatiški ir abejingi ideologiniams įsipareigojimams.
    \end{itemize}
\end{itemize}

\begin{defn}[Moralus lyderiavimas]
  Vadovas-lyderis, kuris yra moraliai atsakingas. Moralus lyderis
  palieka alternatyvas savo pavaldiniams.
  \begin{exmp}
    Pavyzdžiui, „Body-Shop“ įkūrėja skatina užsiimti socialine atsakomybe.
  \end{exmp}
\end{defn}

\subsection{Ūkinė situacija ir lyderiavimas}

Jei dabar yra, tai lyderis turi būti:
\begin{enumerate}
  \item staigaus verslo aktyvumo stadija: verslus, apsukrus, novatorius,
    kūrėjas, išsiskiriantis optimizmu, linkęs į riziką;
  \item situacijos stabilizavimo stadija: profesionalus, konformistas,
    gerą linkintis, ramus, atidus ir sistemingas darbe;
  \item galimo smukimo stadija: atsargus, griežtas, kantrus, taupus,
    aiškiai suvokiantis prioritetus ir padėties stabilizavimo šalininkas;
  \item pakilimo po nuosmukio stadija: avantiūrinės sandaros, su
    analitiniu protu, nebijantis sunkumų ir galįs rizikuoti.
\end{enumerate}

\section{Vadovų ir lyderių skirtumai}

\begin{tabularx}{\textwidth}[]{X | X}
  Vadovai (managers) & Lyderiai (leaders) \\
  \hline
  funkcionieriai & inovatoriai \\
  gina savo veiklą & tobulina savo veiklą \\
  pripažįsta atsakomybę & siekia atsakomybės \\
  kontroliuoja darbuotojus & pasitiki darbuotojais \\
  kompetentingi & kūrybingi \\
  specialistai & lankstūs \\
  minimizuoja riziką & apskaičiuoja riziką \\
  pripažįsta pokalbio galimybes & didina pokalbio galimybes \\
  nustato realius tikslus & kelia padidintus tikslus \\
  ramybė & iššūkis \\
  siekia patogios darbo aplinkos & siekia jaudinančios darbo aplinkos \\
  atsargiai naudoja galią & įtaigiai naudoja galią \\
  deleguoja atsargiai & deleguoja entuziastingai \\
  darbuotojus traktuoja, kaip samdinius & darbuotojus traktuoja, kaip
  potencialius pasekėjus
\end{tabularx}


Kuo iš esmės skiriasi vadovai nuo lyderių:
\begin{itemize}
  \item labiau kontroliuojantis, valdantis;
  \item skatina inovativumą, ne tik pripažįsta, bet ir vykdo.
\end{itemize}
