\chapter{Motyvacija}

\section{Kas yra motyvacija?}

\begin{defn}[Motyvacija]
  \begin{itemize}
    \item Poreikis ar troškimas, teikiantis elgesiui energijos ir
      nukreipiantis į tikslą.
    \item Tai, kas sužadina, nukreipia ir palaiko dėmesį.
    \item Psichologinė savybė, lemianti asmens įsipareigojimo laipsnį.
  \end{itemize}
\end{defn}

\begin{defn}[Motyvacija]
  Visuma faktorių, kurie sukelia, išlaiko ir valdo elgesį, kuris leidžia
  pasiekti tam tikrą tikslą.
\end{defn}

\begin{defn}[Motyvavimas]
  Valdymo proceso dalis, reiškianti poveikio darbuotojų elgsenai darymą,
  siekiant pasiekti organizacijos tikslus.
\end{defn}

\begin{defn}[Motyvavimas]
  Poveikio žmogaus motyvacijai darymas.
\end{defn}

\begin{defn}[Motyvai]
  Tikslai, kurie mus skatina kažkaip elgtis.
\end{defn}

Valdantysis personalas turi sugebėti motyvuoti darbuotojus:
\begin{itemize}
  \item pasilikti organizacijoje;
  \item patikimai atlikti pavestas užduotis (jei nenori daryti užduoties,
    tai liepiant jam atlikti tą užduotį, jis ją padarys blogiau);
  \item savanoriškai užsiimti kokia nors kūrybiška novatoriška veikla.
\end{itemize}

\section{Ankstyvieji motyvacijos požiūriai}

\begin{description}
  \item[Tradicinis modelis] – bizūno ir meduolio koncepcija (jei elgiasi
    blogai – reikia barti, o jei gerai – duoti meduolį). Paminėtinas
    Fredas Teiloras \en{Fred Manville Taylor}, kuris suprato uždarbio
    ties bado riba absurdiškumą, sukurdamas „pakankamo dienos
    išdirbio“ sąvoką.
  \item[Žmonių santykių modelis] – vadovai gali skatinti savo darbuotojus
    pažindami jų socialinius poreikius ir suteikdami jiems galimybę
    jaustis svarbiais ir naudingais. TODO:(Bi~ teorija? Žmogų labai skatina
    jo socialinių poreikių tenkinimas.)
  \item[Žmonių išteklių modelis] – teorija X ir Y. (X teorijos atveju
    prievartiniais metodais, nuobaudomis, o Y atveju – duoti sąlygas,
    kad jis galėtų realizuotis. Remiantis Y teorija atsirado darbo
    praturtinimas: kuo TODO: )
\end{description}

\section{Šiuolaikinės motyvacijos teorijos}

\begin{description}
  \item[Turinio teorijos] – pagrįstos poreikių, kuriuos žmonės siekia 
    patenkinti darbe, aprašymu. (Kiekvienas žmogus turi kokių nors
    poreikių ir tenkindami ar netenkindami jų, mes žmogų kur nors
    nukreipiame.)
  \item[Proceso teorijos] – aiškina „kaip“ ir „kodėl“ žmonės pasirenka
    tam tikrą elgesio būdą tam, kad patenkintų savo asmeninius tikslus.
  \item[Pastiprinimo teorija] – remiasi aiškinimu, kaip atliktų
    veiksmų rezultatai daro įtaką būsimiems žmogaus veiksmams. (TODO:
    Akcentuoja ne žmogaus elgesį, …?)
\end{description}

\section{A. Maslow poreikių hierarchija}

\label{sec:maslow}

\begin{enumerate}
  \item Fiziologiniai poreikiai – būtini norint išgyventi. Tai poreikiai
    pavalgyti, gerti, turėti pastogę.
  \item Saugumo poreikiai – apsaugančios nuo fizinių ir psichologinių
    sukrėtimų aplinkos poreikis bei fiziologinių poreikių nuolatinio
    patenkinimo užtikrinimas. Kitaip tariant – užtikrinimas to, kad
    būsime apsaugoti nuo egzistencinių dalykų trūkumo.
  \item Socialiniai poreikiai – poreikis būti priimtam sau lygių,
    priklausyti kokiai nors socialiniai grupei, meilės, prisirišimo
    poreikis. Šitas poreikis sustiprėja dar ir dėl to, kad pavieniui
    yra sunkiau užtikrinti saugumo ir fiziologinius poreikius.
  \item Pagarbos poreikis – savigarbos, asmeninės sėkmės, kompetencijos,
    aplinkinių žmonių pagarbos, pripažinimo poreikis. Žmogus jaučiasi
    esąs pilnavertė asmenybė. Jis nenori būti atstumtas, nori būti
    pripažintas.
  \item Saviraiškos poreikis – poreikis realizuoti savo potencialias
    galimybes, augant kaip žmogui, asmenybei. Gavus pripažinimą, norisi
    įrodyti, kad tai gauta ne veltui. Žmogui norisi pasirodyti, koks
    jis yra nuostabus, koks jis yra ypatingas.
\end{enumerate}

Pirmos trys pakopos: žemiausio lygio poreikiai, pagrindiniai.

Likusios dvi: aukštesnieji, asmeniniai poreikiai.

Pagrindinė piramidės savybė: žmogui, neturinčiam žemesnio lygio, aukštesnis
nerūpi.

\section{D. Mc Clelland poreikių teorija}

Pabrėžiami trys žmogaus poreikiai, kurių stiprumas (vienas iš jų dominuoja)
lemia žmogaus elgesį:
\begin{description}
  \item[laimėjimų] – pasitenkinimas procesu, kuriančiu sėkmingą darbo
    pabaigą (būdinga sportininkams);
  \item[valdžios] – turėti įtakos kitiems žmonėms, juos veikti ir
    taip dominuoti;
  \item[bendrumo ir priklausymo] – siekiamybė turėti draugų ir tapti
    grupės nariu.
\end{description}

\begin{exmp}
  Jei žmogus turi laimėjimo poreikį, tai jis nori kažką išmokti ir už
  tai gauti diplomą, o neturintis laimėjimo poreikio nemato problemų
  nusipirkti diplomą.
\end{exmp}

\begin{exmp}
  Tarkime, studentų grupei reikia padaryti pristatymą. Laimėjimo poreikį
  turintis žmogus pasirinks moksliukus, o tas, kuris turi bendrumo
  poreikį, pasirinks draugus.
\end{exmp}

\section{F. Herzberg dviejų veiksnių teorija}

Alternatyvūs pavadinimai: higienos-motyvatorių teorija arba
demotyvatorių-motyvatorių teorija.

Higienos veiksnių neįgyvendinimas – reiškia, kad žmogus bus demotyvuotas, 
bet jų įgyvendinimas neiššauks pasitenkinimo. Pavyzdžiui, jei žmogus
negaus atlyginimo, tai jis bus nepatenkintas darbu. Bet, jei atlyginimą
jis gaus, tai dar nereiškia, kad darbu jis bus patenkintas.

Veiksniai-motyvatoriai – reiškia, kad jų patenkinimas iššauks pasitenkinimą,
o nepatenkinimas neiššauks nepasitenkinimo.

Higienos veiksniai:
\begin{itemize}
  \item atlyginimas;
  \item darbo sąlygos;
  \item įmonės politika;
  \item statusas;
  \item priežiūra ir autonomija;
  \item santykiai su kolegomis;
  \item asmeninis gyvenimas.
\end{itemize}

Veiksniai-motyvatoriai:
\begin{itemize}
  \item pasiekimo galimybė;
  \item pripažinimo galimybė;
  \item pranašumo galimybė;
  \item darbo esmė;
  \item tobulėjimo galimybė;
  \item patikėta atsakomybė už atliekamą darbą.
\end{itemize}

\section{Lūkesčių teorija (Lawler, Vroom ir Porter)}

Lūkesčių teorija teigia, kad žmogaus elgesio pasirinkimas priklauso nuo
lūkesčio, kad šį veiksmą lydės konkretus rezultatas, už kurį jis
gaus patrauklų atlygį.

Lūkesčių teorija nagrinėja tris veiksnius:
\begin{description}
  \item[valentingumas (atlygio patrauklumas)] – darbo rezultato ir
    atlygio už jį vertingumas individui;
    \begin{note}
      Žmogus galvoja, ar vertingas jam tas atlygis, kurį jis gaus. Ar
      verta dėl jo stengtis ar nesistengti. Aukštas valentingumas,
      jei atlygis yra svarbus, o jei nesvarbus – žemas.
    \end{note}
  \item[instrumentalumas] – tikėjimas, kad pasiekus atitinkamą rezultatą,
    žmogus gaus laukiamą atlygį;
  \item[lūkesčiai] – išreiškiami stipria viltimi, kad į darbą įdėtos
    pastangos leis jį sėkmingai atlikti.
\end{description}

Galima suformuluoti trimis klausimais:
\begin{itemize}
  \item jei aš padarysiu šį darbą, koks bus rezultatas?
  \item ar man verta dėl jo stengtis?
  \item kokios mano galimybės pasiekti man vertingą rezultatą?
\end{itemize}

Mes pasirenkame, kokias pastangas norime dėti ir kokį veiklos rezultatą
gausime.

\begin{verbatim}
Pastangos → Veiklos rezultatas → Atlygis
\end{verbatim}

Svarbiausia: ryšiai tarp blokų. Instrumentalumas: strėlytė tarp veiklos
rezultato ir atlygio. Lūkesčiai: pastangos → veiklos rezultatas.

\section{Adamso teisingumo teorija}

Pagal teisingumo teoriją, žmonės yra linkę reguliuoti savo elgesį
priklausomai nuo suvokimo, ar teisinga yra pusiausvyra tarp jų indėlio
ir atlygio už atliktą užduotį, lyginant su kitų žmonių indėliu ir
atlygiu už atliktas panašias užduotis.

\begin{description}
  \item[Individualus teisingumas] – indėlis ir atlygis už jį žmogaus
    yra suvokiami, kaip lygiaverčiai. Neteisingumą žmogus jaučia,
    kai atlygis nekompensuoja įdėtų pastangų.
    \begin{exmp}
      Studentas parašė kontrolinį. Jei jis mano, kad parašė gerai ir
      taip pat gavo gerą pažymį, tai jis jaučiasi patenkintas. Jei
      mano, kad parašė gerai, o gavo blogą pažymį, tai studentas tuomet
      jaučia neteisybę. Statistiškai nustatyta, kad dauguma jaučia
      neteisybę ir tuomet, kai mano, jog jų darbas vertas mažiau, nei
      jie gavo.
    \end{exmp}
  \item[Vidinis teisingumas] – individo indėlio ir atlygio santykis yra
    lygus tokį pat darbą atliekančiųjų darbuotojų indėlio ir atlygio
    santykiui.
    \begin{exmp}
      Jei vienas studentas mažai mokėsi ir gavo blogą
      pažymį ir kitas studentas mažai mokėsi ir gavo blogą pažymį, tai
      tuomet jie jaučiasi gerai, nes viskas teisinga.
    \end{exmp}
  \item[Išorinis teisingumas] – individo indėlio ir atlygio santykis
    lygus kitų visuomenės narių (kitų profesijų atstovų) indėlio
    ir atlygio santykiui. Šiuo atveju yra vertinama platesniu mastu
    – žmogus lygina save su kitų organizacijų darbuotojais, netgi
    atliekančiais visiškai kitokią veiklą.
    \begin{exmp}
      Paprastas darbuotojas, dirbantis 8 valandas per dieną, norėtų, kad
      ir Seimo nariai dirbtų po 8 valandas per dieną ir gautų tą patį
      paprasto darbuotojo atlyginimą.
    \end{exmp}
\end{description}

Teorija koncentruojasi į kažkokį vieną konkretų žmogų.

Žmogus jaučiasi patenkintas tada, kai įdėtas atlygis atitinka gautą
rezultatą.

\section{B. Skinner pastiprinimo teorija}

Pastiprinimo teorijos esmė – elgesys priklauso nuo jau buvusio elgesio
rezultatų.

\begin{exmp}
  Žmogus, padaręs kažką gerai ir už tai gavęs gerą atlygį, kitą kartą
  vėl stengsis tai pakartoti, o gavęs blogą atlygį, stengsis daugiau
  to nedaryti.
\end{exmp}

Pastiprinimo teorija nagrinėja, kaip galima modifikuoti mūsų elgesį
priklausomai nuo mūsų norų ir mūsų tikslų. Ši teorija nagrinėja
buvusio elgesio rezultatą naujam elgesiui.

Elgesys, už kurį žmogus apdovanojamas, greičiausiai pasikartos, o
elgesys, už kurį yra baudžiama, greičiausiai ateityje nepasikartos,
o bus modifikuotas.

Pastiprinimo teorija leido vadovams įvaldyti naujas žmonių elgesio
keitimo galimybes. Šios teorijos pagrindu buvo pasiūlyti elgesio
modifikavimo metodai, pagrįsti nuostata, kad vadovai, norintys
pakeisti savo darbuotojų elgesį, visų pirma turi keisti to elgesio
padarinius.

Pastiprinimo tipai:
\begin{description}
  \item[teigiamas pastiprinimas] – apdovanojimas, skatinantis norimo
    elgesio pasikartojimą (viskas, kas skatina darbuotoją pakartoti norimą
    elgesį, gali būti klasifikuojamas kaip teigiamas pastiprinimas);
    \begin{exmp}
      Jei darbuotojas vietoj 20 detalių padaręs 40 gavo didesnį atlygį,
      tai jis kitą dieną vėl bandys padaryti 40, kad vėl gautų didesnį
      atlygį.
    \end{exmp}
  \item[vengimas] – norimo elgesio palaikymas tam, kad būtų išvengta
    jau žinomų nemalonių pasekmių (bandymas išlaikyti elgesį viename
    lygyje, kad darbuotojas nebandytų kažko keisti ir nepridarytų žalos);
  \item[baudimas] – negatyvių pasekmių pritaikymas, kai atsitinka
    nepageidaujamas elgesys (žmogus baudžiamas tol kol, nesuvokia, kad
    elgiasi blogai);
  \item[gesinimas] – pastiprinimo nebuvimas po nepageidaujamo elgesio.
    Idėja yra ta, kad netinkamas elgesys yra paprasčiausiai ignoruojamas.
    Pagal teoriją, žmogus ir be baudimo po kelių netinkamo elgesio kartų
    susipras, kad daro kažką negerai.
\end{description}
