\chapter{Vadybos apibrėžimas}

\begin{defn}[Vadyba]
  Mokymas apie organizacijų veiklos tvarkymo dėsningumus, metodus,
  principus ir sistemas.
\end{defn}

\begin{defn}[Organizacija]
  Tam tikrai veiklai suburtas sistemiškas kolektyvas, turintis turto ir
  tvarkytojų.
\end{defn}

Pagrindinės vadybos funkcijos:
\begin{itemize}
  \item planavimas;
  \item organizavimas;
  \item motyvavimas (lyderiavimas);
  \item kontroliavimas.
\end{itemize}

Valdymo personalas siekia:
\begin{itemize}
  \item žemų resursų sąnaudų (aukštas efektyvumas);
  \item aukšto organizacijos tikslo pasiekimo laipsnio (aukštas
  rezultatyvumas).
\end{itemize}

Valdymo lygiai:
\begin{description}
  \item[Aukščiausio lygio vadovai] – atsakingi už visos organizacijos
    valdymą. Jie nustato tikslus, vysto strategiją ir kuria organizacijos
    veiklos politiką. Pavyzdžiai: prezidentas, rektorius, strategas,
    vadovas.
  \item[Vidurinio lygio vadovai] – atsakanti už strategijos ir politikos
    įgyvendinimą, derina aukščiausios vadovybės reikalavimus su savo
    pavaldinių sugebėjimais. Pavyzdžiai: direktorius, valdytojas,
    vedėjas.
  \item[Žemiausio lygio vadovai] – atsakingi tik už vykdytojų, tai yra
    eilinių darbuotojų darbą: įvertina ir koordinuoja jų veiklą.
    Pavyzdžiai: brigadininkas, koordinatorius.
\end{description}

Gali būti vadovai-lyderiai ir vadovai-administratoriai. Jų santykį
gerai nusako angliška frazė: „Leadership is doing the right things.
Management is doing things right“.

\section{Valdymo turinys}

Valdymo turinį nusako:
\begin{itemize}
  \item funkcijos;
  \item vaidmenys;
  \item įgūdžiai.
\end{itemize}

\subsection{Pagrindinės veiklos funkcijos}

\begin{itemize}
  \item Marketingas.
  \item Finansai.
  \item Gamyba.
  \item Personalo valdymas.
  \item Kokybės valdymas.
  \item Techninių tarnybų veikla.
  \item Kita.
\end{itemize}

\subsection{Vadovo vaidmenys}

Pagal H. Mitzberg.

\begin{itemize}
  \item Formalus autoritetas ir statusas.
  \item Tarpasmeniniai vaidmenys:
    \begin{enumerate}
      \item boso – simbolio, nominalaus vadovo;
      \item sąveikos, ryšių;
      \item lyderio.
    \end{enumerate}
  \item Informaciniai vaidmenys:
    \begin{enumerate}
      \item patarėjo, auklėtojo;
      \item platintojo;
      \item atstovo, kalbėtojo;
    \end{enumerate}
  \item Sprendimų priėmimo vaidmenys:
    \begin{enumerate}
      \item verslininko;
      \item tvarkos įvedimo;
      \item išteklių paskirstymo;
      \item derybininko, tarpininko.
    \end{enumerate}
\end{itemize}

\subsection{Valdymo įgūdžiai}

\begin{description}
  \item[Techniniai įgūdžiai] – sugebėjimas panaudoti žinias, procedūras
    ir metodus savo specializacijos srityje.
  \item[Tarpasmeniniai įgūdžiai] – sugebėjimas bendrauti, suprasti
    ir motyvuoti kitus.
  \item[Koncepciniai įgūdžiai] – sugebėjimas koordinuoti ir integruoti
    visus organizacijos interesus ir veiklą, suvokti organizaciją
    kaip visumą.
\end{description}

\chapter{Vadybos teorijų raida}

\section{Vadybos mokslo užuomazgos}

\begin{description}
  \item[Pirmykštė visuomenė:] kolektyvinė veikla, socialinių junginių
    susidarymas, administracinių teritorinių dalinių formavimas.
  \item[Šumerų civilizacija:] Chamurapio įstatymai ir lyderiavimo
    pavyzdys.
  \item[Senovės Romos ir Graikijos karo vadai:] tvirta hierarchinė
    struktūra.
  \item[Romos imperatorius Diokletianas:] pirmas bandymas taikyti
    valdžios decentralizaciją.
\end{description}

\section{Vadybos teorijų raida}

\begin{description}
  \item[Valdymo mokyklų požiūris:] apima mokslinio valdymo mokyklą, 
    klasikinę administracinę mokyklą, žmogiškųjų santykių mokyklą,
    vadybos mokslo arba kiekybinių metodų mokyklą.
  \item[Procesinis požiūris:] vadyba nagrinėjama kaip nenutrūkstama
    tarpusavyje susijusių vadybos funkcijų (planavimo, organizavimo,
    motyvavimo ir kontroliavimo) visuma.
  \item[Sisteminis požiūris:] sistema – tai visuma, sudaryta iš
    sąveikaujančių dalių, kurių kiekviena daro įtaką bendroms
    charakteristikoms.
  \item[Situacinis (tikimybinis, atsitiktinis) požiūris:] teigia,
    kad įvairių metodų tinkamumą apsprendžia aplinkybės. Svarbiausias
    šio požiūrio elementas – situacija, konkrečios aplinkybės,
    veikiančios organizaciją.
\end{description}

\subsection{Mokslinio valdymo mokykla (1885-1920)}

\begin{description}
  \item[Frederik Taylor] – rėmėsi 4 mokslinio valdymo principais:
    \begin{itemize}
      \item tikro vadybos mokslo sukūrimas;
      \item darbininkų parinkimas moksliniais pagrindais;
      \item glaudūs ir draugiški ryšiai tarp administracijos ir
        darbininkų.
    \end{itemize}
    Jis bandė nustatyti tiesioginį ryšį tarp darbo užmokesčio ir išdirbio
    normų.
  \item[Frank ir Lilian Gilbreth] – sukūrė ciklografinį judesių tyrimo
    metodą. Siekė parodyti darbininkams, kaip padidinti darbo našumą
    ne dirbant greičiau, o geriau. Kaip pakeisti naudojamus metodus
    bei paaiškinti, kodėl tai būtina padaryti.
  \item[H. L. Gantt] – parengė darbų planavimo metodus, kurių pagrindas
    yra grafinis laiko paskirstymo vaizdavimas (Gantt diagramos).
\end{description}

\subsection{Klasikinės organizacijos (administravimo) teorijos mokykla
(1920-1950)}

\begin{description}
  \item[Henry Fayol] – į organizaciją žvelgė, kaip į visumą, bandė 
    suformuoti universalų požiūrį į organizacijų administravimą,
    suformulavo penkias administravimo funkcijas. Iškėlė 14 administravimo
    principų:
    \begin{enumerate}
      \item darbo pasidalijimas;
      \item valdžia – atsakomybė;
      \item drausmė;
      \item nurodymų vienybė;
      \item vadovavimo vienybė;
      \item asmeninių interesų pajungimas bendriems;
      \item personalo apdovanojimas;
      \item centralizacija;
      \item hierarchija;
      \item tvarka;
      \item teisingumas;
      \item personalo pastovumas;
      \item iniciatyva;
      \item personalo vienybė.
    \end{enumerate}
  \item[Max Webber] – biurokratinio valdymo teorija. Visa veikla turi
    būti reglamentuota ir bet koks nukrypimas yra žala organizacijai.
    Iškėlė valdymo principus:
    \begin{itemize}
      \item \strong{hierarchija} su aiškiai apibrėžtais valdymo lygiais;
      \item \strong{beasmenis} pareigų atlikimas;
      \item visi priimti sprendimai turi būti griežtai \strong{fiksuojami};
      \item valdžios padalinimas tarp \strong{pareigybių}, o ne asmenų.
    \end{itemize}
\end{description}

\subsection{Bihevioristinė (elgesio ir žmogiškųjų santykių) mokykla
(1930-1950 ir dabar)}

\begin{itemize}
  \item Pagrindinė idėja: organizacija yra žmonės.
  \item Išskiria socialinių veiksnių įtaką gamybos valdymui ir suranda
    naujų būdų darbo našumui didinti:
    \begin{itemize}
      \item grupiniai sprendimai;
      \item darbo humanizavimas;
      \item žmonių elgesio psichologiniai motyvai gamybos procese;
      \item grupinės normos;
      \item neformalios organizacijos ir neformalūs lyderiai.
    \end{itemize}
  \item \strong{F. Mayo - Hawthorne} eksperimentai: suvokimas, kad
    darbuotojais yra rūpinamasi, skatina stipriau nei fizinis darbo
    sąlygų pakeitimas.

    \begin{note}
      Buvo atliktas toks eksperimentas. Organizacija turėjo dvi vienodas
      gamyklas. Vienos darbuotojams pasakė, kad jie labai rūpi valdžiai
      ir kad ji jiems todėl įrengs patį moderniausią apšvietimą, bet
      realiai nieko nepakeitė. Kitos darbuotojams nieko nepasakė ir
      įrengė tą apšvietimą. Rezultatas: tie, kuriems pasakė, dirbo
      žymiai geriau, nei tie, kuriems pakeitė.
    \end{note}
\end{itemize}

\subsection{Kiekybinių metodų (vadybos mokslo) mokykla (1950-dabar)}

\begin{itemize}
  \item Visus valdymo procesus bandoma metematizuoti ir formalizuoti,
    šiam tikslui plačiai panaudojamas logikos aparatas, žaidimų
    ir eilių teorijos, statistikos ir prognozavimo metodai.
  \item Matematikos modelių, skaičiavimo technikos bei informacinių
    valdymo sistemų taikymas valdymo situacijoms, problemoms spręsti
    ir vykdymui kontroliuoti.
\end{itemize}

\subsection{Sisteminis požiūris (nuo 1970 vidurio)}

\begin{itemize}
  \item Organizacija yra suvokiama kaip sistema, kurios vieni komponentai
    daro poveikį kitiems.
  \item Organizacija sudaryta iš posistemių (padalinių, veiklų), be to,
    pati yra didesnės sistemos (rinkos, verslo šakos) dalis.
  \item Išryškinta sinergijos efekto svarba – tarpusavyje 
    bendradarbiaujantys ir sąveikaujantys padaliniai dirba rezultatyviau
    negu jie dirbtų atskirai.
\end{itemize}

\subsection{Situacinis (atsitiktinumų) požiūris (XX a. 9-asis dešimtmetis)}

\begin{itemize}
  \item Teigia, kad nėra universalaus valdymo, reikia aprašinėti ir
    nagrinėti konkrečias situacijas, surasti teisingą jų sprendimą,
    tinkamą ir kitoms panašioms situacijoms.
  \item Vadovas turi sugebėti atsakyti į klausimą „Kuris metodas čia
    tiktų?“
\end{itemize}

\subsection{Šiuolaikinės vadybos tendencijos}

\begin{itemize}
  \item Nuolat besimokanti organizacija.
  \item Bandoma paneigti istoriškai susiformavusią valdymo piramidę.
    Ji verčiama ant šono. Horizontalėje koordinuojami ryšiai.
  \item Darnioji plėtra (Vikipedija: sąvoka, apimanti visumą metodų, kuriais
    siekiama užtikrinti vystymąsi, tenkinantį žmonių gerovę dabartyje,
    nesumažinant žmonių gerovės galimybių ateityje).
  \item Žinių vadyba (Vikipedija: tai organizacijų praktikos arba metodai,
    kurių pagalba atpažįstamos, organizuojamos, kuriamos, platinamos žinios
    jų atkartojimui, mokymui ir įsisąmoninimui).
  \item Kokybės vadyba.
\end{itemize}

\chapter{Organizacija ir jos aplinka}

\begin{defn}[Organizacija]
  Grupė žmonių, kurių veikla sąmoningai koordinuojama siekiant bendrų
  tikslų.

  \begin{itemize}
    \item Ją sudaro bent du žmonės.
    \item Ji turi bent vieną visiems jos nariams svarbų tikslą.
    \item Organizacijos nariai sąmoningai drauge dirba bendram tikslui
      pasiekti.
  \end{itemize}
\end{defn}

\section{Formalios ir neformalios organizacijos}

\begin{tabularx}{\textwidth}[]{l l l}
  & Formali organizacija & Neformali organizacija \\
  Tikslai & pelnas, rinkos dalis, įvaizdis & narių poreikių patenkinimas \\
  Santykiai & oficialūs & neoficialūs \\
  Valdžios šaltiniai & delegavimas & lyderiavimas \\
  Elgsenos reikalavimai & taisyklės & normos \\
  Tarpusavio sąveikos pagrindas & pareigos & asmenybės savybės, statusas
\end{tabularx}

\section{Organizacijos charakteristikos}

\begin{itemize}
  \item Tikslai.
  \item Ištekliai.
  \item Darbo pasidalijimas.
  \item Organizavimas.
  \item Valdymo būtinumas.
  \item Vidinių sąlygų ir išorinės aplinkos įtaka.
\end{itemize}

\section{Organizacijos vidinė aplinka}

\begin{description}
  \item[Tikslas] – pagrindinė organizacijos siekiamybė, veiklos orientyras
    ir vertinimo kriterijus, kuriam pasiekti nukreipta visa įmonės veikla
    ir taikomos priemonės.
  \item[Struktūra] – atskirų pareigų, grupių ir padalinių tarpusavio
    ryšys, priklausomybė, pavaldumas.
  \item[Užduotys] – organizacijos tikslų konkretizavimas, darbai, kurie
    turi būti atlikti tam tikru laiku tam tikroje vietoje.
  \item[Technologija] – užduočių įvykdymui naudojamas įrenginių, įrankių
    ir techninių žinių derinys.
  \item[Žmonės] – svarbiausias vidinės aplinkos elementas. Jie kuria
    organizaciją, valdo ją ir vykdo jos tikslus.
\end{description}

\section{Tiesioginio poveikio aplinkos veiksniai}

\begin{itemize}
  \item Klientai.
  \item Tiekėjai.
  \item Vyriausybė.
  \item Specialiųjų interesų grupės.
  \item Žiniasklaida.
  \item Profesinės sąjungos.
  \item Finansų institucijos.
  \item Konkurentai.
\end{itemize}

\section{Netiesioginio poveikio aplinkos veiksniai}

\begin{itemize}
  \item Politiniai-teisiniai.
  \item Ekonominiai.
  \item Socialiniai.
  \item Technologiniai.
\end{itemize}

\section{Organizcijos išorinės aplinkos charakteristikos}

\begin{description}
  \item[Judrumas] – aplinkos pokyčių greitis.
  \item[Neapibrėžtumas] – organizacijos informacijos apie konkretų
    veiksnį apimtis ir patikimumas.
  \item[Sudėtingumas] – veiksnių, į kuriuos turi reaguoti organizacija,
    skaičius. Kuo daugiau veiksnių daro įtaką, tuo aplinka, kurioje
    ji veikia yra sudėtingesnė.
  \item[Aplinkos veiksnių priklausomybė] – vieno veiksnio pasikeitimas
    gali lemti kitų veiksnių pokyčius.
\end{description}

\section{Organizacijos elgesys aplinkos atžvilgiu}

\begin{description}
  \item[Adaptavimasis] – prisitaikymas prie aplinkos savybių ir sąlygų.
  \item[Integravimasis] – kompleksinis jungimasis su aplinka.
  \item[Priešinimasis] – nesutikimas su aplinkos diktuojamomis sąlygomis.
  \item[Pajungimas] – aplinkos įsisavinimas ir pritaikymas savo poreikiams.
\end{description}

\chapter{Organizacijų socialinė atsakomybė}

\begin{defn}[Socialinė atsakomybė]
  Suinteresuotų šalių poreikių tenkinimas, išeinantis už organizacijos
  ribų ir besiremiantis jos etinėmis paskatomis prisiimti atsakomybę
  už savo vykdomos veiklos poveikį aplinkai, visuomenei, darbuotojams.
\end{defn}

\begin{defn}[Socialinė atsakomybė]
  Sąmoningai formuojamų ekonominių, politinių, dorovinių santykių
  tarp organizacijos ir visuomenės, įvairių jos struktūrų forma;
  pasirengimas atsakyti už savo poelgius ir veiksmus.
\end{defn}

\section{Socialinės atsakomybės dimensijos}

\begin{description}
  \item[Filantropinė atsakomybė] – gerinti bendruomenės gyvenimą, siekti
    visuomenės gerovės.
  \item[Etinė atsakomybė] – remtis sąžiningumu, teisingumu, dora, vengti
    žalos aplinkai.
  \item[Juridinė atsakomybė] – laikytis įstatymų.
  \item[Ekonominė atsakomybė] – siekti pelno.
\end{description}

\section{Socialinės atsakomybės kryptys}

\begin{itemize}
  \item Atsakomybė darbuotojams.
  \item Atsakomybė ekonominei aplinkai.
  \item Atsakomybė natūraliai aplinkai.
  \item Atsakomybė visuomenei.
\end{itemize}

\section{Organizacijos reakcija į socialines problemas}

Pagal reagavimo momentą.

\begin{description}
  \item[Besipriešinanti] – kompanija tik atsiliepia į socialinę problemą
    po to, kai ji meta iššūkį kompanijos tikslams.
  \item[Gynybinė] – kompanija veikia, kad apsigintų nuo jai metamo
    iššūkio.
  \item[Prisitaikomoji] – kompanija veikia pagal vyriausybės reikalavimus
    ir visuomenės nuomonę.
  \item[Numatanti] – kompanija iš anksto numato poreikius, kurie dar nėra
    išreikšti.
\end{description}

\section{Socialiai atsakingas verslas reiškiasi}

Per:
\begin{itemize}
  \item labdarą;
  \item socialini̇ų programų vystymą;
  \item papildomas informavimo ir švietimo priemones vartotojams.
\end{itemize}

\section{Argumentai už socialinę atsakomybę}

\begin{itemize}
  \item Socialiniai veiksmai gali būti pelningi.
  \item Ilgalaikės perspektyvos – didėjanti visuomenės gerovė duoda
    pagrindą vartojimo augimui.
  \item Pagerina organizacijos įvaizdį.
  \item Tai pagerina verslo sistemos gyvybingumą. Verslas egzistuoja,
    kadangi jis duoda visuomenei naudą. Visuomenė gali atsiimti savo
    dalį. Tai yra atsakomybės geležinis įstatymas.
  \item Tai pagerins ilgą laiką akcijų kainą, kadangi birža labiau
    apsaugos firmą nuo rizikos ir viešų visuomenės atakų ir todėl
    suteiks jai galimybę gauti aukštesnę kainą, proporcingą indėliui.
\end{itemize}

\section{Argumentai prieš socialinę atsakomybę}

\begin{itemize}
  \item Socialinės atsakomybės kaina yra pernelyg didelė ir dar smarkiai
    didės.
  \item Tai gali susilpninti mokėjimų balansą, kadangi prekių kainos kils,
    tam kad apmokėti socialines programas.
  \item Tai sumenkins verslo pirmaeilius tikslus.
  \item Visuomenės „gelbėtojo“ vaidmenį turi atlikti vyriausybė.
\end{itemize}

\chapter{Organizacijos kultūra}

\begin{defn}[Organizacijos kultūra]
  Vertybių sistema, suprantama ir priimtina visiems organizacijos
  nariams, leidžianti organizacijai kryptingai veikti bei palaikoma
  organizacijos istorijos, tradicijų, ceremonijų ir t.t. Tap pat
  padedanti išsiskirti iš kitų organizacijų.
\end{defn}

\section{Kuo svarbi organizacijos kultūra?}

\begin{itemize}
  \item Suteikia organizacijai unikalumo, išskiria organizaciją iš kitų.
  \item Darbuotojų elgseną daro nuoseklesnę, nes pateikia standartus,
    kaip reikėtų elgtis organizacijoje.
  \item Stiprina bendrumo jausmą, nes nariai puoselėja panašias vertybes.
  \item Priimtina kultūra motyvuoja darbuotojus, jiems mieliau dirbti
    organizacijoje.
\end{itemize}

\section{Organizacinės kultūros charakteristikos}

\begin{description}
  \item[Asmeninė iniciatyva] – darbuotojų savarankiškumo ir atsakomybės
    laipsnis.
  \item[Rizikos toleravimas] – laipsnis, kuriuo darbuotojai skatinami
    rizikuoti.
  \item[Veiklos kryptis] – laipsnis, kiek aiškiai organizacija nurodo
    veiklos tikslus ir lūkesčius, darbuotojų suvokimas ko iš jų tikimasi.
  \item[Integracija] – laipsnis, kuriuo darbuotojai skatinami veikti
    koordinuotai.
  \item[Vadovybės parama] – laipsnis, kuriuo vadovai suteikia aiškius
    nurodymus ir paramą savo pavaldiniams.
  \item[Kontrolė] – tiesioginės priežiūros laipsnis darbuotojų elgesiui
    kontroliuoti.
  \item[Identiškumas] – laipsnis, kuriuo darbuotojai tapatina save su
    organizacija, kaip visuma.
  \item[Atlyginimo visuma] – laipsnis, kuriuo atlyginimų sistema remiasi
    darbuotojų veikla ir pasiekimais.
  \item[Konfliktų toleravimas] – laipsnis, kuriuo darbuotojai yra skatinami
    atvirai iškelti konfliktus ir juos diskutuoti.
  \item[Komunikacijos] – kiek organizacijoje yra neformalios komunikacijos.
\end{description}

\section{Organizacijos kultūrą formuoja}

\begin{itemize}
  \item Pačios organizacijos istorija, organizacijos įkūrėjai.
  \item Atrankos procesas.
  \item Vadovavimo stilius.
  \item Socializacijos procesas.
  \item Nacionalinė kultūra, organizacijos klientai bei šalies visuomenės
    normos.
\end{itemize}

\section{Organizacijos kultūra gali būti}

\begin{tabularx}{\textwidth}[]{X X}
  Stipri & Silpna \\
  organizacijos pagrindinės vertybės plačiai pripažįstamos &
  nėra aiškių vertybių bei įsitikinimų, neaišku, kaip siekti sėkmės
  plėtojant savo verslą \\
  asmeniniai darbuotojų įsitikinimai ir vertybės artimos organizacijos
  vertybėms & daugybė įsitikinimų, nesutariama, kurie iš jų svarbiausi \\
  sukuria lojalumą, atsidavimą organizacijai, mažindama darbuotojų kaitą
  & padaliniai turi iš esmės skirtingus įsitikinimus \\
  sunkiau prisitaikyti prie pokyčių & ritualai dezorganizuoti 
\end{tabularx}

\section{Organizacijos kultūrų tipai pagal Sonnefeld}

Beisbolo komanda:
\begin{itemize}
  \item naujovių diegimas, susietas su rizika;
  \item talento įvertinimas ir vystymas;
  \item plati veikimo laisvė;
  \item svarbūs darbo rezultatai;
  \item didelis finansinis atlyginimas;
  \item individualus pripažinimas;
  \item darbuotoją „perka“;
  \item pavyzdžiai: investiciniai bankai, televizijos, programuotojai,
    pramogų industrija.
\end{itemize}

Klubas:
\begin{itemize}
  \item pagarba pagyvenusiems darbuotojams;
  \item svarbiausi veiksniai: lojalumas, patirtis, amžius;
  \item „bendra visuotinė“ žingsnis po žingsnio progresyvi karjera;
  \item darbuotoją „užsiaugina“;
  \item pavyzdžiai: oro linijų bendrovės, bankai, valstybės institucijos.
\end{itemize}

Tvirtovė:
\begin{itemize}
  \item nėra siūlomas saugumas, veiklos pastovumas ar pusiausvyra;
  \item pasitaikius progai siūlo pakeisti nuomonę;
  \item pavyzdžiai: viešbučiai, mažmeninė prekyba, leidyba…
\end{itemize}

Akademija:
\begin{itemize}
  \item pabrėžiamas sisteminis karjeros vystymasis;
  \item nuolatinis tobulėjimas;
  \item specializuotas darbas;
  \item pavyzdžiai: elektronikos, mašinų gamintojai, farmacija.
\end{itemize}

\section{Organizacijos kultūrų tipai pagal Steinman ir Schreyogg}

\begin{description}
  \item[Paranojinė] – darbuotojų santykiuose vyrauja nepasitikėjimas ir
    baimė, bendradarbiai ieško vieni kitų klaidų. Veiksmuose įžvelgiamas
    noras pakenkti. Bet koks veiklos sutrikimas sukelia reakciją.
  \item[Depresinė] – vyrauja pesimistinės nuotaikos, tikima lemtimi, likimu.
    Nuolat laukiama paramos iš kitų organizacijų ar valstybės. Veikla
    rutiniška, vidinė aplinka sustabarėjusi. Darbuotojai vadovaujasi
    principu „aš nieko pakeisti negaliu, nes esu „mažas““.
  \item[Paremta prievarta] – veikla organizuojama remiantis
    administracinėmis priemonėmis. Pagrindiniai akcentai – drausmė, tvarka,
    paklusnumas. Stiprus ir gerai sureguliuotas kontrolės mechanizmas.
    Emocijos nepripažįstamos. Hierarchija – pagrindinis vadovavimo
    principas.
  \item[Šizoidinė kultūra] – aukščiausiojo lygio vadovai atitolę nuo
    pavaldinių. Pavaldiniais nepasitikima. Darbuotojų tarpusavio santykiai
    labai formalūs, „šalti“. Žemesniuose lygiuose vyksta kova dėl geresnio
    posto. Didelę įtaką turi favoritai, neformalios grupuotės. Dominuoja
    karjerizmas.
  \item[Oportunistinė kultūra] – bendradarbių santykius lemia tradicijos,
    įpročiai. Neigiama individualybė, išskirtinumas. Didžiulis dėmesys
    procedūroms, o ne reikalo esmei. Jokie pokyčiai nepageidaujami.
  \item[Įsipareigojanti kultūra] – TODO
\end{description}

\section{Ouchi organizacijos kultūros modelis}

\begin{tabularx}{\textwidth}[]{c c c c}
  & Japonijos firma & Z JAV firma & JAV firma \\
  Įsipareigojimas darbuotojams & visam gyvenimui & ilgalaikis & 
  trumpalaikis \\
  Darbo įvertinimas & lėtas pagal kokybę & lėtas pagal kokybę &
  greitas pagal kokybę \\
  Karjera & labai plati & pakankamai plati & siaura \\
  Kontrolė & numatoma ir neformali & numatoma ir neformali & tiksli ir
  formali \\
  Sprendimų priėjimas & grupinis & grupinis & individualus \\
  Atsakomybė & grupinė & individuali & individuali \\
  Dėmesys žmonėms & visapusiškas & mažas & visapusiškas
\end{tabularx}

\chapter{Valdymo kultūrų geografija}

Kodėl reikia nagrinėti kultūrinius vadybos ypatumus:
\begin{itemize}
  \item galimybė dirbti užsienio kompanijoje;
  \item bendradarbiavimas su užsienio kompanijomis (tiekėjai, klientai);
  \item rinkos plėtra į užsienį.
\end{itemize}

\section{Trys pagrindiniai valdymo kultūros arealai}

\begin{itemize}
  \item Amerikietiškasis (JAV ir Kanada, neįeina Lotynų Amerika).
  \item Japoniškasis (5 tigrai).
  \item Europietiškasis (ES šalys).
\end{itemize}

\section{Amerikietiškasis valdymas (JAV)}

\begin{itemize}
  \item Įsitikinimas, kad žmogus yra padėties šeimininkas ir kiekvienas
    pats kuria savo ateitį. Nevykėlis yra pats kaltas.
  \item Būtinas pažado įvykdymas.
  \item Tikėjimas tuo, kad žmogus turi užimti pareigas, atitinkančias
    jo nuopelnus: žmogaus statusą turi lemti jo nuopelnai.
  \item Savo nuomonės slėpimas, informacijos iškreipimas vertinamas
    kaip negarbingumas.
  \item Sprendimų decentralizavimas, akcentuojant pasitikėjimą
    darbuotojais, tikėjimą tuo, kad dauguma žmonių pozityviai reaguoja
    į jų atsakomybės augimą. Geriausias kelias vystymuisi – dalyvavimas
    sprendimų priėmime.
  \item Laikas – pinigai.
  \item Bendravimas neformalus, vengiama titulų.
\end{itemize}

\section{Amerikiečių derybų ypatumai}

FIXME: Praleista.

\chapter{Planavimo funkcija}

\begin{defn}[Planavimas]
  Vadybos funkcija, apimanti tikslų formulavimo bei jų įgyvendinimui
  tinkamos veiksmų eigos sudarymo procesą.
\end{defn}

Planavimo nauda:
\begin{itemize}
  \item planavimas pats savaime negarantuoja sėkmės, tačiau veda į ją;
  \item neplanuodami darbų, darbuotojai negali žinoti savo veiklos
    rezultatų vertės;
  \item apdraudžia nuo netikėtumų, racionaliai panaudojami ištekliai;
    planavimas neturi alternatyvos. % FIXME Paaiškinti kodėl.
\end{itemize}

\section{Planavimo lygmenys}

\begin{description}
  \item[Strateginis planavimas] – ilgalaikių tikslų apibrėžimas ir jų
    vykdymo numatymas visos organizacijos mastu.
  \item[Operatyvinis planavimas] – orientuotas į trumpalaikius tikslus
    ir lokalias, svarbias tik atskiriems padaliniams ar net darbuotojams,
    priemones, kurių pagalba yra pasiekiami ilgalaikiai tikslai.
\end{description}

\section{Organizacijos planų lygiai}

\begin{description}
  \item[Strateginiai planai] – tai ilgalaikės veiklos planai, organizacijos
    veiklos kryptys.
  \item[Taktiniai planai (dar vadinami etapiniais)] – tai padalinių
    ar veiklos sričių planai.
  \item[Operatyviniai planai (dar vadinami einamaisiais)] – trumpalaikiai
    planai konkrečioms užduotims įgyvendinti, užtikrina taktinių planų
    vykdymą.
\end{description}

\section{Organizacijos planų tipai}

Vienkartiniai planai:
\begin{description}
  \item[programos] – ilgalaikės veiklos ir globalių tikslų įgyvendinimo
    planų rinkiniai (pavyzdžiui, inovacijų programos);
  \item[projektai] – atskiros programos dalys; jie yra ribotos apimties
    ir turi aiškias paskirties bei laiko direktyvas.
\end{description}

Nuolatiniai planai:
\begin{description}
  \item[organizacijos politika] – organizacijos bendrų nuostatų, nuorodų
    sistema, apibrėžianti sprendimų priėmimo ribas;
  \item[taisyklės] – nurodo konkrečius veiksmus, kuriuos būtina arba
    draudžiama atlikti tam tikroje situacijoje;
  \item[procedūros] – nuorodos, standartiniai metodai arba instrukcijų
    rinkiniai, nurodantys seką veiksmų, kuriuos reikia atlikti dažnai
    ar sistemingai.
\end{description}

\section{Planavimo etapai}

\begin{enumerate}
  \item Tikslų nustatymas.
  \item Organizacijos išorinės ir vidinės aplinkos analizė.
  \item Strategijos parinkimas.
  \item Rezultatų vertinimas.
\end{enumerate}

\section{Strateginis valdymas}

\begin{defn}[Strategija]
  Sprendimų visuma, apibrėžianti organizacijos svarbiausius ateities
  tikslus ir veiksmus bei priemones tiems tikslams pasiekti.
\end{defn}

\begin{defn}[Strateginis valdymas]
  Nuolatinis, dinaminis ir nuoseklus procesas, kuriuo remiantis
  organizacija laiku prisitaiko prie išorinės aplinkos pokyčių ir
  efektyviau panaudoja savo išteklių potencialą.
\end{defn}

\section{Organizacijos strategijų rūšys (I)}

\begin{description}
  \item[Riboto augimo] \label{strategija:rusys:ribotas} – organizacija
    siekia stabilumo (nesiplėšo kovodama dėl savo rinkos dalies,
    stengiasi ją didinti nuosekliai, daro nuoseklias ir nedideles
    investicijas).
  \item[Augimo] – organizacija siekia užkariauti kuo didesnę rinkos
    dalį, daro dideles investicijas.
  \item[Mažinimo] – organizacija mažina investicijas, atsisako padalinių,
    darbuotojų ir panašiai. Šios strategijos organizacijos dažniausiai
    imasi, kai joms gresia bankrotas.
  \item[Mišri] – organizacija vienoje srityje imasi vienos strategijos
    (pavyzdžiui, augimo), o kitoje kitos (pavyzdžiui, mažinimo). Taip
    ji galėtų elgtis, kai siekia „persikelti“ iš vienos rinkos į kitą.
\end{description}

\section{Organizacijos strategijų rūšys (II)}

\begin{description}
  \item[Palaikymo strategija] – tas pats, kas ir „riboto augimo“
    (\ref{strategija:rusys:ribotas}).
  \item[Rinkos vystymo strategija] – organizacija plečiasi į kitas
    rinkas. % TODO Patikrinti.
  \item[Savo kiemo strategija] – turi savo klientų ratą ir jiems
    siūlo savo prekės pakaitalus. Pavyzdžiui, jogurto gamintojai
    pradėjo siūlyti sūrelius. % TODO Patikrinti.
  \item[Rinkos užkariavimo strategija] – eina į visiškai naują rinką.
    % TODO Patikrinti.
\end{description}

\section{Bostono konsultacinės grupės matrica}

TODO

\section{Sprendimų priėmimas}

\begin{defn}[Sprendimų priėmimas]
  Veiksmų krypties konkrečiai problemai spręsti nustatymas ir parinkimas.
  \begin{itemize}
    \item Bendriausia prasme valdymo sprendimas yra reakcija į problemas.
    \item Pagal supaprastintą versiją valdymo sprendimai yra alternatyvos
      pasirinkimas.
  \end{itemize}
\end{defn}

\subsection{Problemos diagnostika}

\begin{description}
  \item[Nukrypimas nuo buvusios veiklos:] staigus pokytis
    vykstančiame procese dažnai rodo, kad problema pradeda vystytis.
    Pavyzdžiui, mažėja pardavimai, didėja išlaidos, blogėja produkto
    kokybė.
  \item[Nukrypimas nuo plano:] rezultatai nepasiekia planuotų
    tikslų.
  \item[Išorinė kritika:] klientai nepatenkinti produkcija, ar jos
    pristatymo grafiku ir panašiai.
  \item[Potenciali galimybė gali būti traktuojama kaip problema] .
\end{description}

\subsection{Procesas}

\begin{enumerate}
  \item \label{enum:sprendimai:procesas:tikslai} Tikslai.
  \item \label{enum:sprendimai:procesas:problema} Problemos nustatymas.
  \item Duomenų rinkimas.
  \item Problemos analizė, kliūčių nustatymas.
  \item Alternatyvių sprendimų formulavimas.
  \item Alternatyvų įvertinimas.
  \item Geriausio sprendimo priėmimas.
  \item Įdiegimas.
  \item Monitoringas ir įvertinimas.
  \item Atgal į \ref{enum:sprendimai:procesas:tikslai} arba
    \ref{enum:sprendimai:procesas:problema}.
\end{enumerate}

\subsection{Veiksniai, darantys įtaką sprendimų priėmimui}

\begin{itemize}
  \item Vadovo vertybinė orientacija.
  \item Sprendimų priėmimo aplinka – reikia atsižvelgti į riziką.
  \item Laiko veiksnys, laiko reikšmė – laikas dažnai sąlygoja situacijos
    pasikeitimą.
  \item Informacijos ribotumas.
\end{itemize}

\subsection{Grupiniai valdymo sprendimai}

Privalumai:
\begin{itemize}
  \item jėgų sutelkimas;
  \item darbo specializacija;
  \item priimtas sprendimas lengviau įgyvendinamas.
\end{itemize}

Trūkumai:
\begin{itemize}
  \item laiko švaistymas;
  \item grupiniai konfliktai;
  \item grupės lyderių neigiama įtaka (bauginimas).
\end{itemize}

\subsection{Metodai}

\begin{description}
  \item[Ekspertinis sprendimo priėmimas] – TODO
  \item[Grupės narių nuomonių vidurkio sprendimas] – TODO
  \item[Autoritarinis sprendimas be grupės diskusijos] – TODO
  \item[Autoritarinis sprendimas po grupės diskusijos] – TODO
  \item[Oficialaus valdžios pareigūno arba įgalioto asmens sprendimas] –
    TODO
  \item[Konsensuso metodas] – TODO
\end{description}
