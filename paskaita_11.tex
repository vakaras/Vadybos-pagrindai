\begin{defn}[Motyvai]
  Tikslai, kurie mus skatina kažkaip elgtis.
\end{defn}

\begin{defn}[Motyvavimas]
  Poveikis mūsų motyvacijai.
\end{defn}

Valdantysis personalas turi sugebėti motyvuoti darbuotojus:
\begin{itemize}
  \item pasilikti organizacijoje;
  \item patikimai atlikti pavestas užduotis (jei nenori daryti užduoties,
    tai liepiant jam atlikti tą užduotį, jis ją padarys blogiau);
  \item savanoriškai užsiimti kokia nors kūrybiška novatoriška veikla.
\end{itemize}

\section{Ankstyvieji motyvacijos požiūriai}

\begin{description}
  \item[Tradicinis modelis] – bizūno ir meduolio koncepcija (jei elgiasi
    blogai – reikia barti, o jei gerai – duoti meduolį). Paminėtinas
    Teiloras.
  \item[Žmonių santykių modelis] – TODO:(Bi~ teorija? Žmogų labai skatina
    jo socialinių poreikių tenkinimas.)
  \item[Žmonių išteklių modelis] – teorija X ir Y. (X teorijos atveju
    prievartiniais metodais, nuobaudomis, o Y atveju – duoti sąlygas,
    kad jis galėtų realizuotis. Remiantis Y teorija atsirado darbo
    praturtinimas: kuo TODO: )
\end{description}

\section{Šiuolaikinės motyvacijos teorijos}

\begin{description}
  \item[Turinio teorijos] – kiekvienas žmogus turi kokių nors poreikių
    ir tenkindami ar netenkindami jų mes žmogų kur nors nukreipiame.
  \item[Proceso teorijos] 
  \item[Pastiprinimo teorija] – akcentuoja ne žmogaus elgesį, …?
\end{description}

\section{A. Maslow poreikių hierarchija}

Saugumas – užtikrinimas to, kad būsime apsaugoti nuo egzistencinių
dalykų trūkumo.

Pirmos trys pakopos: žemiausio lygio poreikiai, pagrindiniai.
Likusios dvi: aukštesnieji, asmeniniai poreikiai.

Pagrindinė piramidės savybė: žmogus neturinčiam žemesnio lygio, aukštesnis
nerūpi.

\section{D. Mc Clelland poreikių teorija}

Laimėjimų pavyzdys: sportininkai. Siekiant rezultato 
Jei žmogus turi laimėjimo poreikį, tai jis nori kažką išmokti ir už
tai gauti diplomą, o neturintis laimėjimo poreikio nemato problemų
nusipirkti diplomą.

\begin{exmp}
  Tarkime studentų grupei reikia padaryti pristatymą. Laimėjimo poreikį
  turintis žmogus pasirinks moksliukus, o tas kuris turi bendrumo
  poreikį – pasirinks draugus.
\end{exmp}

\section{F. Herzberg dviejų veiksnių teorija}

Alternatyvus pavadinimas: Higienos–motyvatorių teorija arba demotyvatorių–
motyvatorių teorija.

Higienos veiksnių neįgyvendinimas – reiškia, kad žmogus bus demotyvuotas, 
bet jų įgyvendinimas neiššauks pasitenkinimo.

Veiksniai-motyvatoriai – reiškia, kad jų patenkinimas iššauks pasitenkinimą,
o nepatenkinimas neiššauks nepasitenkinimo.

\section{Lūkesčių teorija (Lawler, Vroom ir Porter)}

Mes pasirenkame kokias pastangas norime dėti ir kokį veiklos rezultatą
gausime.

Atlygio patrauklumas – valentingumas.

Svarbiausia: ryšiai tarp blokų. Instrumentalumas: strėlytė tarp veiklos
rezultato ir atlygio. Lūkesčiai: pastangos → veiklos rezultatas.

\section{Adamso teisingumo teorija}

Teorija koncentruojasi į kažkokį vieną konkretų žmogų.

Žmogus jaučiasi patenkintas tada, kai įdėtas atlygis atitinka gautą
rezultatą.

Statistiškai žmogus dažniausiai jaučiasi nepatenkintas, kai jis gauna
didesnį atlygį, negu jo įdėtos pastangos.

Žymėjimai: I – indėlis, A – atlygis.

\section{B. Skinner pastiprinimo teorija}

Nagrinėja kaip galima modifikuoti mūsų elgesį priklausomai nuo mūsų
norų ir mūsų tikslų.

Ši teorija nagrinėja buvusio elgesio rezultatą naujam elgesiui.

Leido vadovams išvystyti elgesio modifikavimo teorija.

Pastiprinimo tipai:


Robins Organizacinio 

%Kolis: [Planavimas; Motyvacija]

% Klausimų pavyzdžiai:
% \begin{itemize}
%   \item Išvardinkite lūkesčių teorijos tris pagrindinius veiksnius.
%   \item Vienas studentas pasirinko kompetetingus, o kitas draugiškus.
%     Kokia jo motyvacija.
% \end{itemize}
