\chapter{Kontrolė}

\section{Kas yra kontrolė?}

\begin{defn}[Kontrolė]
  Bet kurios veiklos srities tikrinimas, priežiūra, stebėjimas.
\end{defn}

\begin{defn}[Kontrolė]
  Organizacijos veiklos įvertinimo procesas, atsižvelgiant į tikslus,
  ir koregavimo veiksmų atlikimas, stengiantis išlaikyti organizaciją
  kelyje į tikslų pasiekimą.
\end{defn}

\begin{defn}[Kontrolė]
  Procesas, kuriuo siekiama užtikrinti, kad reali veikla atitiktų
  planuojamą, o faktinė situacija – norimą. (\emph{Vikipedija})
\end{defn}

\section{Kontrolės rūšys}

\begin{description}
  \item[Pirminė (parengiamoji) kontrolė] – vykdoma iki organizacijos
    faktinės veiklos pradžios. Pagrindiniai parengiamosios kontrolės
    instrumentai yra atitinkamos taisyklės, procedūros, elgsena.
    Parengiamoji kontrolė taikoma tiesiogiai vykdant darbus.
  \item[Taktinė (einamoji) kontrolė] – vyksta reguliariai, darbo
    eigoje, aptariant iškilusius klausimu ir pasiūlymus darbui
    gerinti. Ji padeda išvengti atotrūkio nuo planų ir instrukcijų.
  \item[Galutinė (baigiamoji) kontrolė] – jos metu gauti rezultatai
    yra palyginami su planuotais. Vadovybė turi galimybę geriau
    įvertinti, ar planai bus realūs.
\end{description}

\section{Kontrolės procesas}

\begin{enumerate}
  \item \label{enum:control_process_01} Veiklos atlikimo standartų
    sukūrimas.
  \item Veiklos atlikimo lygio įvertinimas.
  \item Atliktos veiklos palyginimas su standartais.
  \item Koregavimo veiksmai ir atgal į \ref{enum:control_process_01}.
\end{enumerate}

\section{Standartų rūšys}

\begin{itemize}
  \item Našumo standartai.
  \item Išlaidų standartai.
  \item Kokybės standartai.
  \item Elgsenos standartai.
  \item Laiko standartai.
\end{itemize}

\section{Kontrolės formos}

\begin{itemize}
  \item Apsilankymas darbo vietoje.
  \item Problemų aptarimas posėdžiuose, seminaruose, pasitarimuose.
  \item Kontrolė ataskaitų forma.
  \item Savikontrolė.
  \item Išorinė kontrolė – auditas.
\end{itemize}

\section{Gautų darbo rezultatų lyginimo su nustatytomis normomis būdai}

FIXME

\begin{itemize}
  \item Matavimas.
  \item Registravimas.
  \item Skaičiavimas.
  \item Organoleptinis.
  \item Apklausų.
  \item Ekspertizės.
\end{itemize}

\section{Veiklos koregavimo alternatyvos}

\begin{itemize}
  \item Nieko nedaryti.
  \item Pašalinti nukrypimus.
  \item Keisti standartą.
\end{itemize}

\section{Efektyvios kontrolės požymiai}

\begin{itemize}
  \item Strateginis kontrolės pobūdis.
  \item Orientacija į rezultatus.
  \item Atitiktis veiklai.
  \item Kontrolės savalaikiškumas.
  \item Kontrolės lankstumas.
  \item Kontrolės ekonomiškumas.
\end{itemize}
