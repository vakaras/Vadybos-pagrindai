\chapter{Kontrolė}

\section{Kas yra kontrolė?}

\begin{defn}[Kontrolė]
  Bet kurios veiklos srities tikrinimas, priežiūra, stebėjimas.
\end{defn}

\begin{defn}[Kontrolė]
  Organizacijos veiklos įvertinimo procesas, atsižvelgiant į tikslus,
  ir koregavimo veiksmų atlikimas, stengiantis išlaikyti organizaciją
  kelyje į tikslų pasiekimą.
\end{defn}

\begin{defn}[Kontrolė]
  Procesas, kuriuo siekiama užtikrinti, kad reali veikla atitiktų
  planuojamą, o faktinė situacija – norimą. (\emph{Vikipedija})
\end{defn}

Kontrolės funkcija apima faktinių rezultatų palyginimą su tikslais
bei nukrypimų korekciją.

\section{Kontrolės rūšys}

\begin{description}
  \item[Pirminė (parengiamoji) kontrolė] – vykdoma iki organizacijos
    faktinės veiklos pradžios. Pagrindiniai parengiamosios kontrolės
    instrumentai yra atitinkamos taisyklės, procedūros, elgsena.
    Parengiamoji kontrolė taikoma tiesiogiai vykdant darbus.
  \item[Taktinė (einamoji) kontrolė] – vyksta reguliariai, darbo
    eigoje, aptariant iškilusius klausimu ir pasiūlymus darbui
    gerinti. Ji padeda išvengti atotrūkio nuo planų ir instrukcijų.
  \item[Galutinė (baigiamoji) kontrolė] – jos metu gauti rezultatai
    yra palyginami su planuotais. Vadovybė turi galimybę geriau
    įvertinti, ar planai bus realūs.
\end{description}

\section{Kontrolės procesas}

\begin{enumerate}
  \item \label{enum:control_process_01} Veiklos atlikimo standartų
    sukūrimas.
  \item Veiklos atlikimo lygio įvertinimas.
  \item Atliktos veiklos palyginimas su standartais.
  \item Koregavimo veiksmai ir atgal į \ref{enum:control_process_01}.
\end{enumerate}

\section{Standartų rūšys}

\begin{itemize}
  \item Našumo standartai.
  \item Išlaidų standartai.
  \item Kokybės standartai.
  \item Elgsenos standartai – etikos, išoriniai dėmesio klientui
    reikalavimai ir panašiai.
  \item Laiko standartai – darbo laiko pradžia ir pabaiga, reglamentuotos
    pertraukos ir panašiai.
\end{itemize}

\section{Kontrolės formos}

\begin{itemize}
  \item Apsilankymas darbo vietoje. Būdingos metodikos taikymo
    klaidos:
    \begin{itemize}
      \item pasiduodama nuotaikoms – einama tiesiog „išsilieti“ ant
        pavaldinio;
      \item lankoma be išankstinio apgalvojimo, neturint aiškaus
        tikslo.
    \end{itemize}
  \item Problemų aptarimas posėdžiuose, seminaruose, pasitarimuose.
  \item Kontrolė ataskaitų forma – dirbantieji pateikia ataskaitas
    apie tai, kaip jiems sekėsi dirbti. Naudojama tada, kai galima
    užrašyti ir suskaičiuoti darbo rezultatus. Būdingos metodikos
    taikymo klaidos:
    \begin{itemize}
      \item savitikslės ataskaitos;
      \item nepateisina sąnaudų;
      \item neužtikrinamas grįžtamasis ryšys.
    \end{itemize}
    Ataskaitos gali būti analitinės (atliekama faktų analizė) ir
    informacinės (tikrinamas turimų faktų teisingumas).
  \item Savikontrolė – pavyzdžiui, organizacijos inventorizacija, kai
    tikrinami organizacijos įrenginiai, daiktai. Kitas pavyzdys:
    dėstytojas duoda anketas studentams, kad išreikštų nuomonę apie
    jo darbą.
  \item Išorinė kontrolė – auditas (ateina žmonės iš išorės ir tikrina
    organizacijos darbą). Auditas gali būti atliekamas dėl išorės
    „noro“ siekiant užtikrinti organizacijos procesų atitikimą
    bendriems reikalavimams (pavyzdžiui, mokesčių inspekcija tikrina
    ar tvarkingos organizacijos finansinės ataskaitos) ir dėl
    organizacijos vidinio „noro“: organizacija kviečiasi
    specialistus, kad jie peržiūrėtų jos procesus ar patikrintų
    sukurtų produktų kokybę. (Pavyzdžiui, IT kompanija prieš
    pateikdama savo sukurtą programinę įrangą į rinką, užsisako
    to produkto testavimą kitoje kompanijoje.)
\end{itemize}

\section{Gautų darbo rezultatų matavimo ir lyginimo su nustatytomis
normomis būdai}

\begin{itemize}
  \item Matavimas – išmatuoti objektyviais matavimo kriterijais: minutėm,
    valandom, centimetrais ar panašiai. Pavyzdžiui, kubo kraštus matuojame
    centimetrais.
  \item Registravimas – įvykių registravimas. Užregistruojama darbo
    eiga ir palyginama su standartais. Pavyzdžiui, nustatėme normą padavėjui
    šypsotis, tai stebėdami jį ir registruodami ką jis veikia, galime
    palyginti ką jis padarė iš to ką prašėme ir ko nepadarė.
  \item Skaičiavimas – numato antrinių duomenų skaičiavimą. Pvz
    atsiperkamumas prekės.
  \item Organoleptinis – taikytinas neįprastoms prekėms, maisto
    kokybės vertinimas, kvapo vertinimas. Susijęs su žmogaus jutimo
    organais. Būdas išmatuoti tai kas neturi matavimo vienetų. Pavyzdžiui,
    kvepalų įvertinimas.
  \item Apklausų – pavyzdžiui, kai dėstytojas paklausia studentų, kaip
    jie vertina jo darbą.
  \item Ekspertizės – FIXME: visiškas, pilnavertis produkto vertinimas.
\end{itemize}

\section{Veiklos koregavimo alternatyvos}

\begin{itemize}
  \item Nieko nedaryti.
  \item Pašalinti nukrypimus – nustačius klaidas išsiaiškinti jų priežastis
    ir jas pašalinti. Eliminuoti klaidas darbo procese. Tokiu būdu
    organizacija tobulėja.
  \item Keisti standartą – pavyzdžiui, jei gamybos standartas vietoj
    3 metrų skersmens rutulio pagamina 2 metrų skersmens rutulį, tai
    standartas yra pakeičiamas: nurodoma, kad reikia gaminti 2 metrų
    skersmens rutulius. Standartas dažniausiai lengvinamas, todėl
    produktas ir organizacija ne tobulėja, o tiesiog pasilengvina sau
    gyvenimą.
\end{itemize}

\section{Efektyvios kontrolės požymiai}

\begin{itemize}
  \item Strateginis kontrolės pobūdis – žiūri į ilgalaikę
    kokybės perspektyvą. Standartų palaikymo perspektyvą. Orientuota
    į rezultatą, kuris bus ateityje. Kontrolės procesą reikia
    sutelkti į tobulėjimą.
  \item Orientacija į rezultatus – kontrolė neturi būti labai
    smulkmeniška, ji turi akcentuoti, koks bus pasiektas rezultatas.
    Atsižvelgti reikia ir į patį procesą ir jį kontroliuoti, nors
    rezultatas svarbiausias. Pavyzdžiui, neturėtų būti
    kontroliuojama, kuria ranka dirbti.
  \item Atitiktis veiklai – kontroliuoti tai, kas atneša naudą.
    Reikia nusistatyti prioritetus ką kontroliuoti.
  \item Kontrolės savalaikiškumas – siekti kuo labiau sumažinti tarpą
    tarp rezultato gavimo ir jo kokybės patikrinimo. Kuo greičiau
    identifikuoti kaip gerai padarytas produktas.
  \item Kontrolės lankstumas – nustatoma, iki kokio lygio galima toleruoti
    standarto pažeidimus. Pavyzdžiui, projekto galutinis terminas yra už
    savaitės, tačiau galima paklaida $\pm$ kelios dienos.
  \item Kontrolės ekonomiškumas – kontrolė turi atnešti naudą ir
    išlaidos skirtos vykdyti turi būti mažesnės, nei iš jos gaunama
    nauda.
\end{itemize}
