\chapter{Galia ir valdžia valdymo sistemoje}

\section{Galia}

\begin{defn}[Galia]
  Gebėjimas daryti įtaką kitiems, tai yra keisti individų ir grupių
  elgesį ar pažiūras.
\end{defn}

Vadybos požiūriu, galia yra sugebėjimas valdyti ar panaudoti jėgą.

\subsection{Galios šaltiniai}

Daug kur dar vadinami valdžios šaltiniais.

\begin{description}
  \item[Baudimo galia] – įtaką darančio asmens galimybė nubausti
    (priešingybė atsilyginimui). Pavyzdžiui, atleisti iš darbo,
    ar pažeminti pareigose. Su baudimo galia yra susijusi prievartinė
    valdžia: ją turi $A$ prieš $B$, jei $B$ bijo $A$.
  \item[Atsilyginimo galia] – vienas žmogus (darantis įtaką) turi
    galimybę atsilyginti kitam (kuriam daroma įtaka) už įvairius
    įvykdytus nurodymus. Su šia galia susijusi atpildu grįsta
    valdžia: ją turi $A$ prieš $B$, jei $A$ gali suteikti $B$
    tai, ko jis nori. Pavyzdžiui, paaukštinti pareigose, paskirti
    įdomesnę darbo užduotį ir panašiai.
  \item[Formalioji galia] (formalioji valdžia) – kai pavaldinys
    pripažįsta, kad įtaką darantis asmuo turi teisę daryti įtaką
    neperžengdamas tam tikrų ribų. Kitaip dar vadinama įstatiminė
    valdžia.
  \item[Ekspertinė galia] – darantysis įtaką žmogus turi tam tikrų
    specifinių žinių ar įgūdžių, kurių veikiamasis neturi.
  \item[Patrauklumo galia] – veikiamasis nori būti panašus ar
    susitapatinti su tuo, kuris jam daro įtaką. Dar vadinama
    etalonine.
\end{description}

Charizmatinis galios šaltinis: Max Weber (1864 – 1920) „charizmą“ perėmė
iš Robert Zom, ankstyvosios krikščionybės bendruomenių tyrinėtojo.
„Charizma“ vadinama asmenybės savybė pripažįstama neįprasta.
Dėl jos asmenybė vertinama kaip apdovanota antgamtiškomis, antžmogio
arba mažiausiai specifinėmis jėgomis ir savybėmis, neprieinamomis
kitiems. Ši savybė sąlygota magijos ir buvo būdinga aiškiaregiams,
išminčiams-išgydytojams, įstatymų aiškintojams, medžioklės vedliams,
karo herojams. Šiuo atveju visiškai nesvarbu „objektyvumas“ jokiu
aspektu. Svarbu tik kaip ją faktiškai įvertina „pasekėjai“.

\section{Valdžia}

\begin{defn}[Valdžia]
  Galios forma, dažnai naudojama plačiau nusakyti žmonių sugebėjimą
  valdyti.
\end{defn}

\begin{defn}[Formali valdžia]
  Galios, kuri siejama su organizacijos struktūra ir valdymu, tipas.
  Ji remiasi pripažinimu, teisėtumu daryti įtaką.
\end{defn}

Vadybos požiūriu valdžia yra teisė valdyti ir eikvoti resursus.

Formalios valdžios pagrindas:
\begin{description}
  \item[klasikinis požiūris] (formalios valdžios teorija) – valdžia
    atsiranda labai aukštame visuomenės lygyje ir vadovaujantis
    įstatymais perduodama žemyn iš vieno lygio į kitą;
  \item[pritarimo požiūris] – valdžios šaltinis slypi ne tame, kas
    daro įtaką, o tame, kam yra daroma įtaka.
\end{description}

Valdžia nuo galios skiriasi tuo, kad valdžia yra formali galios suteikimo
išraiška. Du požiūriai: iš viršaus į apačia ir iš apačios į viršų.

\section{Valdžios tipai}

\begin{description}
  \item[Linijinė valdžia] – valdžia vadovų, tiesiogiai atsakingų per
    visą komandų grandinę už tai, kad organizacijos tikslai būtų pasiekti.
    (Valdžia iš viršaus į apačią.)
  \item[Patariamoji valdžia] – valdžia, teikianti linijiniams vadovams
    patarimus ir paslaugas. (Prezidento, ministro padėjėjai.)
  \item[Funkcinė valdžia] – patariamosios valdymo grandies narių valdžia,
    įgalinanti kontroliuoti linijinių grandžių veiklą.
\end{description}

Patariamoji – ta, kuri yra pavaldi tik konkrečiam savo vadovui. Patarėjai
niekada neturi pavaldinių.

Funkcinė panaši į matricinę struktūrą.

\section{Valdžia grindžiami principai}

Valdžios balanso (pusiausvyros) principas:
\begin{itemize}
  \item vadovas pavaldinius veikia per darbo užmokestį, užduotis, karjerą,
    įgaliojimus, socialinių poreikių tenkinimą ir panašiai;
  \item atskirose situacijose vadovas priklauso nuo pavaldinių: per
    sprendimų priėmimui reikalingą informaciją, neformalius kontaktus
    su kitų padalinių darbuotojais, kurių pagalba reikalinga vadovui,
    pavaldinių įtaka kolegoms, pagaliau pavaldinių sugebėjimas vykdyti
    darbus;
    \begin{exmp}
      Tyrimai rodo, kad net kalėjimo prižiūrėtojai priklauso nuo kalinių.
      Prižiūrėtojai turi teisę už nepaklusnumą rašyti raportus, bet
      per didelis raportų skaičius gali sudaryti vaizdą, kad prižiūrėtojai
      negali įvesti tvarkos. Taigi, ieškoma kompromiso – paklusnumo
      už pažeidimų nematymą.
    \end{exmp}
  \item nenaudoti beatodairiškai visos savo valdžios: pavaldinys,
    turėdamas valdžią, gali pasipriešinti; tai būtų protingas valdžios
    balansas; be pavaldinių valdžią turi ir kolegos (ypač per informaciją).
\end{itemize}
Tai, kad valdžia yra galimybė primesti savo valią, nepriklausomai nuo
kito žmogaus jausmų, norų ir gabumų yra tik iliuzija: absoliuti valdžia
neegzistuoja.


Valdžios apimties principas (kontrolės apimtis) tai pavaldinių
skaičius, kuriems vadovas gali efektyviai vadovauti.

Prieš II pasaulinį karą, remiantis V. A. Graičiūno darbais,
nustatyta, kad vadovas negali tiesiogiai valdyti daugiau nei 5-6
pavaldinių, atliekančių bendrą darbą. Pastaruoju metu teigiama, kad
žmonių, kurie tiesiogiai atsakingi vienam asmeniui, skaičius
priklauso nuo darbų sudėtingumo, įvairumo, fizinio artimumo, nuo
darbuotojų savybių bei vadovo sugebėjimų. Suprantama, kad kuo
sudėtingesnis darbas, tuo mažesnė valdymo apimtis.

\section{Įgaliojimai ir delegavimas}

\begin{defn}[Įgaliojimai]
  Teisė atlikti tam tikrus veiksmus, priimti tam tikrus sprendimus.

  Įgaliojimai suteikiami per delegavimą, kartu sukuriant ir
  atsakomybę prieš suteikiantįjį įgaliojimus bei prisiimant atsakomybę
  prieš tą, iš kurio buvo gauti įgaliojimai. Tai yra, jei $A$
  delegavo įgaliojimus $B$, o $B$ – $C$, tai $C$ yra atsakingas už
  savo veiksmus prieš $B$, bet $B$ tuo pačiu atsakingas už $C$ darbus
  prieš $A$.

  Pastaba: pati viena atsakomybė nejuda nuo vieno individo pas kitą
  (pati viena atsakomybė negali būti deleguojama).

  \begin{description}
    \item[Linijiniai įgaliojimai] – tie, kuriuos vadovas tiesiogiai
      perduoda savo pavaldiniams per pavaldumo liniją iš viršaus į
      apačią.
    \item[Aparatiniai (štabiniai) įgaliojimai]:
      \begin{description}
        \item[konsultaciniai] – laikini ar pastovūs konsultantai
          specifinėms problemoms;
        \item[aptarnavimo] – personalo, ryšių su visuomene, juridiniai,
          ekologiniai ir panašiai klausimai;
        \item[asmeninis aparatas] – padėjėjas, biuras.
      \end{description}
  \end{description}
\end{defn}

Įgaliojimai: teisė padaryti kažką. Linijinis – vadovo funkcijų 
pasidalijimas su pavaldiniais. Aparatinis – kai vadovas suteikia
galimybę jam padėti.

\begin{defn}[Delegavimas]
  Formalios valdžios ir atsakomybės suteikimas pavaldiniui konkrečiai
  veiklai ar veikloms atlikti.
\end{defn}

Vadovų nenoro deleguoti įgaliojimus priežastys:
\begin{itemize}
  \item „aš tai padarysiu geriau“;
  \item nesugebėjimas vadovauti;
  \item nepasitikėjimas pavaldiniais;
  \item aiškios kontrolės sistemos nebuvimas;
  \item valdžios troškimas.
\end{itemize}

Pavaldinių priešinimosi įgaliojimų delegavimo procesui priežastys:
\begin{itemize}
  \item baimė padaryti klaidą;
  \item nepasitikėjimas savo jėgomis;
  \item nėra suinteresuotumo prisiimti papildomą atsakomybę;
  \item pavaldinys jau ir taip perkrautas darbu;
  \item pavaldinys neturi pakankamai informacijos.
\end{itemize}

\section{Centralizacija ir decentralizacija}

\begin{defn}[Centralizuota organizacija]
  Tokia, kurioje daugumą sprendimų priima centrinė vadovybė arba
  aukščiausio lygio vadovai. Savo ruožtu žemesnio lygio darbuotojams
  paliekama mažiau veikimo laisvės ir autonomijos.
\end{defn}

Centralizacijos privalumai:
\begin{itemize}
  \item sprendimus priima kompetentingiausias organizacijos žmogus, nes
    asmenys, dirbantys aukščiausiame valdymo lygyje, yra kompetentingesni;
  \item ši sistema įgalina išvengti darbų dubliavimo;
  \item didesnė kontrolė;
  \item taupesnis personalo panaudojimas;
  \item lengvesnis komunikavimas.
\end{itemize}

Centralizacijos trūkumai:
\begin{itemize}
  \item asmenys, priimantys galutinį sprendimą, mažai kontaktuoja su
    žemesniu lygiu (vienas priima, kiti vykdo; lėtesnis sprendimų
    priėmimas);
  \item didelė tikimybė priimti klaidingą sprendimą;
  \item perdėta biurokratija;
  \item nelankstumas;
  \item iniciatyvos slopinimas;
  \item personalo ugdymo slopinimas.
\end{itemize}

\begin{defn}[Decentralizuota organizacija]
  Tokia, kurioje žymi dalis sprendimų priėmimo teisės ir autonomijos
  yra deleguojama žemesniems valdymo lygiams.
\end{defn}

Decentralizacijos privalumai:
\begin{itemize}
  \item kadangi kiekvienas valdymo lygis savarankiškai priima sprendimus,
    tai sprendimai priimami greičiau (nėra laiko nuostolių);
  \item sprendimus priimantys vadybininkai turi daugiau informacijos
    apie konkrečią situaciją;
  \item personalo ugdymas;
  \item didesnis jautrumas išorės aplinkai.
\end{itemize}

Decentralizacijos trūkumai:
\begin{itemize}
  \item sprendimai gali būti priimami be reikiamos koordinacijos;
  \item vadybininkai atskiruose padaliniuose gali būti įtraukti į
    grupinių interesų žaidimus, atsiranda potenciali galimybė
    ignoruoti organizacijos tikslus;
  \item trukdo unifikuoti, reglamentuoti valdymo sistemą;
  \item sunkiai kontroliuojama.
\end{itemize}
