\chapter{Kontrolė}

\section{Kas yra kontrolė?}

\begin{defn}[Kontrolė]
  Bet kurios veiklos srities tikrinimas, priežiūra, stebėjimas.
\end{defn}

\begin{defn}[Kontrolė]
  Organizacijos veiklos įvertinimo procesas, atsižvelgiant į tikslus,
  ir koregavimo veiksmų atlikimas, stengiantis išlaikyti organizaciją
  kelyje į tikslų pasiekimą.
\end{defn}

\begin{defn}[Kontrolė]
  Procesas, kuriuo siekiama užtikrinti, kad reali veikla atitiktų
  planuojamą, o faktinė situacija – norimą. (\emph{Vikipedija})
\end{defn}

\section{Kontrolės rūšys}

\begin{description}
  \item[Pirminė (parengiamoji) kontrolė] – vykdoma iki organizacijos
    faktinės veiklos pradžios. Pagrindiniai parengiamosios kontrolės
    instrumentai yra atitinkamos taisyklės, procedūros, elgsena.
    Parengiamoji kontrolė taikoma tiesiogiai vykdant darbus.
  \item[Taktinė (einamoji) kontrolė] – vyksta reguliariai, darbo
    eigoje, aptariant iškilusius klausimu ir pasiūlymus darbui
    gerinti. Ji padeda išvengti atotrūkio nuo planų ir instrukcijų.
  \item[Galutinė (baigiamoji) kontrolė] – jos metu gauti rezultatai
    yra palyginami su planuotais. Vadovybė turi galimybę geriau
    įvertinti, ar planai bus realūs.
\end{description}

\section{Kontrolės procesas}

\begin{enumerate}
  \item \label{enum:control_process_01} Veiklos atlikimo standartų
    sukūrimas.
  \item Veiklos atlikimo lygio įvertinimas.
  \item Atliktos veiklos palyginimas su standartais.
  \item Koregavimo veiksmai ir atgal į \ref{enum:control_process_01}.
\end{enumerate}

\section{Standartų rūšys}

\begin{itemize}
  \item Našumo standartai.
  \item Išlaidų standartai.
  \item Kokybės standartai.
  \item Elgsenos standartai.
  \item Laiko standartai.
\end{itemize}

\section{Kontrolės formos}

\begin{itemize}
  \item Apsilankymas darbo vietoje.
  \item Problemų aptarimas posėdžiuose, seminaruose, pasitarimuose.
  \item Kontrolė ataskaitų forma – dirbantieji pateikia ataskaitas
    apie tai, kaip jiems sekėsi dirbti. Naudojama tada, kai galima
    užrašyti ir suskaičiuoti darbo rezultatus.
  \item Savikontrolė – pavyzdžiui, organizacijos inventorizacija, kai
    tikrinami organizacijos įrenginiai, daiktai. Kitas pavyzdys:
    dėstytojas duoda anketas studentams, kad išreikštų nuomonę apie
    jo darbą.
  \item Išorinė kontrolė – auditas (ateina žmonės iš išorės ir tikrina
    organizacijos darbą).
\end{itemize}

\section{Gautų darbo rezultatų lyginimo su nustatytomis normomis būdai}

\begin{itemize}
  \item Matavimas – FIXME: pvz. kubo kraštus matuojame cm.
  \item Registravimas – užregistruojama darbo eiga ir palyginama su
    standartais.
  \item Skaičiavimas – TODO.
  \item Organoleptinis – neįprastoms prekėms, maisto kokybės vertinimas,
    kvapo vertinimas. TODO: Kas tai per velnias apskritai?
  \item Apklausų – pavyzdžiui, kai dėstytojas paklausia studentų, kaip
    jie vertina jo darbą.
  \item Ekspertizės – FIXME: visiškas, pilnavertis produkto vertinimas.
\end{itemize}

\section{Veiklos koregavimo alternatyvos}

\begin{itemize}
  \item Nieko nedaryti.
  \item Pašalinti nukrypimus – nustačius klaidas išsiaiškinti jų priežastis
    ir jas pašalinti.
  \item Keisti standartą – pavyzdžiui, jei gamybos standartas vietoj
    3 metrų skersmens rutulio pagamina 2 metrų skersmens rutulį, tai
    standartas yra pakeičiamas: nurodoma, kad reikia gaminti 2 metrų
    skersmens rutulius.
\end{itemize}

\section{Efektyvios kontrolės požymiai}

\begin{itemize}
  \item Strateginis kontrolės pobūdis – FIXME: kai rezultatas į ateitį.
  \item Orientacija į rezultatus – FIXME: akcentuoti koks procesas turi
    būti. (Prieštara: svarbus rezultatai, ar tai kaip jie gaunami?)
  \item Atitiktis veiklai – kontroliuoti tai, kas atneša naudą.
  \item Kontrolės savalaikiškumas – siekti kuo labiau sumažinti tarpą
    tarp rezultato gavimo ir jo kokybės patikrinimo.
  \item Kontrolės lankstumas – nustatomi rėmai už kurio standartai
    gali išeiti ar negali išeiti. Pavyzdžiui, projekto galutinis terminas
    $\pm$ kelios dienos.
  \item Kontrolės ekonomiškumas – kontrolė turi atnešti naudą, ir išlaidos
    skirtos vykdyti turi būti mažesnės, nei iš jos gaunama nauda.
\end{itemize}
