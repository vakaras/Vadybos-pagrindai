\chapter{Konfliktai}

\section{Kas yra konfliktas?}

Lotyniškai \emph{conflictus} – susidūrimas.

\begin{defn}[Konfliktas]
  Priešingų, nesuderinamų interesų, pažiūrų, elgesio tendencijų
  susidūrimas, kai kito žmogaus ar grupės pozicija kuriuo nors
  klausimu yra atmetama ir laikoma kliūtimi tolesnei veiklai.
\end{defn}


\begin{defn}[Konfliktas]
  Kova už vertybes ir statusą, valdžią, išteklius. Šios kovos tikslas
  – priešininko neutralizavimas arba sunaikinimas.
\end{defn}

\begin{defn}[Konfliktas]
  Dviejų pusių priešiškumo būsena. (\emph{Wikipedia})
\end{defn}

\begin{defn}[Organizacinis konfliktas]
  Konfliktas, kurį sukelia organizacijos specifinės savybės,
  struktūros ypatybės bei jos sąveika su kitomis organizacijomis.
\end{defn}

\section{Pagrindinės konfliktų priežastys}

\begin{itemize}
  \item Ištekliai. Pvz.: materialiniai, žmogiškieji, laiko.
  \item Poreikiai. Fiziologinio, saugumo, pripažinimo (pvz. plagijavimas) ar
    kitų poreikių netenkinimas.
  \item Vertybės. Pvz.: politiniai, religiniai konfliktai.
  \item Informacija. Pavyzdžiui, dėl jos slėpimo, iškraipymo.
\end{itemize}

\section{Konfliktų tipai}

\begin{description}
  \item[Funkciniai (konstruktyvūs)] – sąlygojantys organizacijos efektyvumo
    augimą.
  \item[Disfunkciniai (destruktyvūs)] – sąlygojantys asmeninio
    pasitenkinimo, grupinio bendradarbiavimo ir organizacijos efektyvumo
    mažėjimą.
\end{description}

\section{Konfliktų lygiai}

\begin{itemize}
  \item Asmeninis (vidinis).
  \item Tarpasmeninis.
  \item Asmens ir grupės.
  \item Tarpgrupinis.
\end{itemize}

\section{Konfliktų savybės}

% \begin{tabularx}{\textwidth}[]{X | X}
%   Teigiamos & Neigiamos \\
%   \hline
%   Mažėja įtampa tarp konfliktuojančių šalių. &
%   Prarandama daug emocinių ir materialinių išteklių. \\
%   Gaunama nauja informacija apie oponentą. &
%   Blogėja psichologinis klimatas, mažėja drausmė, galimi organizacijos
%   narių atleidimai. \\
%   Kolektyvas susitelkia prieš bendrą „priešą“. &
%   Nugalėtojų grupė suvokiama, kaip priešai. \\
%   Skatinami pokyčiai. &
%   Dėl pernelyg aktyvaus įsitraukimo į konfliktą nukenčia darbas. \\
%   Išnyksta pavaldinių nuolankumo sindromas. &
%   Sudėtingas dalykinių santykių atkūrimas po konflikto. \\
% \end{tabularx}

\section{Vadovavimo konfliktams stiliai}

\begin{itemize}
  \item Prievartinis.
  \item Bendradarbiavimo stilius.
  \item Prisitaikymo stilius.
  \item Vengimas – laukimas, kol išsispręs; nesikišimas.
  \item Kompromisinis stilius.
  \item Stimuliavimo – kurstyti konstruktyvų konfliktavimą.
\end{itemize}

\section{Konflikto sprendimo strategijos}

\subsection{Win-Loose}

Viena pusė laimi, kita – pralaimi.

Konfliktai sprendžiami vadovaujantis principu „arba – arba“, nors tai
dar labiau supriešina konfliktuojančias puses. Pavyzdžiui: į stažuotę
gali važiuoti tik vienas. Čia priemonės pergalei pasiekti nėra
svarbios.

\subsection{Loose-Loose}

Pralaimi abi pusės.

Abi pusės sutinka dalinai atsisakyti trokštamų tikslų. Pavyzdžiui, vadovai
susipyksta matant pavaldiniams, tad praranda dalį pagarbos, autoriteto
jų akyse.

\subsection{Win-Win}

Abi pusės laimi.

Pagrindinis konfliktuojančių pusių uždavinys – rasti sprendimą, kuris
tenkintų abi puses.
