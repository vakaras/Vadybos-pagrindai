\chapter{Vadybos apibrėžimas}

\begin{defn}[Vadyba]
  Mokymas apie organizacijų veiklos tvarkymo dėsningumus, metodus,
  principus ir sistemas.
\end{defn}

\begin{defn}[Organizacija]
  Tam tikrai veiklai suburtas sistemiškas kolektyvas, turintis turto ir
  tvarkytojų.
\end{defn}

Pagrindinės vadybos funkcijos:
\begin{itemize}
  \item planavimas – procesas, kai iškeliami tikslai ir ieškoma
    priemonių jiems pasiekti;
  \item organizavimas – procesas, per kurį ištekliai išdėstomi
    ir sutvarkomi taip, kad sėkmingai būtų įgyvendinti planai;
  \item motyvavimas (lyderiavimas) – poveikis organizacijos dalyvių
    veiklai (ar elgsenai), ją nukreipiant, apribojant bei
    skatinant;
  \item kontroliavimas – pasiektų ir planuotų rezultatų palyginimas.
\end{itemize}

Valdymo personalas siekia:
\begin{itemize}
  \item žemų resursų sąnaudų (aukštas efektyvumas);
  \item aukšto organizacijos tikslo pasiekimo laipsnio (aukštas
  rezultatyvumas).
\end{itemize}

Valdymo lygiai:
\begin{description}
  \item[Aukščiausio lygio vadovai] – atsakingi už visos organizacijos
    valdymą. Jie nustato tikslus, vysto strategiją ir kuria organizacijos
    veiklos politiką. Pavyzdžiai: prezidentas, rektorius, strategas,
    vadovas.
  \item[Vidurinio lygio vadovai] – atsakanti už strategijos ir politikos
    įgyvendinimą, derina aukščiausios vadovybės reikalavimus su savo
    pavaldinių sugebėjimais. Pavyzdžiai: direktorius, valdytojas,
    vedėjas.
  \item[Žemiausio lygio vadovai] – atsakingi tik už vykdytojų, tai yra
    eilinių darbuotojų darbą: įvertina ir koordinuoja jų veiklą.
    Pavyzdžiai: brigadininkas, koordinatorius.
\end{description}

Gali būti vadovai-lyderiai ir vadovai-administratoriai. Jų santykį
gerai nusako angliška frazė: „Leadership is doing the right things.
Management is doing things right“.

Taip pat vadovai gali būti:
\begin{description}
  \item[funkciniai] – atsakingi ti už vieną funkcinę sritį
    (gamybą, marketingą ir panašiai);
  \item[linijiniai] – atsakingi už visas kokios nors posistemės
    (akcinės bendrovės, jos padalinio) veiklas (gamybą, marketingą,
    finansus).
\end{description}

\section{Valdymo turinys}

Valdymo turinį nusako:
\begin{itemize}
  \item funkcijos;
  \item vaidmenys;
  \item įgūdžiai.
\end{itemize}

\subsection{Pagrindinės veiklos funkcijos}

\begin{itemize}
  \item Marketingas.
  \item Finansai.
  \item Gamyba.
  \item Personalo valdymas.
  \item Kokybės valdymas.
  \item Techninių tarnybų veikla.
  \item Kita.
\end{itemize}

\subsection{Vadovo vaidmenys}

Pagal H. Mitzberg.

\begin{itemize}
  \item Formalus autoritetas ir statusas.
  \item Tarpasmeniniai vaidmenys:
    \begin{enumerate}
      \item boso – simbolio, nominalaus vadovo;
      \item sąveikos, ryšių;
      \item lyderio.
    \end{enumerate}
  \item Informaciniai vaidmenys:
    \begin{enumerate}
      \item patarėjo, auklėtojo;
      \item platintojo;
      \item atstovo, kalbėtojo;
    \end{enumerate}
  \item Sprendimų priėmimo vaidmenys:
    \begin{enumerate}
      \item verslininko;
      \item tvarkos įvedimo;
      \item išteklių paskirstymo;
      \item derybininko, tarpininko.
    \end{enumerate}
\end{itemize}

\subsection{Valdymo įgūdžiai}

\begin{description}
  \item[Techniniai įgūdžiai] – sugebėjimas panaudoti žinias, procedūras
    ir metodus savo specializacijos srityje.
  \item[Tarpasmeniniai įgūdžiai] – sugebėjimas bendrauti, suprasti
    ir motyvuoti kitus.
  \item[Koncepciniai įgūdžiai] – sugebėjimas koordinuoti ir integruoti
    visus organizacijos interesus ir veiklą, suvokti organizaciją
    kaip visumą.
\end{description}

\chapter{Vadybos teorijų raida}

\section{Vadybos mokslo užuomazgos}

\begin{description}
  \item[Pirmykštė visuomenė:] kolektyvinė veikla, socialinių junginių
    susidarymas, administracinių teritorinių dalinių formavimas.
  \item[Šumerų civilizacija:] Hamurabio įstatymai ir lyderiavimo
    pavyzdys.
  \item[Senovės Romos ir Graikijos karo vadai:] tvirta hierarchinė
    struktūra.
  \item[Romos imperatorius Diokletianas:] pirmas bandymas taikyti
    valdžios decentralizaciją.
\end{description}

\section{Vadybos teorijų raida}

\begin{description}
  \item[Valdymo mokyklų požiūris:] apima mokslinio valdymo mokyklą, 
    klasikinę administracinę mokyklą, žmogiškųjų santykių mokyklą,
    vadybos mokslo arba kiekybinių metodų mokyklą.
  \item[Procesinis požiūris:] vadyba nagrinėjama kaip nenutrūkstama
    tarpusavyje susijusių vadybos funkcijų (planavimo, organizavimo,
    motyvavimo ir kontroliavimo) visuma.
  \item[Sisteminis požiūris:] sistema – tai visuma, sudaryta iš
    sąveikaujančių dalių, kurių kiekviena daro įtaką bendroms
    charakteristikoms.
  \item[Situacinis (tikimybinis, atsitiktinis) požiūris:] teigia,
    kad įvairių metodų tinkamumą apsprendžia aplinkybės. Svarbiausias
    šio požiūrio elementas – situacija, konkrečios aplinkybės,
    veikiančios organizaciją.
\end{description}

\subsection{Mokslinio valdymo mokykla (1885-1920)}

\begin{description}
  \item[Frederik Taylor] – rėmėsi 4 mokslinio valdymo principais:
    \begin{itemize}
      \item tikro vadybos mokslo sukūrimas;
      \item darbininkų parinkimas moksliniais pagrindais;
      \item glaudūs ir draugiški ryšiai tarp administracijos ir
        darbininkų.
    \end{itemize}
    Jis bandė nustatyti tiesioginį ryšį tarp darbo užmokesčio ir išdirbio
    normų.
  \item[Frank ir Lilian Gilbreth] – sukūrė ciklografinį judesių tyrimo
    metodą. Siekė parodyti darbininkams, kaip padidinti darbo našumą
    ne dirbant greičiau, o geriau. Kaip pakeisti naudojamus metodus
    bei paaiškinti, kodėl tai būtina padaryti.
  \item[H. L. Gantt] – parengė darbų planavimo metodus, kurių pagrindas
    yra grafinis laiko paskirstymo vaizdavimas (Gantt diagramos).
\end{description}

\subsection{Klasikinės organizacijos (administravimo) teorijos mokykla
(1920-1950)}

\begin{description}
  \item[Henry Fayol] – į organizaciją žvelgė, kaip į visumą, bandė 
    suformuoti universalų požiūrį į organizacijų administravimą,
    suformulavo penkias administravimo funkcijas. Iškėlė 14 administravimo
    principų:
    \begin{enumerate}
      \item darbo pasidalijimas;
      \item valdžia – atsakomybė;
      \item drausmė;
      \item nurodymų vienybė;
      \item vadovavimo vienybė;
      \item asmeninių interesų pajungimas bendriems;
      \item personalo apdovanojimas;
      \item centralizacija;
      \item hierarchija;
      \item tvarka;
      \item teisingumas;
      \item personalo pastovumas;
      \item iniciatyva;
      \item personalo vienybė.
    \end{enumerate}
  \item[Max Webber] – biurokratinio valdymo teorija. Visa veikla turi
    būti reglamentuota ir bet koks nukrypimas yra žala organizacijai.
    Iškėlė valdymo principus:
    \begin{itemize}
      \item \strong{hierarchija} su aiškiai apibrėžtais valdymo lygiais;
      \item \strong{beasmenis} pareigų atlikimas;
      \item visi priimti sprendimai turi būti griežtai \strong{fiksuojami};
      \item valdžios padalinimas tarp \strong{pareigybių}, o ne asmenų.
    \end{itemize}
\end{description}

\subsection{Bihevioristinė (elgesio ir žmogiškųjų santykių) mokykla
(1930-1950 ir dabar)}

\begin{itemize}
  \item Pagrindinė idėja: organizacija yra žmonės.
  \item Išskiria socialinių veiksnių įtaką gamybos valdymui ir suranda
    naujų būdų darbo našumui didinti:
    \begin{itemize}
      \item grupiniai sprendimai;
      \item darbo humanizavimas;
      \item žmonių elgesio psichologiniai motyvai gamybos procese;
      \item grupinės normos;
      \item neformalios organizacijos ir neformalūs lyderiai.
    \end{itemize}
  \item \strong{F. Mayo - Hawthorne} eksperimentai: suvokimas, kad
    darbuotojais yra rūpinamasi, skatina stipriau nei fizinis darbo
    sąlygų pakeitimas.

    \begin{note}
      Buvo atliktas toks eksperimentas. Organizacija turėjo dvi vienodas
      gamyklas. Vienos darbuotojams pasakė, kad jie labai rūpi valdžiai
      ir kad ji jiems todėl įrengs patį moderniausią apšvietimą, bet
      realiai nieko nepakeitė. Kitos darbuotojams nieko nepasakė ir
      įrengė tą apšvietimą. Rezultatas: tie, kuriems pasakė, dirbo
      žymiai geriau, nei tie, kuriems pakeitė.
    \end{note}
\end{itemize}

\subsection{Kiekybinių metodų (vadybos mokslo) mokykla (1950-dabar)}

\begin{itemize}
  \item Visus valdymo procesus bandoma matematizuoti ir formalizuoti,
    šiam tikslui plačiai panaudojamas logikos aparatas, žaidimų
    ir eilių teorijos, statistikos ir prognozavimo metodai.
  \item Matematikos modelių, skaičiavimo technikos bei informacinių
    valdymo sistemų taikymas valdymo situacijoms, problemoms spręsti
    ir vykdymui kontroliuoti.
\end{itemize}

\subsection{Sisteminis požiūris (nuo 1970 vidurio)}

\begin{itemize}
  \item Organizacija yra suvokiama kaip sistema, kurios vieni komponentai
    daro poveikį kitiems.
  \item Organizacija sudaryta iš posistemių (padalinių, veiklų), be to,
    pati yra didesnės sistemos (rinkos, verslo šakos) dalis.
  \item Išryškinta sinergijos efekto svarba – tarpusavyje 
    bendradarbiaujantys ir sąveikaujantys padaliniai dirba rezultatyviau
    negu jie dirbtų atskirai.
\end{itemize}

\subsection{Situacinis (atsitiktinumų) požiūris (XX a. 9-asis dešimtmetis)}

\begin{itemize}
  \item Teigia, kad nėra universalaus valdymo, reikia aprašinėti ir
    nagrinėti konkrečias situacijas, surasti teisingą jų sprendimą,
    tinkamą ir kitoms panašioms situacijoms.
  \item Vadovas turi sugebėti atsakyti į klausimą „Kuris metodas čia
    tiktų?“
\end{itemize}

\subsection{Šiuolaikinės vadybos tendencijos}

\begin{itemize}
  \item Nuolat besimokanti organizacija.
  \item Bandoma paneigti istoriškai susiformavusią valdymo piramidę.
    Ji verčiama ant šono. Horizontalėje koordinuojami ryšiai.
  \item Darnioji plėtra (Vikipedija: sąvoka, apimanti visumą metodų, kuriais
    siekiama užtikrinti vystymąsi, tenkinantį žmonių gerovę dabartyje,
    nesumažinant žmonių gerovės galimybių ateityje).
  \item Žinių vadyba (Vikipedija: tai organizacijų praktikos arba metodai,
    kurių pagalba atpažįstamos, organizuojamos, kuriamos, platinamos žinios
    jų atkartojimui, mokymui ir įsisąmoninimui).
  \item Kokybės vadyba.
\end{itemize}

\chapter{Organizacija ir jos aplinka}

\begin{defn}[Organizacija]
  Grupė žmonių, kurių veikla sąmoningai koordinuojama siekiant bendrų
  tikslų.

  \begin{itemize}
    \item Ją sudaro bent du žmonės.
    \item Ji turi bent vieną visiems jos nariams svarbų tikslą.
    \item Organizacijos nariai sąmoningai drauge dirba bendram tikslui
      pasiekti.
  \end{itemize}
\end{defn}

\section{Formalios ir neformalios organizacijos}

\xtable
{
  w [ 3 | 5 | 5 ]
  a [ p | p | p ]
  h [ | Formali organizacija | Neformali organizacija ]
  %
  e [
    Tikslai | pelnas, rinkos dalis, įvaizdis |
    narių poreikių patenkinimas
    ]
  e [
    Santykiai | oficialūs | neoficialūs
    ]
  e [
    Valdžios šaltiniai | delegavimas | lyderiavimas
    ]
  e [
    Elgsenos reikalavimai | taisyklės | normos
    ]
  e [
    Tarpusavio sąveikos pagrindas | pareigos | asmenybės savybės, statusas
    ]
}

\section{Organizacijos charakteristikos}

\begin{itemize}
  \item Tikslai.
  \item Ištekliai.
  \item Darbo pasidalijimas.
  \item Organizavimas.
  \item Valdymo būtinumas.
  \item Vidinių sąlygų ir išorinės aplinkos įtaka.
\end{itemize}

\section{Organizacijos vidinė aplinka}

\begin{description}
  \item[Tikslas] – pagrindinė organizacijos siekiamybė, veiklos orientyras
    ir vertinimo kriterijus, kuriam pasiekti nukreipta visa įmonės veikla
    ir taikomos priemonės.
  \item[Struktūra] – atskirų pareigų, grupių ir padalinių tarpusavio
    ryšys, priklausomybė, pavaldumas.
  \item[Užduotys] – organizacijos tikslų konkretizavimas, darbai, kurie
    turi būti atlikti tam tikru laiku tam tikroje vietoje.
  \item[Technologija] – užduočių įvykdymui naudojamas įrenginių, įrankių
    ir techninių žinių derinys.
  \item[Žmonės] – svarbiausias vidinės aplinkos elementas. Jie kuria
    organizaciją, valdo ją ir vykdo jos tikslus.
\end{description}

\section{Tiesioginio poveikio aplinkos veiksniai}

\begin{itemize}
  \item Klientai.
  \item Tiekėjai.
  \item Vyriausybė.
  \item Specialiųjų interesų grupės.
  \item Žiniasklaida.
  \item Profesinės sąjungos.
  \item Finansų institucijos.
  \item Konkurentai.
\end{itemize}

\section{Netiesioginio poveikio aplinkos veiksniai}

\begin{itemize}
  \item Politiniai-teisiniai.
  \item Ekonominiai.
  \item Socialiniai.
  \item Technologiniai.
\end{itemize}

\section{Organizcijos išorinės aplinkos charakteristikos}

\begin{defn}[Išorinė aplinka]
  Visi už organizacijos ribų esantys elementai, kurie yra svarbūs
  jos veiklai. Įeina tiesioginio ir netiesioginio poveikio
  elementai. Elementai skirstomi į:
  \begin{description}
    \item[ištekliai (įėjimai)] – aplinkos ištekliai, tokie kaip
      žaliavos ir darbo jėga, galintys patekti į bet kurią
      organizacinę struktūrą;
    \item[rezultatai (išėjimai)] – transformuoti ištekliai, grąžinti
      į išorinę aplinką, kaip produktai ir paslaugos.
    \item[tiesioginio poveikio elementai] – aplinkos elementai,
      darantys tiesioginę įtaką organizacijos veiklai:
      \begin{description}
        \item[išoriniai] – profesinės sąjungos, tiekėjai, konkurentai,
          vartotojai, specialiųjų interesų grupės ir vyriausybės
          tarnybos;
        \item[vidiniai] – darbuotojai, akcininkai ir valdymo organai.
      \end{description}
  \end{description}
\end{defn}

\begin{description}
  \item[Judrumas] – aplinkos pokyčių greitis.
  \item[Neapibrėžtumas] – organizacijos informacijos apie konkretų
    veiksnį apimtis ir patikimumas.
  \item[Sudėtingumas] – veiksnių, į kuriuos turi reaguoti organizacija,
    skaičius. Kuo daugiau veiksnių daro įtaką, tuo aplinka, kurioje
    ji veikia yra sudėtingesnė.
  \item[Aplinkos veiksnių priklausomybė] – vieno veiksnio pasikeitimas
    gali lemti kitų veiksnių pokyčius.
\end{description}

\section{Organizacijos elgesys aplinkos atžvilgiu}

\begin{description}
  \item[Adaptavimasis] – prisitaikymas prie aplinkos savybių ir sąlygų.
  \item[Integravimasis] – kompleksinis jungimasis su aplinka.
  \item[Priešinimasis] – nesutikimas su aplinkos diktuojamomis sąlygomis.
  \item[Pajungimas] – aplinkos įsisavinimas ir pritaikymas savo poreikiams.
\end{description}

\chapter{Organizacijų socialinė atsakomybė}

\begin{defn}[Socialinė atsakomybė]
  Suinteresuotų šalių poreikių tenkinimas, išeinantis už organizacijos
  ribų ir besiremiantis jos etinėmis paskatomis prisiimti atsakomybę
  už savo vykdomos veiklos poveikį aplinkai, visuomenei, darbuotojams.
\end{defn}

\begin{defn}[Socialinė atsakomybė]
  Sąmoningai formuojamų ekonominių, politinių, dorovinių santykių
  tarp organizacijos ir visuomenės, įvairių jos struktūrų forma;
  pasirengimas atsakyti už savo poelgius ir veiksmus.
\end{defn}

\section{Socialinės atsakomybės dimensijos}

\begin{description}
  \item[Filantropinė atsakomybė] – gerinti bendruomenės gyvenimą, siekti
    visuomenės gerovės.
  \item[Etinė atsakomybė] – remtis sąžiningumu, teisingumu, dora, vengti
    žalos aplinkai.
  \item[Juridinė atsakomybė] – laikytis įstatymų.
  \item[Ekonominė atsakomybė] – siekti pelno.
\end{description}

\section{Socialinės atsakomybės kryptys}

\begin{itemize}
  \item Atsakomybė darbuotojams.
  \item Atsakomybė ekonominei aplinkai.
  \item Atsakomybė natūraliai aplinkai.
  \item Atsakomybė visuomenei.
\end{itemize}

\section{Organizacijos reakcija į socialines problemas}

Pagal reagavimo momentą.

\begin{description}
  \item[Besipriešinanti] – kompanija tik atsiliepia į socialinę problemą
    po to, kai ji meta iššūkį kompanijos tikslams.
  \item[Gynybinė] – kompanija veikia, kad apsigintų nuo jai metamo
    iššūkio.
  \item[Prisitaikomoji] – kompanija veikia pagal vyriausybės reikalavimus
    ir visuomenės nuomonę.
  \item[Numatanti] – kompanija iš anksto numato poreikius, kurie dar nėra
    išreikšti.
\end{description}

\section{Socialiai atsakingas verslas reiškiasi}

Per:
\begin{itemize}
  \item labdarą;
  \item socialini̇ų programų vystymą;
  \item papildomas informavimo ir švietimo priemones vartotojams.
\end{itemize}

\section{Argumentai už socialinę atsakomybę}

\begin{itemize}
  \item Socialiniai veiksmai gali būti pelningi.
  \item Ilgalaikės perspektyvos – didėjanti visuomenės gerovė duoda
    pagrindą vartojimo augimui.
  \item Pagerina organizacijos įvaizdį.
  \item Pagerina verslo sistemos gyvybingumą. Verslas egzistuoja,
    kadangi jis duoda visuomenei naudą. Visuomenė gali atsiimti savo
    dalį. Tai yra atsakomybės geležinis įstatymas.
  \item Tai pagerins ilgą laiką akcijų kainą, kadangi birža labiau
    apsaugos firmą nuo rizikos ir viešų visuomenės atakų ir todėl
    suteiks jai galimybę gauti aukštesnę kainą, proporcingą indėliui.
\end{itemize}

\section{Argumentai prieš socialinę atsakomybę}

\begin{itemize}
  \item Socialinės atsakomybės kaina yra pernelyg didelė ir dar smarkiai
    didės.
  \item Tai gali susilpninti mokėjimų balansą, kadangi prekių kainos kils,
    tam, kad būtų apmokėtos socialinės programos.
  \item Tai sumenkins verslo pirmaeilius tikslus.
  \item Visuomenės „gelbėtojo“ vaidmenį turi atlikti vyriausybė.
\end{itemize}

\chapter{Organizacijos kultūra}

\begin{defn}[Organizacijos kultūra]
  Vertybių sistema, suprantama ir priimtina visiems organizacijos
  nariams, leidžianti organizacijai kryptingai veikti bei palaikoma
  organizacijos istorijos, tradicijų, ceremonijų ir t.t. Taip pat
  padedanti išsiskirti iš kitų organizacijų.
\end{defn}

\section{Kuo svarbi organizacijos kultūra?}

\begin{itemize}
  \item Suteikia organizacijai unikalumo, išskiria organizaciją iš kitų.
  \item Darbuotojų elgseną daro nuoseklesnę, nes pateikia standartus,
    kaip reikėtų elgtis organizacijoje.
  \item Stiprina bendrumo jausmą, nes nariai puoselėja panašias vertybes.
  \item Priimtina kultūra motyvuoja darbuotojus, jiems mieliau dirbti
    organizacijoje.
\end{itemize}

\section{Organizacinės kultūros charakteristikos}

\begin{description}
  \item[Asmeninė iniciatyva] – darbuotojų savarankiškumo ir atsakomybės
    laipsnis.
  \item[Rizikos toleravimas] – laipsnis, kuriuo darbuotojai skatinami
    rizikuoti.
  \item[Veiklos kryptis] – laipsnis, kiek aiškiai organizacija nurodo
    veiklos tikslus ir lūkesčius, darbuotojų suvokimas, ko iš jų tikimasi.
  \item[Integracija] – laipsnis, kuriuo darbuotojai skatinami veikti
    koordinuotai, kartu. Parodo, ar darbuotojų bendravimo santykiai
    yra draugiški, ar oficialūs.
  \item[Vadovybės parama] – laipsnis, kuriuo vadovai suteikia aiškius
    nurodymus ir paramą savo pavaldiniams.
  \item[Kontrolė] – tiesioginės priežiūros laipsnis darbuotojų elgesiui
    kontroliuoti.
  \item[Identiškumas] – laipsnis, kuriuo darbuotojai tapatina save su
    organizacija, kaip visuma.
  \item[Atlyginimo visuma] – laipsnis, kuriuo atlyginimų sistema remiasi
    darbuotojų veikla ir pasiekimais.
  \item[Konfliktų toleravimas] – laipsnis, kuriuo darbuotojai yra skatinami
    atvirai iškelti konfliktus ir juos diskutuoti.
  \item[Komunikacijos] – kiek organizacijoje yra neformalios komunikacijos.
\end{description}

\section{Organizacijos kultūrą formuoja}

\begin{itemize}
  \item Pačios organizacijos istorija, organizacijos įkūrėjai.
  \item Atrankos procesas.
  \item Vadovavimo stilius.
  \item Socializacijos procesas.
  \item Nacionalinė kultūra, organizacijos klientai bei šalies visuomenės
    normos.
\end{itemize}

\section{Silpnos ir stiprios kultūros požymiai}

\xtable
{
  w [ 1 | 1 ]
  a [ p | p ]
  h [ Stipri | Silpna ]
  %
  e [ organizacijos pagrindinės vertybės plačiai pripažįstamos
    | nėra aiškių vertybių bei įsitikinimų, neaišku, kaip siekti
      sėkmės plėtojant savo verslą ]
  e [ asmeniniai darbuotojų įsitikinimai ir vertybės artimos
      organizacijos vertybėms
    | daugybė įsitikinimų, nesutariama, kurie iš jų svarbiausi ]
  e [ sukuria lojalumą, atsidavimą organizacijai, mažindama
      darbuotojų kaitą
    | padaliniai turi iš esmės skirtingus įsitikinimus ]
  e [ sunkiau prisitaikyti prie pokyčių
    | ritualai dezorganizuoti ]
}

\section{Organizacijos kultūrų tipai pagal Sonnefeld}

Beisbolo komanda:
\begin{itemize}
  \item naujovių diegimas, susietas su rizika;
  \item talento įvertinimas ir vystymas;
  \item plati veikimo laisvė;
  \item svarbūs darbo rezultatai;
  \item didelis finansinis atlyginimas;
  \item individualus pripažinimas;
  \item darbuotoją „perka“;
  \item pavyzdžiai: investiciniai bankai, televizijos, programuotojai,
    pramogų industrija.
\end{itemize}

Klubas:
\begin{itemize}
  \item pagarba pagyvenusiems darbuotojams;
  \item svarbiausi veiksniai: lojalumas, patirtis, amžius;
  \item „bendra visuotinė“ žingsnis po žingsnio progresyvi karjera;
  \item darbuotoją „užsiaugina“;
  \item pavyzdžiai: oro linijų bendrovės, bankai, valstybės institucijos.
\end{itemize}

Tvirtovė:
\begin{itemize}
  \item nėra siūlomas saugumas, veiklos pastovumas ar pusiausvyra;
  \item pasitaikius progai siūlo pakeisti nuomonę;
  \item pavyzdžiai: viešbučiai, mažmeninė prekyba, leidyba…
\end{itemize}

Akademija:
\begin{itemize}
  \item pabrėžiamas sisteminis karjeros vystymasis;
  \item nuolatinis tobulėjimas;
  \item specializuotas darbas;
  \item pavyzdžiai: elektronikos, mašinų gamintojai, farmacija.
\end{itemize}

\section{Organizacijos kultūrų tipai pagal Steinman ir Schreyogg}

\begin{description}
  \item[Paranojinė] – darbuotojų santykiuose vyrauja nepasitikėjimas ir
    baimė, bendradarbiai ieško vieni kitų klaidų. Veiksmuose įžvelgiamas
    noras pakenkti. Bet koks veiklos sutrikimas sukelia reakciją.
  \item[Depresinė] – vyrauja pesimistinės nuotaikos, tikima lemtimi, likimu.
    Nuolat laukiama paramos iš kitų organizacijų ar valstybės. Veikla
    rutiniška, vidinė aplinka sustabarėjusi. Darbuotojai vadovaujasi
    principu „aš nieko pakeisti negaliu, nes esu „mažas““.
  \item[Paremta prievarta] – veikla organizuojama remiantis
    administracinėmis priemonėmis. Pagrindiniai akcentai – drausmė, tvarka,
    paklusnumas. Stiprus ir gerai sureguliuotas kontrolės mechanizmas.
    Emocijos nepripažįstamos. Hierarchija – pagrindinis vadovavimo
    principas.
  \item[Šizoidinė kultūra] – aukščiausiojo lygio vadovai atitolę nuo
    pavaldinių. Pavaldiniais nepasitikima. Darbuotojų tarpusavio santykiai
    labai formalūs, „šalti“. Žemesniuose lygiuose vyksta kova dėl geresnio
    posto. Didelę įtaką turi favoritai, neformalios grupuotės. Dominuoja
    karjerizmas.
  \item[Oportunistinė kultūra] – bendradarbių santykius lemia tradicijos,
    įpročiai. Neigiama individualybė, išskirtinumas. Didžiulis dėmesys
    procedūroms, o ne reikalo esmei. Jokie pokyčiai nepageidaujami.
  \item[Įsipareigojanti kultūra] – dalykiniu ir socialiniu požiūriu atvira,
    pasiruošusi priimti kitas kultūras, keistis. Didelis dėmesys skiriamas
    interesų derinimui, akcentuojamas bendradarbiavimas.
    % http://www.mokslai.lt/referatai/referatas/personalo-vadyb-7-puslapis9.html
\end{description}

\section{Ouchi organizacijos kultūros modelis}

\xtable
{
  w [ 1 | 1 | 1 | 1 ]
  a [ p | p | p | p ]
  h [ | Japonijos firma | Z JAV firma | JAV firma ]
  %
  e [ Įsipareigojimas darbuotojams | visam gyvenimui | ilgalaikis
    | trumpalaikis ]
  e [ Darbo įvertinimas | lėtas pagal kokybę | lėtas pagal kokybę
    | greitas pagal kokybę ]
  e [ Karjera | labai plati | pakankamai plati | siaura ]
  e [ Kontrolė | numatoma ir neformali | numatoma ir neformali
    | tiksli ir formali ]
  e [ Sprendimų priėjimas | grupinis | grupinis | individualus ]
  e [ Atsakomybė | grupinė | individuali | individuali ]
  e [ Dėmesys žmonėms | visapusiškas | mažas | visapusiškas ]
}

\chapter{Valdymo kultūrų geografija}

Kodėl reikia nagrinėti kultūrinius vadybos ypatumus:
\begin{itemize}
  \item galimybė dirbti užsienio kompanijoje;
  \item bendradarbiavimas su užsienio kompanijomis (tiekėjai, klientai);
  \item rinkos plėtra į užsienį.
\end{itemize}

\section{Trys pagrindiniai valdymo kultūros arealai}

\begin{itemize}
  \item Amerikietiškasis (JAV ir Kanada, neįeina Lotynų Amerika).
  \item Japoniškasis (5 tigrai).
  \item Europietiškasis (ES šalys).
\end{itemize}

\section{Amerikietiškasis valdymas (JAV)}

\begin{itemize}
  \item Įsitikinimas, kad žmogus yra padėties šeimininkas ir kiekvienas
    pats kuria savo ateitį. Nevykėlis yra pats kaltas.
  \item Būtinas pažado įvykdymas.
  \item Tikėjimas tuo, kad žmogus turi užimti pareigas, atitinkančias
    jo nuopelnus: žmogaus statusą turi lemti jo nuopelnai.
  \item Savo nuomonės slėpimas, informacijos iškreipimas vertinamas
    kaip negarbingumas.
  \item Sprendimų decentralizavimas, akcentuojant pasitikėjimą
    darbuotojais, tikėjimą tuo, kad dauguma žmonių pozityviai reaguoja
    į jų atsakomybės augimą. Geriausias kelias vystymuisi – dalyvavimas
    sprendimų priėmime.
  \item Laikas – pinigai.
  \item Bendravimas neformalus, vengiama titulų.
\end{itemize}

\section{Amerikiečių derybų ypatumai}

FIXME: Praleista.

\chapter{Planavimo funkcija}

\begin{defn}[Planavimas]
  Vadybos funkcija, apimanti tikslų formulavimo bei jų įgyvendinimui
  tinkamos veiksmų eigos sudarymo procesą.
\end{defn}

Planavimo nauda:
\begin{itemize}
  \item planavimas disciplinuoja pavaldinius, juos sujungia į visumą
    (padeda koordinuoti veiklą);
  \item planavimas verčia galvoti į priekį, palengvina kontrolę;
  \item neplanuodami darbų, darbuotojai negali žinoti savo veiklos
    rezultatų vertės;
  \item apdraudžia nuo netikėtumų, racionaliai panaudojami ištekliai;
    planavimas neturi alternatyvos. % FIXME Paaiškinti kodėl.
\end{itemize}

Pavaldiniai nenori planuoti, nes planavimas:
\begin{itemize}
  \item padidina kontrolę – atbaido nuo kokybiško planavimo;
  \item užima daug laiko – jam paskirtos laiko sąnaudos nepateikia
    apčiuopiamo, greito rezultato;
  \item reikalauja profesionalaus pasiruošimo;
  \item dažnai kai kuriems žmonėms yra tiesiog nepriimtinas.
\end{itemize}

\section{Planavimo lygmenys}

\begin{description}
  \item[Strateginis planavimas] – ilgalaikių tikslų apibrėžimas ir jų
    vykdymo numatymas visos organizacijos mastu.
  \item[Operatyvinis planavimas] – orientuotas į trumpalaikius tikslus
    ir lokalias, svarbias tik atskiriems padaliniams ar net darbuotojams,
    priemones, kurių pagalba yra pasiekiami ilgalaikiai tikslai.
\end{description}

\section{Organizacijos planų lygiai}

\begin{description}
  \item[Strateginiai planai] – tai ilgalaikės veiklos planai, organizacijos
    veiklos kryptys.
  \item[Taktiniai planai (dar vadinami etapiniais)] – tai padalinių
    ar veiklos sričių planai.
  \item[Operatyviniai planai (dar vadinami einamaisiais)] – trumpalaikiai
    planai konkrečioms užduotims įgyvendinti, užtikrina taktinių planų
    vykdymą.
\end{description}

\section{Organizacijos planų tipai}

Vienkartiniai planai:
\begin{description}
  \item[programos] – ilgalaikės veiklos ir globalių tikslų įgyvendinimo
    planų rinkiniai (pavyzdžiui, inovacijų programos);
  \item[projektai] – atskiros programos dalys; jie yra ribotos apimties
    ir turi aiškias paskirties bei laiko direktyvas.
\end{description}

Nuolatiniai planai:
\begin{description}
  \item[organizacijos politika] – organizacijos bendrų nuostatų, nuorodų
    sistema, apibrėžianti sprendimų priėmimo ribas;
  \item[taisyklės] – nurodo konkrečius veiksmus, kuriuos būtina arba
    draudžiama atlikti tam tikroje situacijoje;
  \item[procedūros] – nuorodos, standartiniai metodai arba instrukcijų
    rinkiniai, nurodantys seką veiksmų, kuriuos reikia atlikti dažnai
    ar sistemingai.
\end{description}

\section{Planavimo etapai}

\begin{enumerate}
  \item Tikslų nustatymas.
  \item Organizacijos išorinės ir vidinės aplinkos analizė.
  \item Strategijos parinkimas.
  \item Rezultatų vertinimas.
\end{enumerate}

\section{Strateginis valdymas}

\begin{defn}[Strategija]
  Sprendimų visuma, apibrėžianti organizacijos svarbiausius ateities
  tikslus ir veiksmus bei priemones tiems tikslams pasiekti.
\end{defn}

\begin{defn}[Strateginis valdymas]
  Nuolatinis, dinaminis ir nuoseklus procesas, kuriuo remiantis
  organizacija laiku prisitaiko prie išorinės aplinkos pokyčių ir
  efektyviau panaudoja savo išteklių potencialą.
\end{defn}

\subsection{Strategijos lygiai}
  
\begin{itemize}
  \item Korporacijos lygio strategija.
  \item Verslo vieneto strategija.
  \item Funkcinio lygio strategija.
  \begin{itemize}
    \item Marketingo strategija.
    \item Finansų strategija.
    \item Inovacijų strategija.
    \item Gamybos strategija.
    \item Personalo strategija.
  \end{itemize}
\end{itemize}

\subsection{Organizacijos strategijų rūšys}

\begin{description}
  \item[Riboto augimo] \label{strategija:rusys:ribotas} – organizacija
    siekia stabilumo (nesiplėšo kovodama dėl savo rinkos dalies,
    stengiasi ją didinti nuosekliai, daro nuoseklias ir nedideles
    investicijas).
  \item[Augimo] – organizacija siekia užkariauti kuo didesnę rinkos
    dalį, daro dideles investicijas.
  \item[Mažinimo] – organizacija mažina investicijas, atsisako padalinių,
    darbuotojų ir panašiai. Šios strategijos organizacijos dažniausiai
    imasi, kai joms gresia bankrotas.
  \item[Mišri] – organizacija vienoje srityje imasi vienos strategijos
    (pavyzdžiui, augimo), o kitoje kitos (pavyzdžiui, mažinimo). Taip
    ji galėtų elgtis, kai siekia „persikelti“ iš vienos rinkos į kitą.
  \item[Palaikymo strategija] – tas pats, kas ir „riboto augimo“
    (\ref{strategija:rusys:ribotas}).
  \item[Rinkos vystymo strategija] – esami produktai parduodami naujose
    rinkose, geografinis rinkos dalies plėtimas.
  \item[Savo kiemo strategija] – turi savo klientų ratą ir jiems
    siūlo savo prekės pakaitalus. Pavyzdžiui, jogurto gamintojai
    pradėjo siūlyti sūrelius.
  \item[Rinkos užkariavimo strategija] – eina į visiškai naują rinką.
\end{description}

\subsection{Bostono konsultacinės grupės matrica}

TODO

\section{Sprendimų priėmimas}

\begin{defn}[Sprendimų priėmimas]
  Veiksmų krypties konkrečiai problemai spręsti nustatymas ir parinkimas.
  \begin{itemize}
    \item Bendriausia prasme valdymo sprendimas yra reakcija į problemas.
    \item Pagal supaprastintą versiją valdymo sprendimai yra alternatyvos
      pasirinkimas.
  \end{itemize}
\end{defn}

Valdymo sprendimų klasifikavimas.
\begin{itemize}
  \item Pagal pasikartojimų dažnumą
    \begin{description}
      \item[programuojami] – kai turime eilinę ar smulkią problemą, tai
        ją galima sutvarkyti priimant sprendimą pagal rašytą ar
        nerašytą politiką, procedūrą ar taisykles, supaprastinančias
        sprendimų priėmimą pasikartojančiose situacijose, kai
        aprobuojamos ir atmetamos alternatyvos;
      \item[neprogramuojami] – paprastai priimami tada, kai reikia
        spręsti organizacijos išteklių paskirstymo, blogėjančių
        gaminių, santykių su visuomene gerinimo problemas. Priimant
        neprogramuojamus sprendimus ypač naudingas yra racionalių
        sprendimų priėmimo modelis.
    \end{description}
  \item Pagal sprendžiamos problemos mastą: strateginiai, taktiniai,
    operatyviniai.
\end{itemize}

\subsection{Problemos diagnostika}

Paprasčiausi metodai problemos įvardijimui:
\begin{description}
  \item[nukrypimas nuo buvusios veiklos] – staigus pokytis
    vykstančiame procese dažnai rodo, kad problema pradeda vystytis
    (pavyzdžiui, mažėja pardavimai, didėja išlaidos, blogėja produkto
    kokybė);
  \item[nukrypimas nuo plano] – rezultatai nepasiekia planuotų
    tikslų;
  \item[išorinė kritika] – klientai nepatenkinti produkcija, ar jos
    pristatymo grafiku ir panašiai.
\end{description}

Verta pastebėti, kad potenciali galimybė taip pat gali būti traktuojama
kaip problema.

\subsection{Racionalių sprendimų priėmimo procesas}

\begin{enumerate}
  \item \label{enum:sprendimai:procesas:tikslai} Tikslai – išsikeliamas
    tikslas, vizija.
  \item \label{enum:sprendimai:procesas:problema} Problemos nustatymas
    – nustatoma problema, kuri iškilo siekiant tikslo.
  \item Duomenų rinkimas – nustatoma TODO.
  \item Problemos analizė, kliūčių nustatymas – TODO.
  \item Alternatyvių sprendimų formulavimas – sugalvojami bent du
    skirtingus problemos sprendimo būdus.
  \item Alternatyvų įvertinimas – kiekvienai alternatyvai prognozuojami
    tikėtini padariniai, jei ji būtų įgyvendinta.
  \item Geriausio sprendimo priėmimas.
  \item Įdiegimas – priėmę sprendimą, nustatome, kokių įrankių reikės
    ir jais pasinaudojame.
  \item Monitoringas ir įvertinimas – žiūrime, ar tikslas buvo pasiektas.
    Jei ne, tai atgal į \ref{enum:sprendimai:procesas:tikslai} arba
    \ref{enum:sprendimai:procesas:problema}. Pasitikriname, gal pasirinktas
    tikslas yra nerealus, ar neteisingai nustatėme problemą.
\end{enumerate}

\subsection{Veiksniai, darantys įtaką sprendimų priėmimui}

\begin{itemize}
  \item Vadovo vertybinė orientacija.
  \item Sprendimų priėmimo aplinka – reikia atsižvelgti į riziką,
    neapibrėžtumus (pavyzdžiui, valiutos kurso kitimą).
  \item Laiko veiksnys, laiko reikšmė – laikas dažnai sąlygoja situacijos
    pasikeitimą.
  \item Informacijos ribotumas.
\end{itemize}

\subsection{Grupiniai valdymo sprendimai}

Privalumai:
\begin{itemize}
  \item jėgų sutelkimas – daugiau kompetetingų žmonių sugalvos daugiau
    gerų sprendimų;
  \item darbo specializacija – vadovas negali būti visų galų meistras
    ir, subūręs darbo grupę iš skirtingų specialistų, gali tikėtis,
    kad problema bus išnagrinėta keliais skirtingais aspektais;
  \item priimtas sprendimas lengviau įgyvendinamas – kiekvienas žmogus,
    dalyvavęs sprendimo priėmimo procese, supranta, kodėl toks sprendimas
    buvo priimtas, ir todėl jam lengviau yra jį įgyvendinti.
\end{itemize}

Trūkumai:
\begin{itemize}
  \item laiko švaistymas – kyla daug diskusijų, kurios užsitęsia;
  \item grupiniai konfliktai;
  \item grupės lyderių neigiama įtaka (bauginimas).
\end{itemize}

\subsection{Metodai}
% papildyta iš atminties bei pagal
% http://www.mokslai.lt/referatai/referatas/imones-strateginis-valdymas-puslapis17.html
\begin{description}
  \item[Ekspertinis sprendimo priėmimas] – pagrįstas ekspertų nuomone ir
    pasiūlymais.
  \item[Grupės narių nuomonių vidurkio sprendimas] – išklausoma visų
    nuomonė ir nustatomas nuomonių vidurkis.
  \item[Autoritarinis sprendimas be grupės diskusijos] – sprendimus priima
    paskirtas vadovas (savininkas), nesikonsultuodamas su kitais kolektyvo
    nariais.
  \item[Autoritarinis sprendimas po grupės diskusijos] –  problema
    apsvarstoma susirinkime, vysta diskusija, o sprendimą priima vadovas.
  \item[Oficialaus valdžios pareigūno arba įgalioto asmens sprendimas] –
    panašus į ekspertinį sprendimo priėmimą, tačiau eksperto ieškoma ne
    organizacijos viduje, o kviečiama iš išorės.
  \item[Konsensuso metodas] – siekiama prieiti vieningo, visiems priimtino
    sprendimo.
\end{description}

\chapter{Organizavimas}

\section{Kas yra organizavimas?}

\begin{defn}[Organizavimas]
  Toks darbo, valdžios ir išteklių paskirstymo tarp organizacijos
  narių bei jų (narių) veiksmų suderinimo procesas, kuris leidžia jiems
  pasiekti organizacijos tikslus.
\end{defn}

\section{Pagrindiniai organizavimo procesai}

\begin{description}
  \item[Darbo pasidalijimas] – darbo paskirstymas tarp darbuotojų ar
    grupių, skaidant jį į užduotis.
  \item[Struktūrinių grandžių formavimas] – darbuotojų ir
    užduočių grupavimas.
  \item[Hierarchijos sukūrimas] – atsakomybės lygių ir vertikalių
    ryšių modelio sukūrimas.
  \item[Koordinavimas] – atskirų organizacijos veiklos dalių integravimas
    siekiant organizacijos tikslų.
\end{description}

\section{Organizacijos projektavimas}

\begin{defn}[Organizacijos projektavimas]
  Sprendimų priėmimo procesas, kur vadovai pasirenka organizacijos
  struktūrą, tinkamą organizacijos strategijai ir aplinkai, kurioje
  organizacijos nariai tą strategiją įgyvendina.
\end{defn}

Organizacijos projektavimo logika gali būti grindžiama:
\begin{enumerate}
  \item „iš viršaus – žemyn“ – suformavus funkcinius blokus, jie
    nuosekliai detalizuojami (sukuriamas administracinis aparatas,
    pasiskelbia vadovas, jis turi pavaldžius sau vadovus, kurie
    turi pavaldžius sau darbuotojus);
  \item „iš apačios į viršų“ – žemiausio lygio vadovai formuoja
    funkcinius padalinius, kurie laipsniškai agreguojami aukštesniuose
    valdymo lygiuose (evoliucinis metodas, kai virš tam tikros funkcijos
    pradeda augti aukštesnis biurokratinis aparatas arba aukštesnės
    vadovybės aparatas);
  \item mišrus variantas (vystosi stichiškai; pavyzdžiui, gali pradėti
    vystytis iš apačios į viršų, o paskui kažkuriuo metu pradėti vystytis
    iš viršaus į apačią);
  \item variantas pritraukiant išorės konsultantus (pats lengviausias,
    bet reikalaujantis daugiausiai pinigų: samdomas konsultantas, 
    išmanantis, kaip organizacija turi veikti).
\end{enumerate}

\section{Organizacinė struktūra}

\begin{defn}[Organizacinė struktūra]
  Veiklų pasidalijimo, grupavimo, santykių tarp darbuotojų ir santykių
  su vadovais ar padaliniais koordinavimo formali išraiška.
\end{defn}

Trys pagrindiniai komponentai:
\begin{itemize}
  \item organizacinė struktūra parodo formalius ryšius organizacijoje
    (hierarchijos lygius, kontrolės laipsnį bei centralizavimo lygį);
  \item organizacinė struktūra parodo, kaip iš atskirų darbuotojų
    suformuojamos darbo grupės, iš grupių – padaliniai, o iš jų –
    visa organizacija;
  \item organizacinė struktūra apima ir atskirų organizacijos padalinių
    komunikavimo, veiklos koordinavimo ir integravimo sistemų projektavimą.
\end{itemize}

\section{Organizacijos struktūrų tipai}

\begin{itemize}
  \item Paprasta.
  \item Funkcinė.
  \item Divizinė (produkto / rinkos / vartotojo).
  \item Matricinė.
  \item Tinklinė.
\end{itemize}

\subsection{Paprasta struktūra}

TODO: Piešinukas.

Ši struktūra dažniausiai pasitaiko naujose, mažo biudžeto ar
neambicingose įmonėse, kurios neturi tikslo augti, o tiesiog veikia
savo rėmuose.

Privalumai:
\begin{itemize}
  \item savininkas-vadovas pats kontroliuoja visą organizacijos veiklą,
    puikiai pažįsta organizaciją, tiksliai gali pasakyti, kas už ką
    atsakingas (toks žmogus yra viską kontroliuojantis, viską matantis,
    gerai žinantis visą organizaciją);
  \item operatyviai priimami sprendimai (nėra aukštų hierarchijos lygių,
    vadovas gali tiesiai prieiti prie kurio nors darbuotojo, jam
    perduoti savo sprendimą (sumanymą) ir darbuotojas tada jį vykdo –
    nereikia eiti per biurokratijos aparatus);
  \item aiški motyvacija (pinigai). („Kuo daugiau dirbat, tuo daugiau
    moku“ – į tokią organizaciją žmonės dažniausiai ateina tik dėl
    pinigų, todėl vadovui nėra prasmės kurti kitokių motyvacijų.)
\end{itemize}

Trūkumai:
\begin{itemize}
  \item vieno asmens atsakomybė – blogas sprendimas gali viską sugriauti
    (kyla problema, kad vienam žmogui dažnai neužtenka kompetencijos
    ir jis gali priimti neteisingus sprendimus, nuo kurių nukenčia
    visa organizacija);
  \item visas dėmesys tik einamiesiems reikalams, neskiriama dėmesio
    strategijai, gyvenama šia diena (vadovas pamiršta kokia yra
    organizacijos vizija, kaip ją vystyti, nes darbuotojams rūpi tik
    atlikti darbą ir eiti namo);
  \item didėjant veiklos apimtims, ženkliai didėja užimtumas darbais
    (atsiranda naujų reikalingų darbų, o darbuotojai nebegali apimti
    visų uždavinių; vadovas nebesugeba visko sužiūrėti; vadovas ir
    darbuotojai pradeda persidirbti; tada susimąstoma apie organizacijos
    skaidymą į hierarchinius lygius).
\end{itemize}

\subsection{Funkcinė struktūra}

TODO: Piešinukas.

Ši struktūra yra paplitusi Lietuvoje. Organizacija turi prezidentą arba
tarybą, akcininkus, o veikla suskirstyta pagal funkcijas. Tada,
kiekvienai funkcijai vykdyti gali būti formuojami padaliniai.

Privalumai:
\begin{itemize}
  \item efektyviai naudojami ištekliai (masto efektas);
    \begin{note}
      Organizacija, turinti daug skirtingų produktų. Pavyzdžiui,
      universitetas (rektorius – direktorius, padaliniai – prorektoriai
      (studijų prorektorius, finansų prorektorius, …), po prorektorių –
      fakultetai, bibliotekos ir t.~t.). Vienas funkcinis padalinys
      (pavyzdžiui, finansų) gali efektyviai paskirstyti lėšas fakultetams,
      studijoms ir panašiai. Vienas funkcinis padalinys mato visą
      visumą ir jam lengva atsiskaityti vadovui, bet viskas vyksta
      viename kontekste.
    \end{note}
  \item ugdomi specializuoti darbuotojų gebėjimai;
    \begin{note}
      Atrenkami žmonės, turintys labai specializuotų gebėjimų. Jie
      atlieka panašų darbą, puikiai supranta vienas kitą. Tai
      dažniausiai yra labai aukštos kvalifikacijos specialistai,
      dirbantys pagal savo specializaciją.
    \end{note}
  \item geras veiklos koordinavimas funkcijų viduje.
\end{itemize}

Trūkumai:
\begin{itemize}
  \item lėtas sprendimų priėmimas;
    \begin{exmp}
      Vienas padalinys nutarė kažką pakeisti, kiti nežino, ką jis daro,
      ir turi laukti rezultato. Tarkim, sukūrė ypatingą reklamą ir siūlo
      pakelti kainą. Tuomet padalinys, atsakingas už marketingą, laukia
      reklamos, o padalinys, atsakingas už finansus, laukia, kol galės
      įvertinti.
    \end{exmp}
  \item neaiški veiklos atsakomybė, nes organizacijos sėkmė priklauso
    nuo visų funkcinių padalinių darbo;
    \begin{exmp}
      Jei produktas žlunga, kas atsakingas už nesėkmę? Reklamos skyrius
      ar gamybos skyrius?
    \end{exmp}
  \item nėra horizontalių ryšių – darbuotojai menkai suvokia visos
    įmonės problemas. Dėl to sulėtėja problemų sprendimas.
    \begin{note}
      Kiekvienas padalinys atsakingas tik už savo sritį ir jiems
      nerūpi kitų padalinių darbas.
    \end{note}
\end{itemize}

\subsection{Divizinė struktūra}

TODO: Piešinukas.
TODO: Kokiai organizacijai tinka?

Divizinė struktūra gali būti trijų tipų:
\begin{itemize}
  \item skaidymas pagal produktus, kai organizacija išskaidoma į
    padalinius, kur kiekvienas atsakingas už tam tikrą produktą;
  \item skaidymas pagal geografiją, kai organizacija išskaidoma
    į padalinius, kur kiekvienas yra atsakingas už tam tikrą
    geografiškai apibrėžtą rinką;
  \item skaidymas pagal klientą (pavyzdžiui, padalinys, kuriantis
    vyriškų drabužių liniją, ir padalinys, kuriantis moteriškų drabužių
    liniją).
\end{itemize}

Privalumai:
\begin{itemize}
  \item didelis lankstumas ir dėmesys vartotojui, nes yra aiškus
    pasiskirstymas pagal produktą arba regioną;
    \begin{exmp}
      Pavyzdžiui, jei vienas padalinys dirba Azijoje, o kitas Amerikoje,
      tai tie padaliniai gali adaptuoti kainodarą, reklamą, gamybą
      konkrečiai kultūrai, pagal tos šalies papročius, įpročius.
    \end{exmp}
  \item aiški atsakomybė (kiekvienas padalinys atsakingas už savo
    konkretaus produkto sėkmę);
  \item decentralizuotas sprendimų priėmimo procesas.
    \begin{note}
      Padaliniams suteikiama ganėtinai didelė laisvė: jie gali
      prisitaikyti prie savo klientų, suskirstyti darbą taip, kaip
      jiems yra efektyviausia.
    \end{note}
\end{itemize}

Trūkumai:
\begin{itemize}
  \item padaliniai konkuruoja dėl išteklių;
    \begin{exmp}
      Kompiuterių ir telefonų gaminimas. „Telefonų padalinys“ nuolat
      vis kuria kažką naujo ir tam reikia daug lėšų, kurių tada negauna
      už kompiuterių gaminimą atsakingas padalinys. Problema gali būti
      ne tik dėl materialinių išteklių, bet ir dėl žmogiškųjų – kodėl
      „telefonų padalinyje“ dirba geresni darbuotojai, nei „kompiuterių
      padalinyje“.
    \end{exmp}
  \item dubliuojami ištekliai;
  \item silpna koordinacija tarp divizijų, apsunkintas bendros strategijos
    laikymasis.
    \begin{note}
      Galime turėti labai gerą strategiją, tikslą, tačiau produktai
      vystosi skirtingai ir bendra koncepcija šioje vietoje kaip ir
      pasimeta, nes jie izoliuoti vieni nuo kitų ir bendrą vaizdą
      pamatyti yra labai sunku. Šiuo atveju yra labai svarbios
      vadovo pastangos siekiant išlaikyti bendrą strategiją.
    \end{note}
\end{itemize}

\subsection{Matricinė struktūra}

TODO: Piešinukas.
TODO: Kokiai organizacijai tinka?

Turime funkcinius padalinius, tačiau dar papildomai yra vadovai,
atsakingi už konkrečius projektus. Projekto vadovas „pasiima“ iš
kiekvieno funkcinio padalinio po keletą specialistų ir suformuoja
sau darbo grupę, kuri yra atsakinga už tam tikro produkto vystymą.

Privalumai:
\begin{itemize}
  \item žmogiškieji ištekliai lanksčiai paskirstomi tarp produktų;
    \begin{note}
      Šiuo atveju galima matyti ne tik konkretaus padalinio veikimą,
      rezultatus, poreikius, bet tuo pačiu numatyti ir visos
      organizacijos. Yra įmanoma ir matyti visumą, ir koncentruotis į
      vieną produktą.
    \end{note}
  \item atsiranda galimybė susiformuoti komandai: kitoks tarpusavio 
    supratimas, bendrų tikslų prioritetas, atsiranda horizontalūs
    ryšiai;
    \begin{note}
      Komandos kartu mato visą visumą produkto rėmuose ir mato visumą
      žemyn. Jiems lengva tarpusavyje susiderinti, nebeatsiranda
      tarpusavio konkurencijos.
    \end{note}
  \item horizontalusis tarpfunkcinis koordinavimas pagal veiklas 
    (produktus) ir vertikalusis koordinavimas pagal valdymo funkcijas.
\end{itemize}

Trūkumai:
\begin{itemize}
  \item dvigubas pavaldumas;
    \begin{note}
      Reikia laikytis projekto vadovo reikalavimų, bet tuo pačiu
      padalinys turi elgtis kaip vientisas komponentas. Todėl gali
      atsirasti konfliktų, tačiau jie dažniausiai lengvai sprendžiami
      dėl horizontalių ryšių.
    \end{note}
  \item didelės struktūros įgyvendinimo išlaidos, nes reikia daugiau
    vadovų;
    \begin{note}
      Reikia samdyti vadovus atsakingus už funkcinius padalinius, taip
      pat turime samdyti ir labai aukštos kvalifikacijos žmones, kurie
      būtų atsakingi už visą projektą.
    \end{note}
  \item daug posėdžių, pasitarimų.
    \begin{note}
      Kur susirinkimai ir pasitarimai, ten atsiranda konfliktai,
      nesutarimai – laiko ir nervų tampymas.
    \end{note}
\end{itemize}

\subsection{Tinklinė struktūra}

TODO: Piešinukas.
TODO: Kokiai organizacijai tinka?

Tinklinę struktūrą dažniausiai turi didžiulės kompanijos, veikiančios
pasauliniu mastu. Jos turi pakankamai pinigų, kad galėtų samdytis
visus reikiamus žmogiškuosius išteklius.

Privalumai:
\begin{itemize}
  \item konkuravimas pasauliniu mastu;
    \begin{note}
      Padalinio organizacija egzistuoja pati sau, ji yra pakankamai
      tvirta. Tai dažniausiai yra kažkokia specializuota kompanija,
      kuri užsiima, pavyzdžiui, tik marketingu. Tokios kompanijos
      turi labai gerus specialistus ir motininė kompanija, pasirašiusi
      sutartį su tokia kompanija, gali būti užtikrinta, kad jai bus
      suteiktos aukščiausio lygio paslaugos.
    \end{note}
  \item lankstumas;
    \begin{note}
      Jei mums nepatinka tai, kaip dirba, pavyzdžiui, organizacija, kurią
      mes samdom savo marketingui, tai ją galime laisvai pakeisti
      kita organizacija.
    \end{note}
  \item mažesnis valdymo personalo poreikis (nėra valdymo hierarchijos:
    yra tik aukščiausia vadovybė, nėra vidurinių bei skyrių vadovų);
  \item įgalina sujungti milžiniškus išteklius, dažniausiai geriausiu jų
    variantu.
    \begin{exmp}
      Finansams samdydami geriausią banką galime gauti labai gerą
      rezultatą iš karto, ko negalėtume tikėtis samdydami pavienius
      darbuotojus.
    \end{exmp}
\end{itemize}

Trūkumai:
\begin{itemize}
  \item tiesioginės kontrolės nebuvimas;
    \begin{note}
      Motininė organizacija negali kontroliuoti to, kas vyksta partnerių
      organizacijose, negali reikalauti kažkokios konkrečios logikos
      ar vystymosi kelio, nes tai yra savarankiškos kompanijos.
    \end{note}
  \item žemas dirbančiųjų lojalumas kompanijai;
    \begin{note}
      Darbuotojai nejaučia, kad reikia dirbti tam, kad motininė
      organizacija klestėtų. Jiems rūpi, kad klestėtų ta organizacija,
      kurioje jie dirba tiesiogiai.
    \end{note}
  \item pavojus netekti organizacijos dalies.
    \begin{note}
      Taip pat lengvai, kaip motininė organizacija gali pakeisti
      nepatikusį partnerį, taip pat lengvai partneris gali pasitraukti
      nuo motininės organizacijos.
    \end{note}
\end{itemize}

\section{Organizacinių struktūrų raida}

\begin{description}
  \item[Piramidės] hierarchinio modelio struktūra: orientacija į
    vadovą, bet ne į veiklos produktyvumą; griežtas taisyklių,
    procedūrų laikymasis; geros sąlygos biurokratijai plėtotis.
  \item[Išlygintos piramidės] struktūra: eliminuojama vidurinioji
    valdymo grandis, dažniausiai apsistojama ties dviem – trim valdymo
    lygiais; pagerėja komunikacijos kanalai tarp sprendimus
    priimančių ir juos vykdančių.
  \item[Šerdies ir žiedo (branduolio ir periferijos)] struktūra:
    šerdį sudaro vadovai, atsakantys už veiklos strategiją, o žiede
    yra visi darbuotojai, įgyvendinantys strategiją; tokia organizacija
    yra dinamiška, joje nėra griežtai apibrėžtų hierarchinių lygmenų,
    nes darbas vyksta projektų realizavimo principu. Tokios struktūros
    organizacija labai lengvai prisitaiko prie reikiamų sąlygų.
\end{description}
